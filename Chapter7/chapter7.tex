\chapter{Results}
\label{ch:7}

% **************************** Define Graphics Path **************************
\ifpdf
    \graphicspath{{Chapter7/Figs/Raster/}{Chapter7/Figs/PDF/}{Chapter7/Figs/}}
\else
    \graphicspath{{Chapter7/Figs/Vector/}{Chapter7/Figs/}}
\fi


%********************************** % First Section  *************************************
\section{Likelihood Model}  %Section - 1.1 
\label{sec:results_likelihood}
layout the likelihood model used to perform the simultaneous fit to get the 
background numbers


\subsection{Notes on syst modelling}
lognormal instead of gaussian PDF - these are used for systematics modelling.

see:
\verb!http://www.physics.ucla.edu/~cousins/stats/cousins_lognormal_prior.pdf!

if you had a parameter with mean of 0.3 and variance of 0.1, you could have a 
distro which would give you negative values of your PDF, if you used gaussian. 
therefore we use lognormal.

The lognormal distribution has three variables as input. $Logn(x|\rho, \sigma)$

\begin{itemize}
\item $\sigma$ parameter - this is similar to the width, if it were a gaussian 
distribution. It is essentially the derived systematic error as taken from the 
closure tests (I think...!). The smaller the value, the more the $\rho$ 
parameters may need to be pulled in order to make the fit.
\item $\rho$ parameter - the mean of the syst error. this is a parameter
(nuisance?) which is allowed to float in the fit. It essentially represents how 
much you have to pull on your systematics in order for the observations to match
the background predictions. If it's pulled a lot (many sigma away from 1.) then 
your observations really don't fit your predictions, or you've really 
underestimated your systematics! Large deviations from 1 (in terms of the
$\sigma$ parameter) indicate an unhealthy fit...
\end{itemize}



%********************************** % Second Section  *************************************
\section{Fit Results}  %Section - 1.2
\label{sec:results_fit}

\subsection{Fit without Signal region}
results

\subsection{Fit with Signal region}
results

\subsection{Pulls and p-values}
no signal! background only
