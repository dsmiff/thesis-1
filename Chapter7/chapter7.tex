\chapter{Results}
\label{ch:7}

% **************************** Define Graphics Path **************************
\ifpdf
    \graphicspath{{Chapter7/Figs/Raster/}{Chapter7/Figs/PDF/}{Chapter7/Figs/}}
\else
    \graphicspath{{Chapter7/Figs/Vector/}{Chapter7/Figs/}}
\fi


%********************************** % First Section  *************************************
\section{Likelihood Model}  %Section - 1.1 
\label{sec:results_likelihood}

For a given category of \nb and \nj, let $n_i$ be the observed number of events 
following full selection in the $i^{th}$ of $N$ \HT bins. The likelihood is 
therefore constructed as:

\begin{equation}
L_{hadronic} = \prod_i \text{Pois}(n^i | b^i + s^i)
\end{equation}

where $b^i$ and $s^i$ represent the expected number of background events from SM 
processes and the expected number of signal events in the bin $i$, and Pois is 
the Poisson distribution:

\begin{equation}
f(k;\lambda) = \frac{1}{k!}\lambda^k e^{-\lambda}
\end{equation}

The background is considered to be entirely EWK in origin ($b^i = \text{EWK}^i$), as QCD
is made negligible, which in turn can be deconstructed using the relative fraction
of \zinv events, \fzinv as:

\begin{equation}
\text{Z}_{inv}^i = \fzinv \times \text{EWK}^i
\end{equation}
\begin{equation}
\text{ttW}^i = (1-\fzinv) \times \text{EWK}^i
\end{equation}

where $\text{EWK}^i$, $\text{Z}_{\text{inv}}^i$ and $\text{ttW}^i$ are the number of 
expected events in the $i^{th}$ bin from the total electroweak background, \zinv
and SM W boson production and top quark decays, respectively.

These backgrounds are predicted using sideband control samples and transfer 
factors (chapter~\ref{ch:6}). Let $n_{ph}^i$, $n_{\mu}^i$ and $n_{\mu\mu}^i$ be 
the observed event counts in the \gj, \mj and \mmj control samples, 
respectively, which corresponding yields in MC: $MC_{ph}^i$, $MC_{\mu}^i$ and
$MC_{\mu\mu}^i$. Transfer factors are defined as the inverse as those of the 
analysis:

\begin{equation}
r_{ph}^i = \frac{MC_{ph}^i}{MC_{\text{Z}_{\text{inv}}}^i}
\end{equation}

\subsection{Notes on syst modelling}
lognormal instead of gaussian PDF - these are used for systematics modelling.

see:
\verb!http://www.physics.ucla.edu/~cousins/stats/cousins_lognormal_prior.pdf!

if you had a parameter with mean of 0.3 and variance of 0.1, you could have a 
distro which would give you negative values of your PDF, if you used gaussian. 
therefore we use lognormal.

The lognormal distribution has three variables as input. $Logn(x|\rho, \sigma)$

\begin{description}
\item[$\sigma$ parameter]\hfill \\ this is similar to the width, if it were a gaussian 
distribution. It is essentially the derived systematic error as taken from the 
closure tests (I think...!). The smaller the value, the more the $\rho$ 
parameters may need to be pulled in order to make the fit.
\item[$\rho$ parameter] \hfill \\ the mean of the syst error. this is a parameter
(nuisance?) which is allowed to float in the fit. It essentially represents how 
much you have to pull on your systematics in order for the observations to match
the background predictions. If it's pulled a lot (many sigma away from 1.) then 
your observations really don't fit your predictions, or you've really 
underestimated your systematics! Large deviations from 1 (in terms of the
$\sigma$ parameter) indicate an unhealthy fit...
\end{description}



%********************************** % Second Section  *************************************
\section{Fit Results}  %Section - 1.2
\label{sec:results_fit}

\subsection{Fit without Signal region}
results

\subsection{Fit with Signal region}
results

\subsection{Pulls and p-values}
no signal! background only
