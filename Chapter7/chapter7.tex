\chapter{Results}
\label{ch:7}

% **************************** Define Graphics Path **************************
\ifpdf
    \graphicspath{{Chapter7/Figs/Raster/}{Chapter7/Figs/PDF/}{Chapter7/Figs/}}
\else
    \graphicspath{{Chapter7/Figs/Vector/}{Chapter7/Figs/}}
\fi


%********************************** % First Section  *************************************
\section{Likelihood Model}  %Section - 1.1 
\label{sec:results_likelihood}
The likelihood model is defined for each analysis category of \nb and \nj.

\subsection{Hadronic Signal Region}

Let $n^i$ be the observed number of events 
following full selection in the $i^{th}$ of $N$ bins of \HT. The likelihood is 
constructed as:

\begin{equation}
L_{hadronic} = \prod_i \text{Pois}(n^i | b^i + s^i),
\end{equation}

where $b^i$ and $s^i$ represent the expected number of background events from SM 
processes and the expected number of signal events in the bin $i$. Pois is 
the Poisson distribution:

\begin{equation}
f(k;\lambda) = \frac{1}{k!}\lambda^k e^{-\lambda}.
\end{equation}


\subsection{Electroweak Background Contribution}
The background is considered to be entirely EWK in origin ($b^i = \text{EWK}^i$), as QCD
is made negligible, so can be deconstructed using the relative fraction
of \zinv events, \fzinv as:

\begin{equation}
\text{Z}_{inv}^i = \fzinv \times \text{EWK}^i,
\label{eq:ewk_frac_z}
\end{equation}
\begin{equation}
\text{ttW}^i = (1-\fzinv) \times \text{EWK}^i,
\label{eq:ewk_frac_ttw}
\end{equation}

where $\text{EWK}^i$ is number of expected events from the total EWK background,
$\text{Z}_{\text{inv}}^i$ is the number of expected events from the \zinv 
contribution and $\text{ttW}^i$ is the number of expect events from the W boson 
production and top quark decay contribution, all in the $i^{th}$ bin. The variable 
\fzinv is allowed to float between 0 and 1.

EWK backgrounds are predicted using sideband control samples and transfer 
factors (chapter~\ref{ch:6}). Let $n_{\gamma}^i$, $n_{\mu}^i$ and $n_{\mu\mu}^i$ be 
the observed event counts in the \gj, \mj and \mmj control samples, 
respectively, with corresponding yields in MC: \mcp, \mcm, \mcmm. These are
further seperated into seperate contributions from \zinv,
\mczinv, and \text{ttW}, \mcttw. Transfer
factors are defined as the inverse as those of the analysis:

\begin{equation}
r_{\gamma}^i = \frac{\mcp}{\mczinv};\;\;\;
r_{\mu\mu}^i = \frac{\mcmm}{\mczinv};\;\;\;
r_{\mu}^i = \frac{\mcm}{\mcttw},
\end{equation}

and so likelihoods are written as:

\begin{equation}
L_{\gamma} = \prod_i \text{Pois}(n_{\gamma}^i | \rhopz \cdot r^i_{\gamma} \cdot Z^i_{\text{inv}}),
\label{eq:lterm_pho}
\end{equation}
\begin{equation}
L_{\mu\mu} = \prod_i \text{Pois}(n_{\mu\mu}^i | \rhommz \cdot r^i_{\mu\mu} \cdot Z^i_{\text{inv}}),
\label{eq:lterm_mumu}
\end{equation}
\begin{equation}
L_{\mu} = \prod_i \text{Pois}(n_{\mu}^i | \rhomw \cdot r^i_{\mu} \cdot ttW^i + s_{\mu}^i).
\label{eq:lterm_mu}
\end{equation}

Both equations~\ref{eq:lterm_pho} and \ref{eq:lterm_mumu} are used to estimate the 
maximum likelihood value for $Z^i_{\text{inv}}$, alongside equation~\ref{eq:lterm_mu} 
for $ttW^i$, all considered simulataneously through the relationships defined in
equations~\ref{eq:ewk_frac_z} and \ref{eq:ewk_frac_ttw}.

The terms \rhopz, \rhommz and \rhomw are correction factors accomodating the
systematic uncertainties associated with the control sample based background 
predictions. The relative uncertainties are derived in
section~\ref{sec:closure_tests} and represented by the terms \sigmapz, \sigmammz
and \sigmamw, their 
values summarised in table~\ref{tab:syst_values}. Systematics enter the total
likelihood as:

\begin{equation}
L_{\gamma\text{ syst.}} = \prod_j \text{Logn}(1.0 | \rhopz, \sigmapz),
\end{equation}
\begin{equation}
L_{\mu\mu\text{ syst.}} = \prod_j \text{Logn}(1.0 | \rhommz, \sigmammz),
\end{equation}
\begin{equation}
L_{\mu\text{ syst.}} = \prod_j \text{Logn}(1.0 | \rhomw, \sigmamw),
\end{equation}

where Logn is the log-normal distribution (as recommended by \cite{cousins-log-normal}):

\begin{equation}
\text{Logn}(x|\mu, \sigma) = \frac{1}{x\sqrt{2\pi}\text{ln}k} exp \Bigg(-\frac{\text{ln}^2 \big(\frac{x}{\mu}\big)}{2\text{ln}^2k}\Bigg); k = 1+\sigma
\end{equation}

These terms are combined as:

\begin{equation}
L_{EWK\text{ syst.}} = L_{\gamma\text{ syst.}} \times L_{\mu\mu\text{ syst.}} \times L_{\mu\text{ syst.}}.
\end{equation}

For $\nb \geq 2$ categories, the \mj control sample is used to predict \zinv and
\ttw combined, giving:

\begin{equation}
{r'}^i_{\mu} = \frac{\mcm}{MC^i_{\ttbar + W + Z_{\text{inv}}}};
\end{equation}

\begin{equation}
L_{\mu} = \prod_i \text{Pois}(n_{\mu}^i | \rhomw \cdot {r'}_{\mu}^i \cdot {EWK}^i + s_{\mu}^i).
\end{equation}

Terms corresponding to the \mmj and \gj samples are subsequently dropped.

\subsection{Signal Contribution}

Let $x$ be the cross section of the signal model under test, which can be varied
according to a multiplicative factor $f$ (a.k.a. the ``mu-factor''), and $l$ be 
the luminosity of the relevant collected data sample. $\epsilon^i_{had}$ and
$\epsilon^i_{\mu}$ are the signal acceptances of the hadronic and muon 
selections, respectively, for the given signal model. Finally, let $\delta$ be 
the relative systematic uncertainty on that signal acceptance, and $\rho_{sig}$ 
be the corrective factor to the signal yield floated to accomodate this uncertainty. 
Therefore the yield of signal events for the hadronic sample, $s^i$, and the 
yield of signal in the muon sample (i.e. ``signal contamination''), $s^i_{\mu}$,
can be written as:

\begin{equation}
s^i = f\rho_{sig}xl\epsilon_{had}
\end{equation}
\begin{equation}
s^i_{\mu} = f\rho_{sig}xl\epsilon_{\mu}
\end{equation}

Furthermore, the signal systematic contribution to the likelihood is included as
the term:

\begin{equation}
L_{signal} = \text{Logn}(1.0 | \rho_{sig}, \delta)
\end{equation}

\subsection{Total Likelihood}

For a given analysis category $k$ (\nb, \nj), the total likelihood is 
constructed as:

\begin{equation}
L^k_{total} = L^k_{hadronic} \times L^k_{\mu} \times L^k_{\gamma} \times L^k_{\mu\mu} 
\times L^k_{\text{EWK syst.}}
\label{eq:total_likelihood}
\end{equation}

The number of nuisance paramters varies between different analysis categories, 
dependent on the number of \HT bins and control samples, summarised in
table~\ref{tab:nuisance_param_summary}.

\begin{table}[ht!]
  \caption{Summary of nuisance parameters.}
  \label{tab:nuisance_param_summary}
  \centering
  \footnotesize
  \begin{tabular}{ lll }
    \hline
    \hline
    Description                             & Categories    & Nuisance Parameters \\ [1.0ex]
    \hline
    \multirow{2}{*}{11 \HT bins, ($\mu, \mu\mu, \gamma$)}    & \multirow{2}{*}{\njlow/\njhigh, \nb = 0, 1}&
    $\{EWK^i, \fzinv\}_{i=0}^{10}, \{ \rhopz \}_{i=3}^{6},$\\
    && $\{ \rhommz, \rhomw \}_{i=0}^{6}$  \\
    8 \HT bins, ($\mu$)                     & \njlow/\njhigh, \nb = 2, 3, >4    & $\{EWK^i, \fzinv\}_{i=0}^{
    8}, \{\rhomwz \}_{i=0}^{6}$  \\
    3 \HT bins, ($\mu$)                     & \njlow/\njhigh, \nb = 2, 3, >4    & $\{EWK^i\}_{i=0}^{3},
    \{\rhomwz \}_{i=0}$\\
    \hline
    \hline
  \end{tabular}
\end{table}

When considering signal an additional term is introduced:

\begin{equation}
L = L_{signal} \times \prod_k L^k_{total}
\label{eq:total_likelihood_wsignal}
\end{equation}

% \subsection{Notes on syst modelling}
% lognormal instead of gaussian PDF - these are used for systematics modelling.

% see:
% \verb!http://www.physics.ucla.edu/~cousins/stats/cousins_lognormal_prior.pdf!

% if you had a parameter with mean of 0.3 and variance of 0.1, you could have a 
% distro which would give you negative values of your PDF, if you used gaussian. 
% therefore we use lognormal.

% The lognormal distribution has three variables as input. $Logn(x|\rho, \sigma)$

% \begin{description}
% \item[$\sigma$ parameter]\hfill \\ this is similar to the width, if it were a gaussian 
% distribution. It is essentially the derived systematic error as taken from the 
% closure tests (I think...!). The smaller the value, the more the $\rho$ 
% parameters may need to be pulled in order to make the fit.
% \item[$\rho$ parameter] \hfill \\ the mean of the syst error. this is a parameter
% (nuisance?) which is allowed to float in the fit. It essentially represents how 
% much you have to pull on your systematics in order for the observations to match
% the background predictions. If it's pulled a lot (many sigma away from 1.) then 
% your observations really don't fit your predictions, or you've really 
% underestimated your systematics! Large deviations from 1 (in terms of the
% $\sigma$ parameter) indicate an unhealthy fit...
% \end{description}



%********************************** % Second Section  *************************************
\section{Fit Results}  %Section - 1.2
\label{sec:results_fit}

\subsection{Fit without Signal region}
results

\subsection{Fit with Signal region}
results

\subsection{Pulls and p-values}
no signal! background only
