\chapter{Interpretation}
\label{ch:9}

% **************************** Define Graphics Path **************************
\ifpdf
    \graphicspath{{Chapter9/Figs/Raster/}{Chapter9/Figs/PDF/}{Chapter9/Figs/}}
\else
    \graphicspath{{Chapter9/Figs/Vector/}{Chapter9/Figs/}}
\fi

%********************************** % First Section  *************************************
\section{Signal Acceptance}  %Section - 1.1 
\label{sec:interpretation_acceptance}

Model acceptance is determined as a function of analysis category (\nb, \nj) and
\HT bin. This calculation is performed individually for each mass point in the
scan plane for both the hadronic selection and muon selection (to determine 
signal contamination in the control region). Interpretations are made using a
selected subset of the analysis 
categories, but an inclusive \HT selection. The analysis categories used are 
selected after an inspection of each for their significance given signal 
injection from specific mass points (REF).

\subsection{T2cc}
Signal efficiency times acceptance for the \texttt{T2cc} model is shown in 
figure~\ref{fig:sms-t2cc-sig}.

\begin{figure}[ht!]
  \centering
  \begin{subfigure}[b]{0.47\textwidth}
    \includegraphics[width=\textwidth]{Figs/sms/t2cc/v24/T2cc_v24_had_eff_maps_eq0b_le3j_SITV.pdf}
    \caption{Signal region, (2--3,0)}
    \label{fig:t2cc_sig_eff_le3j_0b}
  \end{subfigure}
  \begin{subfigure}[b]{0.47\textwidth}
    \includegraphics[width=\textwidth]{Figs/sms/t2cc/v24/T2cc_v24_muon_eff_maps_eq0b_le3j_SITV.pdf}
    \caption{\mj region, (2--3,0)}
    \label{fig:t2cc_mu_eff_le3j_0b}
  \end{subfigure} \\
  \begin{subfigure}[b]{0.47\textwidth}
    \includegraphics[width=\textwidth]{Figs/sms/t2cc/v24/T2cc_v24_had_eff_maps_eq1b_le3j_SITV.pdf}
    \caption{Signal region, (2--3,1)}
    \label{fig:t2cc_sig_eff_le3j_1b}
  \end{subfigure}
  \begin{subfigure}[b]{0.47\textwidth}
    \includegraphics[width=\textwidth]{Figs/sms/t2cc/v24/T2cc_v24_muon_eff_maps_eq1b_le3j_SITV.pdf}
    \caption{\mj region, (2--3,1)}
    \label{fig:t2cc_mu_eff_le3j_1b}
  \end{subfigure} \\
  \begin{subfigure}[b]{0.47\textwidth}
    \includegraphics[width=\textwidth]{Figs/sms/t2cc/v24/T2cc_v24_had_eff_maps_eq0b_ge4j_SITV.pdf}
    \caption{Signal region, ($\geq 4$,0)}
    \label{fig:t2cc_sig_eff_ge4j_0b}
  \end{subfigure}
  \begin{subfigure}[b]{0.47\textwidth}
    \includegraphics[width=\textwidth]{Figs/sms/t2cc/v24/T2cc_v24_muon_eff_maps_eq0b_ge4j_SITV.pdf}
    \caption{\mj region, ($\geq 4$,0)}
    \label{fig:t2cc_mu_eff_ge4j_0b}
  \end{subfigure} \\
  \begin{subfigure}[b]{0.47\textwidth}
    \includegraphics[width=\textwidth]{Figs/sms/t2cc/v24/T2cc_v24_had_eff_maps_eq1b_ge4j_SITV.pdf}
    \caption{Signal region, ($\geq 4$,1)}
    \label{fig:t2cc_sig_eff_ge4j_1b}
  \end{subfigure}
  \begin{subfigure}[b]{0.47\textwidth}
    \includegraphics[width=\textwidth]{Figs/sms/t2cc/v24/T2cc_v24_muon_eff_maps_eq1b_ge4j_SITV.pdf}
    \caption{\mj region, ($\geq 4$,1)}
    \label{fig:t2cc_mu_eff_ge4j_1b}
  \end{subfigure} \\
  \caption{Signal efficiency times acceptance for the \Ttwocc simplified, for 
  the hadronic selection (left) and the \mj selection (right), shown for the 
  four most sensitive analysis categories with an inclusive selection on \HT.}
  \label{fig:t2cc_eff}
\end{figure}

% \begin{figure}[h!]
%   \centering
%   \begin{subfigure}[b]{0.6\textwidth}
%     \includegraphics[width=\textwidth, trim=0 0 0 30, clip=true]{Figs/sms/t2cc/v24/T2cc_v24_sig_inj_250_170.pdf}
%     \caption{$m_{\sTop} = 250\gev, m_{\rm LSP} = 170\gev$}
%     \label{fig:t2cc_sig_inj_dm80}
%   \end{subfigure}
%   \begin{subfigure}[b]{0.6\textwidth}
%     \includegraphics[width=\textwidth, trim=0 0 0 30, clip=true]{Figs/sms/t2cc/v24/T2cc_v24_sig_inj_250_240.pdf}
%     \caption{$m_{\sTop} = 250\gev, m_{\rm LSP} = 240\gev$}
%     \label{fig:t2cc_sig_inj_dm10}
%   \end{subfigure}
%   \caption{balls}
%   \label{fig:}
% \end{figure}

%********************************** % Second Section  *************************************
\section{Systematic Uncertainties on Signal Acceptance }  %Section - 1.2
\label{sec:interpretation_uncertainties}

Fully reliant on MC for signal interpretations.
A number of different considerations

\subsection{Jet energy scale}
\subsection{Initial State Radiation}
\subsection{Btag scale factors}
\subsection{PDF}
\subsection{MHT/MET cleaning cut}
\subsection{Dead ECAL Filter}
\subsection{Shape systematics}
\subsection{Integrated Luminosity}
\subsection{Number of MC Partons}
\subsection{Any others?}
\subsection{Summary}

%********************************** % Third Section  *************************************
\section{Limits on models of Supersymmetry}  %Section - 1.3
\label{sec:interpretation_limits}

\subsection{Overview of limit setting procedure}
Using green and blue band fits for expected and observed
considering MC stats of signal MC samples
point-by-point systematics

discussion of CLs method. Look up `CLs 4 LHC'

\subsection{Limits}
different limits for the various models


%********************************** % Fourth Section  *************************************
\section{Interpretation of excesses}
\label{sec:interpretation_excess}
best fit points in T2cc
global context?
p-values with SUSY assumption? is that even a thing? who knows? who cares.