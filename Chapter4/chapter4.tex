\chapter{Object Definitions}

% **************************** Define Graphics Path **************************
\ifpdf
    \graphicspath{{Chapter4/Figs/Raster/}{Chapter4/Figs/PDF/}{Chapter4/Figs/}}
\else
    \graphicspath{{Chapter4/Figs/Vector/}{Chapter4/Figs/}}
\fi


%********************************** % First Section  *************************************
\section{Introduction}  %Section - 1.1 
\label{sec:objects_introduction}
What is an object in CMS?

Physics objects are observed as energy deposits in the various detector 
subsystems throughout the CMS detector. Specific definitions for the different 
objects are provided by the Physics Object Groups (POGs) and are used in this 
analysis. The follow sections describe the objects used.


%********************************** % Second Section  *************************************
\section{Jets}  %Section - 1.2
\label{sec:objects_jets}
calojets used exclusively in analysis

Jets are produced when a single quark or gluon is produced, subsequently 
hadronizing and showering throughout the calorimeter systems. Energy deposits in
the hadronic calorimeter are clustered using the anti$-k_T$ algorithm [ref] with
a 0.5 cone size parameter. The raw energy measurements can be affected by 
effects from overlapping pp collisions, known as pileup (PU) [first place 
mentioned?], therefore corrections are made to account for this. Clustered jets 
are also corrected to establish a uniform response in $\eta$ and pT [mathchange].
Table~\ref{tab:jet_id_loose} summarises the ID requirements used in this 
analysis, which correspond to the ``Loose'' working point [ref]. Jets are also 
corrected according to the L1FastJet, L2, L3, and L2L3Residual corrections [ref].


TABLE HERE!

\subsection{Tagging jets from b-quarks}
discuss the basic tagging algorithm used and working points

Jets originating from b-quarks can be identified using information from the 
vertex detector due to their increased lifetime with respect to light quarks. 
B-tagging algorithms are designed to determine the probability that a jet 
originates from a b-quark, given the jet's tracks displacement from the primary 
vertex (PV). The Combined Secondary Vertex (CSV) is used in this analysis, using
the ``Medium'' working point.

%********************************** % Third Section  *************************************
\section{Electrons}  %Section - 1.3
\label{sec:objects_electrons}


%********************************** % Fourth Section  *************************************
\section{Muons}  %Section - 1.4
\label{sec:objects_muons}


%********************************** % Fifth Section  *************************************
\section{Photons}  %Section - 1.5
\label{sec:objects_photons}


%********************************** % Sixth Section  *************************************
\section{Energy Sums}  %Section - 1.6
\label{sec:objects_energy_sums}

\subsection{ET}
\subsection{HT}
\subsection{MET}
\subsection{MHT}

%********************************** % Seventh Section  *************************************
\section{Single Isolated Tracks}  %Section - 1.7
\label{sec:objects_sit}
Selection criteria for a single isolated track - used later in the analysis for soft-lepton/
had tau rejection