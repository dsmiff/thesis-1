\chapter{Object Definitions}

% **************************** Define Graphics Path **************************
\ifpdf
    \graphicspath{{Chapter4/Figs/Raster/}{Chapter4/Figs/PDF/}{Chapter4/Figs/}}
\else
    \graphicspath{{Chapter4/Figs/Vector/}{Chapter4/Figs/}}
\fi


%********************************** % First Section  *************************************
\section{Introduction}  %Section - 1.1 
\label{sec:objects_introduction}

Physics objects are observed as energy deposits in the various detector 
subsystems throughout the CMS detector. Specific definitions for the different 
objects are provided by the Physics Object Groups (POGs) and are used in this 
analysis. The following sections describe the objects used.


%********************************** % Second Section  *************************************
\section{Jets}  %Section - 1.2
\label{sec:objects_jets}

Jets are produced when a single quark or gluon is produced, subsequently 
hadronizing and showering throughout the calorimeter systems. Energy deposits in
the hadronic calorimeter are clustered using the anti$-k_T$ algorithm [ref] with
a 0.5 cone size parameter. The raw energy measurements can be affected by 
effects from overlapping pp collisions, known as pileup (PU) [first place 
mentioned?], therefore corrections are made to account for this. Clustered jets 
are also corrected to establish a uniform response in $\eta$ and \Pt.
Table~\ref{tab:jet_id_loose} summarises the ID requirements used in this 
analysis, which correspond to the ``Loose'' working point [ref]. Jets are also 
corrected according to the L1FastJet, L2, L3, and L2L3Residual corrections [ref].

\begin{table}[!ht]
  \caption{Criteria for the ``loose'' jet ID working point.\label{tab:jet_id_loose}REWORD}
  \footnotesize
  \begin{center}
    \begin{tabular}{ll}
      \hline
      \hline
      Requirement                & Description                                                      \\
      \hline
      f$_{HPD} < 0.98$           & Fractional contribution from the ``hottest'' Hybrid Photo Diode. \\
      f$_{EM} > 0.01$            & Minimum electromagnetic fractional component.                    \\
      N$^{90}_{\rm hits} \geq$ 2 & Number of channels containing at least 90\% of total energy      \\
      \hline
      \hline
    \end{tabular}
  \end{center}
\end{table}


\subsection{Tagging jets from b-quarks}

Jets originating from b-quarks can be identified using information from the 
vertex detector due to their increased lifetime with respect to light-quarks (u,
d, s, c). 
B-tagging algorithms are designed to determine the probability that a jet 
originates from a b-quark, given, amongst other properties, the jet's tracks
displacement from the primary 
vertex (PV). Each algorithm
calculates a value used to discriminate between a jet from a b-quark and a
light-quark.

The Combined Secondary Vertex (CSV) algorithm is used in this analysis,
using the ``Medium'' working point, or a discriminate threshold of $>0.679$. This 
working point corresponds to a mistag rate for gluons/light-quarks of 1\%, and a
pT dependent efficiency in the range of 60-70\% (plot?) [refs].


%********************************** % Fourth Section  *************************************
\section{Muons}  %Section - 1.4
\label{sec:objects_muons}
Muons are identified using the muon POGs Tight working point definition, as 
summarised in table~\ref{tab:muon-id}. This definition is used for both muon 
selection in the \mj and \mmj control regions, and the muon veto in the hadronic
signal region.

\begin{table}[ht!]
  \caption{Muon identification (Tight working point).\label{tab:muon-id}}
  \centering
  \footnotesize
  \begin{tabular}{ lcp{8cm} }
    \hline
    \hline
    Varible & Requirment & Description \\
    \hline
    Global Muon                            & True      & Muon object 
    is reconstructed from both hits in the muon systems and matched hits in the 
    silicon tracker \\
    PFMuon                                 & True      & FIXME \\
    $\chi^{2} /ndof$ of fit                & $<10$     & Goodness of fit 
    of the global muon track fit. Suppresses hadronic punch-through and muons 
    decaying in flight.\\
    Muon chamber hits                      & $>0$      & At least 1 hit in a 
    muon chamber. Suppresses hadronic punch-through and muons 
    decaying in flight.\\
    Muon station hits                      & $>1$      & Muon hits in at least 2
    muon stations. Suppresses punch-through and accidental track-to-segment matches.
    (Also makes consistent with trigger muon requirements.) \\
    Transverse impact $d_{xy}$             & $<0.2mm$ & Tracker track is close 
    to PV (define?) in the x-y plane. Helps suppress cosmic ray muons and muons 
    from decays in flight. \\
    Longitudinal dist $d_{z}$              & $<0.5mm$ & Tracker track is close 
    to PV in z-direction. Suppresses muons from cosmic rays, decays in flight 
    and PU. \\
    Pixel hits                             & $>0$      & At least 1 pixel hit. 
    Suppresses muons from decays in flight. \\
    Track layer hits                       & $>5$      & Guarantees good \Pt 
    measurement. \\
    PF Isolation ($\Delta\beta$ corrected) & $<0.12$   & Particle Flow based 
    isolation, based on a cone size of $\Delta R < 0.4$, with ``$\Delta \beta$'' 
    PU corrections applied. \\
    \hline
    \hline
  \end{tabular}
\end{table}

LOOKUP: ``muons from decays in flight''

%********************************** % Fifth Section  *************************************
\section{Photons}  %Section - 1.5
\label{sec:objects_photons}

Photons defintions are made relative to their position in the detector, either 
in the barrel or the endcap, summarised in table~\ref{tab:photon-id-egamma}. 
This Tight working point ID is defined by the POG group and used for both photon
selection in the \gj control sample and as a veto in the hadronic signal region.
BRIEFLY MENTION THE PF-BASED RHO PU CORRECTIONS

\begin{table}[ht!]
  \caption{Photon identification (Tight working point).\label{tab:photon-id-egamma}}
  \centering
  \footnotesize
  \begin{tabular}{ cccp{4cm} }
    \hline
    \hline
    Categories                    & Barrel                             & EndCap 
    & Description                         \\
    \hline
    % Conversion safe electron veto & Yes                                & Yes &
% https://twiki.cern.ch/twiki/bin/viewauth/CMS/ConversionTools#Conversion_safe_electron_veto_fo  \\
    Single Tower H/E              & 0.05                               & 0.05                               
    & Ratio of energy deposited in the HCAL towers within $\Delta R<0.15$ of the ECAL 
    supercluster, and the ECAL supercluster itself. \\
    $\sigma_{i\eta i\eta}$        & 0.11                               & 0.31 & 
    FIXME \\
    &&&\multirow{5}{4cm}{PF-based isolation requirements to ensure no hadronic or electromagnetic 
    activity with a cone defined by $\Delta R < 0.3$. RHO CORRECTION}\\
    PF charged hadron isolation   & 0.70                               & 0.50                               & \\
    PF neutral hadron isolation   & 0.4 + 0.04 $\times$ $\Pt^{\gamma}$  & 1.5 + 0.04 $\times$ $\Pt^{\gamma}$&
    \\
    PF photon isolation           & 0.5 + 0.005 $\times$ $\Pt^{\gamma}$ & 1.0 + 0.005 $\times$ $\Pt^{\gamma}$& \\
    \\
    \hline
    \hline
  \end{tabular}
\end{table}

%********************************** % Third Section  *************************************
\section{Electrons}  %Section - 1.3
\label{sec:objects_electrons}
The Electron POGs Loose working point ID is used in this analysis to veto 
electrons from all areas of the analysis. This cut based identification is 
defined seperately for the barrel and endcap regions of the detector and is
summarised in table~\ref{tab:ele-id}.

\begin{table}[ht!]
  \caption{Electron identification (Loose working point).\label{tab:ele-id}}
  \centering
  \footnotesize
  \begin{tabular}{ lccp{8cm} }
    \hline
    \hline
    Categories                                               & Barrel    & EndCap    & 
    Description \\
    \hline
    $\Delta \eta_{In}$                                       & 0.007     & 0.009     & 
    The difference between the track and ECAL supercluser in the $\eta$ dimension. \\
    $\Delta \phi_{In}$                                       & 0.15      & 0.10      &
    The difference between the track and ECAL supercluser in the $\phi$ dimension. \\
    $\sigma_{i\eta i\eta}$                                   & 0.01      & 0.03      & 
    The cluster shape covariance of the ECAL supercluster in the $\eta$ dimension. \\
    H/E                                                      & 0.12      & 0.10      &
    Ratio of energy deposited in the HCAL towers within $\Delta R<0.15$ of the ECAL 
    supercluster, and the ECAL supercluster itself. \\
    d0 (vtx)                                                 & 0.02      & 0.02      &
    The [transverse] distance of the track from the PV. \\
    dZ (vtx)                                                 & 0.2       & 0.20      &
    The distance of the track from the PV in the $z$ dimension. \\
    $\lvert(1/E_{\textrm{ECAL}} - 1/p_{\textrm{trk}})\rvert$ & 0.05      & 0.05      &
    Comparison of the ECAL supercluster energy and the track \Pt. Suppresses low 
    \Pt fakes. \\
    PF relative isolation                                    & 0.15      & 0.15      &
    FIXME \\
    Vertex fit probability                                   & 10$^{-6}$ & 10$^{-6}$ &
    FIXME \\
    Missing hits                                             & 1         & 1         &
    FIXME \\
    \hline
    \hline
  \end{tabular}
\end{table}

%********************************** % Sixth Section  *************************************
\section{Energy Sums}  %Section - 1.6
\label{sec:objects_energy_sums}
Physics objects are combined in the form of kinematic variables known as energy 
sums. These are done on the fly in the analysis using identified objects, with
the exception of \met which is constructed from PF objects and subject to type-I
corrections (DEFINE).

The definitions of the energy sum variables are:

\begin{equation}
    \begin{split}
    \Et = \sum_{i}^{\textrm{all objects}} |\Ptvect_i|\\
    \met = -\big|\sum_{i}^{\textrm{all objects}} \Ptvect_i\big|\\
    \HT = \sum_{i}^{\textrm{jets}} |\Ptvect_i|\\
    \mht = -\big|\sum_{i}^{\textrm{jets}} \Ptvect_i\big|\\
    \end{split}
\label{eq:energy_sums}
\end{equation}

A full set of \met filters are defined by the MET-POG which account for various 
physics and detector effects which can give un-physical or spurious \met 
signals. EXPLICITLY LIST?

%********************************** % Seventh Section  *************************************
\section{Single Isolated Tracks}  %Section - 1.7
\label{sec:objects_sit}
Single Isolated Tracks are used to identify hadronically decaying $\tau$ leptons
and leptonically decaying W bosons where the corresonding lepton has been
unidentified for whatever reason. 
The selection is based on particle flow candidates and is summarised in
table~\ref{tab:sit-id}. The ID is used as a veto in all selections, with the 
selected `tag' lepton ignored in the \mj and \mmj control selections. The ID 
criteria are taken from SINGLE LEPTON SUSY ANALYSIS REFERENCE.

\begin{table}[ht!]
  \caption{Single Isolated Track identification.\label{tab:sit-id}}
  \centering
  \footnotesize
  \begin{tabular}{ lcp{8cm} }
    \hline
    \hline
    Varible & Requirment & Description \\
    \hline
    Charge                      & $\neq 0$      & Candidate is charged. \\
    Track \Pt                   & $> 10 \gev$   & Track transverse momentum. \\
    $\Delta z(track, PV)$       & $<0.05cm$     & Longitudinal distance of track 
    from primary vertex. \\
    Relative Track Isolation    & $<0.1$        & Isolation relative to other PF 
    candidate tracks in cone of $\Delta R <0.3$ \\
    \hline
    \hline
  \end{tabular}
\end{table}