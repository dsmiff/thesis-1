\chapter{Conclusion}
\label{ch:conclusion}

% **************************** Define Graphics Path **************************
\ifpdf
    \graphicspath{{Chapter10/Figs/Raster/}{Chapter10/Figs/PDF/}{Chapter10/Figs/}}
\else
    \graphicspath{{Chapter10/Figs/Vector/}{Chapter10/Figs/}}
\fi

% \emph{When I do actually write this, it's going to be really, really great.}

% % Please give me a phd. I've done all this work. I didn't just ride my bike.
% % I promise to not give being a `Dr' a bad name. I'll be a good one. I swear down.

% A search for Supersymmetry in the jets and missing energy final state has been
% presented.

% \emph{supersymmetry recap}

% \emph{how we searched for it}

% \emph{analysis - alphat, qcd reduction. correlates various variables, allowing
% for a sliding mht threshold...?}

% \emph{three dimensions}

% \emph{signal triggers populate the signal region}

% \emph{detailed study of any remaining residual QCD backgrounds}

% \emph{ewk predicted from sidebands. how this method works. adjoint control
% regions and transfer factors.}

% \emph{relevant systematics are determined from a statistically powerful suite
% of closure tests, each probing different facets of the analysis.}

% \emph{results are determined via a simultaneous fit using an extensive
% likelihood model}

% \emph{no statistically significant excesses were observed - everything
% compatible with fluctuations}

% \emph{interpretations made in terms of two different decay channels of the stop
% particle, relevant at low mass splittings (mass degenerate scenarios)}

% \emph{strong, competitive limits produced}

A search for Supersymmetry has been presented based on the full 19.7 \fb, 
$\sqrt{s} = 8$ \tev dataset collected with the CMS detector at the LHC. The
analysis looks for pair-produced gluinos or squarks decaying hadronically to
jets and missing transverse energy due to the presence of the weakly interacting
lightest supersymmetric particle. The search is performed in dimensions of \HT,
the total transverse hadronic energy, and the multiplicity of jets and b-tagged
jets.

Dominant hadronic backgrounds from mis-measured QCD events are reduced by
several orders of magnitude to a negligible level using the dimensionless
kinematic variable, \alphat. Following the full signal selection criteria,
remaining backgrounds consist of electroweak decays of
heavy bosons or top quarks and are predicted using an extrapolation technique
from dedicated control samples,
orthogonal to the signal region. A total background estimation is determined
from a full likelihood model fit simulatenously to the signal and control
regions. Systematic errors on the background prediction are determined using a
suite of statistically powerful closure tests designed probe the various
different aspects and extrapolations made in the analysis.

Observations in data statistically agree with the results of the background
prediction, and as such interpretations have been made with respect to two
simplified spectra SUSY models. This thesis focuses on compressed models, where
the pair-produced sparticle is near-degenerate with the LSP. In this region of
phase space two decays of the stop sparticle become particularly relevant: a
decay via a charm quark and the LSP, and a pair of fermions and the LSP. Limits
on the mass of the stop squark with a decay to the charm have been set of up to
275 GeV, and for the fermion pair decay of up to 250 GeV.

While many turned to \runone of the LHC expectant for obvious signs of SUSY,
Nature has yet to deliver. Instead the phase space in which SUSY may
reside has been greatly reduced.
However, while answering many questions about the possible
nature of what lies Beyond the Standard Model, the 8
\tev run also left many open. With the
large increase of centre of mass energy to 13 \tev, and a similarly large
rise in luminosity, \runtwo of the LHC will aim to address many of these
outstanding questions. Thanks to the Large Hadron Collider we find ourselves in
the historically familiar position - we are again standing at the edge of
the energy frontier, peering into the unknown. Whatever we see, it's going to be
exciting.
