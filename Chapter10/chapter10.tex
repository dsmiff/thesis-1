\chapter{Conclusion}
\label{ch:conclusion}

% **************************** Define Graphics Path **************************
\ifpdf
    \graphicspath{{Chapter10/Figs/Raster/}{Chapter10/Figs/PDF/}{Chapter10/Figs/}}
\else
    \graphicspath{{Chapter10/Figs/Vector/}{Chapter10/Figs/}}
\fi

A search for Supersymmetry has been presented based on the full 18.5~\fb, 
$\sqrt{s} = 8$~\tev `Parked' dataset from the CMS detector at the LHC.
This dataset was collected using
using experimental, low-threshold triggers to increase sensitivity to soft
decays of new physics. The
analysis searches for pair-produced gluinos or squarks decaying hadronically to
jets and missing transverse energy due to the presence of the weakly-interacting
lightest supersymmetric particle (LSP). The search is performed in dimensions of
the total transverse hadronic energy, and the multiplicity of jets and b-tagged
jets.

Dominant hadronic backgrounds from mismeasured QCD events are reduced by
several orders of magnitude to a negligible level using the dimensionless
kinematic variable \alphat. Following the full signal selection criteria,
remaining backgrounds consist of electroweak decays of
heavy bosons or top quarks and are predicted using an extrapolation technique
from dedicated control regions,
orthogonal to the signal region. A total background estimation is determined
from a likelihood model fit simultaneously considering both the signal and
control regions. Systematic errors on the background prediction are determined
using a suite of statistically powerful closure tests designed to probe the
many different aspects of the analysis.

Observations in data statistically agree with the results of the background
prediction, and as such interpretations have been made with respect to two
simplified spectra SUSY models. Given the increased sensitivity of the triggers
used, this thesis focuses on compressed models, where
the pair-produced sparticles are near-degenerate in mass with the LSP. In this
region of
phase space two decays of the stop sparticle become dominant: a
decay via a c quark and the LSP, and an off-shell W, b quark and the LSP.
Limits on the mass of the stop squark with a c quark decay have been set
of up to 275 GeV, with the limit remaining strong across the entire
mass-splitting
range - a feature currently unmatched by any other single analysis. Similarly
for the four-body decay, stop squark mass limits are set of up to 250 GeV, found
to be strongest when the neutralino is near-degenerate in mass with the stop.

While many turned to \runone of the LHC hoping for signs of SUSY,
Nature has yet to deliver. Instead the phase space in which SUSY may
reside has been greatly reduced. While answering many questions about
the possible nature of what lies Beyond the Standard Model, the 8
\tev run also left many open. With the large increase of centre of mass energy
to 13 \tev, and a similarly large rise in luminosity, \runtwo of the LHC will
aim to address many of these outstanding questions. Thanks to the Large Hadron
Collider we find ourselves in a historically familiar position - we are again
standing at the edge of the energy frontier, peering into the unknown. Whatever
we see, it's going to be exciting.
