\chapter{The AlphaT Analysis}

% **************************** Define Graphics Path **************************
\ifpdf
    \graphicspath{{Chapter5/Figs/Raster/}{Chapter5/Figs/PDF/}{Chapter5/Figs/}}
\else
    \graphicspath{{Chapter5/Figs/Vector/}{Chapter5/Figs/}}
\fi


%********************************** % First Section  *************************************
\section{Analysis Overview}  %Section - 1.1 
\label{sec:selection_analysis_overview}

Intro to the aim of the analysis. What we're looking for. ``generic''

mention data control samples

binned in HT, nb, nj


The \alphat analysis is a generic search for SUSY in the jets and 
\met channel. The aim of the analysis is to remain as inclusive as possible to a
wide range of potential SUSY signatures by covering as broad a region of phase 
space as possible. Events are binned in exclusive categories of \HT (define?), 
\nj and \nb, which allows for targetted interpretations across the vast array of
possible simplified model final states, while reducing background yields to a 
minimum.

Analyses searching in the jets and \met final state encounter significant 
backgrounds from SM sources of both genuine \met with associated jet production 
and fake \met, predominantly originating from jet mis-measurement. The \alphat 
analysis makes predictions of the former using a data-driven transfer factor 
technique to extrapolate yields from statistically independent control samples 
into the signal region. However the latter, sources of fake \met predominantly 
(entirely?) from QCD, are reduced to a negligible level using the kinematic 
variable \alphat.

The analysis itself, along with the background [prediction] techniques for both 
cases will be described in the following sections.


\subsection{The AlphaT kinematic variable}
Used to surpress QCD

leaves sources of genuine MET when >0.5 cut made

add plot of HT vs MHT and how it varies with alphaT?

plots of where events sit in alphaT vs X (not sure!) plane for different samples (QCD, signal etc)

basically motivate alphaT

QCD multijet events dominate the SM background in any jets and \met search, and 
so each analysis must find a way to account for it. Jet mis-measurements in a 
purely QCD multijet event can lead to non-negligible amounts of \mht, and will 
therefore pass the signal selection. 
Other analyses attempt
to measure the contribution with very small associated errors. This is however 
made very difficult given the hadronic environment of the LHC, and the lack of 
precise measurements and calculations of multijet cross sections. As an 
alternative approach, the goal of the 
\alphat analysis is to reduce QCD down to an entirely negligible level so that 
it can be removed entirely from the analysis. This is 
achieved with the novel kinematic variable, \alphat, defined for dijets as [REF]:

\begin{equation}
\alphat = \frac{E_T^{j_2}}{M_T}
\label{eq:alphat_dijet}
\end{equation}

where $E_T^{j_2}$ is the transverse energy of the less energetic jet and $M_T$ 
is the transverse mass of the dijet system, defined as (MAKE BRACKETS BIGGER!):

\begin{equation}
M_T = \sqrt{(\sum^2_{i=1}{E_T^{j_i}})^2 - (\sum^2_{i=1}{p_x^{j_i}})^2 - (\sum^2_{i=1}{p_y^{j_i}})^2}
\label{eq:mt}
\end{equation}

where $E_T^{j_i}$, $p_x^{j_i}$ and $p_y^{j_i}$ are the transverse energy and 
transverse momentum in the x and y plane, for the jet $j_i$.

A perfectly measured dijet that contains back to back jets in $\phi$ will have an
\alphat value of 0.5, whereas 
events containing \mht from jet mis-measurements will have values of $\alphat<0.5$.
Events
containing processed with sources of genuine \met, whether from SM or SUSY 
sources, will have values of $\alphat > 0.5$. 

The \alphat variable can be generalised to an n-jet case by constructing a 
pseudo-dijet event, constructing each pseudo-jet such that the difference in \HT
between the two pseudo-jets, \deltaHT, is minimised. \alphat then takes the 
form:

\begin{equation}
\alphat = \frac{1}{2} \times \frac{\HT-\deltaHT}{\sqrt{\HT^2 - \mht^2}} = 
\frac{1}{2} \times \frac{1-\frac{\deltaHT}{\HT}}{\sqrt{1 - \frac{\mht}{\HT}^2}}
\label{eq:alphat_njet}
\end{equation}

show some sweet plots about correlations between \mht and \deltaHT for QCD, 
signal, SM real MET etc.


%********************************** % First Section  *************************************
\section{Hadronic Signal Region}
\label{sec:selection_hadronic}
The signal region. Jets + MET selection.
% Introduce that QCD is a dominant background due to jet mis-measurement.

\subsection{SM Backgrounds}
\subsubsection{Genuine EWK MET}
\subsubsection{Fake MET}
already partly described above - perhaps move the meaty descriptions here?

can explain here the MHT/MET cut, and how that removes jets under threshold 
which can conspire to give significant met or mht whatever

\subsection{Selection Criteria}

\subsection{Signal Triggers}

Add emphasis on parked trigger, as it opens up phase space to compressed susy models

\subsection{Formula method}

tie in with the split into nb and nj categories - necessary as stats will reduce, so we use
this `trick' to reduce errors



\section{Datasets and MC samples}

can describe the different methods of MC generation