\chapter{The AlphaT Analysis}

% **************************** Define Graphics Path **************************
\ifpdf
    \graphicspath{{Chapter5/Figs/Raster/}{Chapter5/Figs/PDF/}{Chapter5/Figs/}}
\else
    \graphicspath{{Chapter5/Figs/Vector/}{Chapter5/Figs/}}
\fi


%********************************** % First Section  *************************************
\section{Analysis Overview}  %Section - 1.1 
\label{sec:selection_analysis_overview}

Intro to the aim of the analysis. What we're looking for. ``generic''

mention data control samples

binned in HT, nb, nj


The \alphat analysis is a generic search for SUSY in the jets and 
\met channel. The aim of the analysis is to remain as inclusive as possible to a
wide range of potential SUSY signatures by covering as broad a region of phase 
space as possible. Events are binned in exclusive categories of \HT (define?), 
\nj and \nb, which allows for targetted interpretations across the vast array of
possible simplified model final states, while reducing background yields to a 
minimum.

Analyses searching in the jets and \met final state encounter significant 
backgrounds from SM sources of both genuine \met with associated jet production 
and fake \met, predominantly originating from jet mis-measurement. The \alphat 
analysis makes predictions of the former using a data-driven transfer factor 
technique to extrapolate yields from statistically independent control samples 
into the signal region. However the latter, sources of fake \met predominantly 
(entirely?) from QCD, are reduced to a negligible level using the kinematic 
variable \alphat.

The analysis itself, along with the background [prediction] techniques for both 
cases will be described in the following sections.


\subsection{The AlphaT kinematic variable}
Used to surpress QCD

leaves sources of genuine MET when >0.5 cut made

add plot of HT vs MHT and how it varies with alphaT?

plots of where events sit in alphaT vs X (not sure!) plane for different samples (QCD, signal etc)

basically motivate alphaT

%********************************** % First Section  *************************************
\section{Hadronic Signal Region}
\label{sec:selection_hadronic}
The signal region. Jets + MET selection.
Introduce that QCD is a dominant background due to jet mis-measurement.

\subsection{Selection Criteria}

\subsection{Signal Triggers}

Add emphasis on parked trigger, as it opens up phase space to compressed susy models

\subsection{Formula method}

tie in with the split into nb and nj categories - necessary as stats will reduce, so we use
this `trick' to reduce errors

\subsection{SM Backgrounds}
\subsubsection{Genuine EWK MET}
\subsubsection{Fake MET}



\section{Datasets and MC samples}

can describe the different methods of MC generation