\chapter{The AlphaT Analysis}
\label{ch:5}

% **************************** Define Graphics Path **************************
\ifpdf
    \graphicspath{{Chapter5/Figs/Raster/}{Chapter5/Figs/PDF/}{Chapter5/Figs/}}
\else
    \graphicspath{{Chapter5/Figs/Vector/}{Chapter5/Figs/}}
\fi


%********************************** % First Section  *************************************
\section{Analysis Overview}  %Section - 1.1 
\label{sec:selection_analysis_overview}

Intro to the aim of the analysis. What we're looking for. ``generic''

mention data control samples

binned in HT, nb, nj


The \alphat analysis is a generic search for SUSY in the jets and 
\met channel. The aim of the analysis is to remain as inclusive as possible to a
wide range of potential SUSY signatures by covering as broad a region of phase 
space as possible. Events are binned in exclusive categories of \HT (define?), 
and the multiplicity of jets and b-tagged jets,  
\nj and \nb respectively. Such binning allows for targetted interpretations across
the vast array of
possible simplified model final states, while reducing background yields to a 
minimum.

Analyses searching in the jets and \met final state encounter significant 
backgrounds from SM sources of both genuine \met [with associated jet production] 
and fake \met, predominantly originating from jet mis-measurement. The \alphat 
analysis makes predictions for the former using a data-driven transfer factor 
technique to extrapolate yields from statistically independent control samples 
into the signal region. However the latter, sources of fake \met predominantly 
(entirely?) from QCD, are reduced to a negligible level using the kinematic 
variable \alphat.

The main analysis signal region is described in the remainder of this chapter,
with background prediction techniques described in detail later in
chapter~\ref{ch:6}.


\subsection{The AlphaT kinematic variable}
Used to surpress QCD

leaves sources of genuine MET when >0.5 cut made

add plot of HT vs MHT and how it varies with alphaT?

plots of where events sit in alphaT vs X (not sure!) plane for different samples (QCD, signal etc)

basically motivate alphaT

QCD multijet events dominate the SM background in any jets and \met search, and 
so each analysis must find a way to account for it. Jet mis-measurements in a 
purely QCD multijet event can lead to non-negligible amounts of \mht, and will 
therefore pass the signal selection. 
Other analyses attempt
to measure the contribution with very small associated errors. This is however 
made very difficult given the hadronic environment of the LHC, and the lack of 
precise measurements and calculations of multijet cross sections. As an 
alternative approach, the goal of the 
\alphat analysis is to reduce QCD down to an entirely negligible level so that 
it can be removed entirely from the analysis. This is 
achieved with the novel kinematic variable, \alphat, defined for dijets as [REF]:

\begin{equation}
\alphat = \frac{E_T^{j_2}}{M_T}
\label{eq:alphat_dijet}
\end{equation}

where $E_T^{j_2}$ is the transverse energy of the less energetic jet and $M_T$ 
is the transverse mass of the dijet system, defined as (MAKE BRACKETS BIGGER!):

\begin{equation}
M_T = \sqrt{(\sum^2_{i=1}{E_T^{j_i}})^2 - (\sum^2_{i=1}{p_x^{j_i}})^2 - (\sum^2_{i=1}{p_y^{j_i}})^2}
\label{eq:mt}
\end{equation}

where $E_T^{j_i}$, $p_x^{j_i}$ and $p_y^{j_i}$ are the transverse energy and 
transverse momentum in the x and y plane, for the jet $j_i$.

A perfectly measured dijet that contains back to back jets in $\phi$ will have an
\alphat value of 0.5, whereas 
events containing \mht from jet mis-measurements will have values of $\alphat<0.5$.
Events
containing processed with sources of genuine \met, whether from SM or SUSY 
sources, will have values of $\alphat > 0.5$. 

The \alphat variable can be generalised to an n-jet case by constructing a 
pseudo-dijet event, constructing each pseudo-jet such that the difference in \HT
between the two pseudo-jets, \deltaHT, is minimised. \alphat then takes the 
form:

\begin{equation}
\alphat = \frac{1}{2} \times \frac{\HT-\deltaHT}{\sqrt{\HT^2 - \mht^2}} = 
\frac{1}{2} \times \frac{1-\frac{\deltaHT}{\HT}}{\sqrt{1 - \frac{\mht}{\HT}^2}}
\label{eq:alphat_njet}
\end{equation}

show some sweet plots about correlations between \mht and \deltaHT for QCD, 
signal, SM real MET etc.


%********************************** % First Section  *************************************
\section{Hadronic Signal Region}
\label{sec:selection_hadronic}
The signal region. Jets + MET selection.
% Introduce that QCD is a dominant background due to jet mis-measurement.

This section outlines the basic hadronic pre-selection made for the signal 
region considered, then going on to describe the relevant background sources and
the trigger considerations required.

\subsection{Selection Criteria}
\label{sec:selec_crit}
As introduced earlier, the signature considered by this analysis is that of 
mutliple jets and \met. Therefore, in the so-called `hadronic' signal region we 
require at least jets and that this topology has a value of \alphat greater than
at least 0.55. While no absolute \met requirement is made, the requirement on 
\alphat imposes an implied requirement, which actually allows us to maintain 
sensitivity to much lower regions in \met than other analyses are able to 
considered. Events are required to have at least $\HT > 200 \gev$ so significant
hadronic activity is present.

We veto the presence of any leptonic or 
photonic activity to ensure purely hadronic events are considered, thereby 
surpressing events with genuine \met from leptonic decays to neutrinos.

Following this basic preselection events are further categorized into the three 
binning dimenions of \HT, \nb and \nj, as shown in table~\ref{tab:ht-bins}. Two 
jet multiplicity categories are defined as 2 or 3 jets, and 4 or more. B-tagged 
jet multiplicities are exactly 0, 1, 2, 3 or 4 and more. Combinations of these 
various categories are then further sub-divided into multiple \HT bins ranging 
from 200\gev to an inclusive bin of $\HT>1075\gev$.

\begin{table}[h!]
  \caption{(\nj,\nb) event categories and \HT binning scheme.\label{tab:ht-bins}}
  \centering
  % \scriptsize % too big to fit on page
  \tiny % make it small enough to fit!
  \begin{tabular}{ lrrrrrrrrrrr }
    \hline
    \hline
    (\nj,\nb)       & \multicolumn{11}{c}{\HT bins (\gev)}                                                                                \\
    \hline
    (2-3,0)           & 200--275 & 275--325 & 325--375 & 375--475 & 475--575 & 575--675 & 675--775 & 775--875 & 875--975 & 975--1075 & $>$1075  \\
    (2-3,1)           & 200--275 & 275--325 & 325--375 & 375--475 & 475--575 & 575--675 & 675--775 & 775--875 & 875--975 & 975--1075 & $>$1075  \\
    (2-3,2)           & 200--275 & 275--325 & 325--375 & 375--475 & 475--575 & 575--675 & 675--775 & 775--875 & $>$875   & \multicolumn{2}{c}{} \\
%    (2-3,3)          & 200--275 & 275--325 & 325--375 & 375--475 & 475--575 & 575--675 & 675--775 & 775--875 & $>$875   & \multicolumn{2}{c}{} \\
    ($\geq$4,0)       & 200--275 & 275--325 & 325--375 & 375--475 & 475--575 & 575--675 & 675--775 & 775--875 & 875--975 & 975--1075 & $>$1075  \\
    ($\geq$4,1)       & 200--275 & 275--325 & 325--375 & 375--475 & 475--575 & 575--675 & 675--775 & 775--875 & 875--975 & 975--1075 & $>$1075  \\
    ($\geq$4,2)       & 200--275 & 275--325 & 325--375 & 375--475 & 475--575 & 575--675 & 675--775 & 775--875 & $>$875   & \multicolumn{2}{c}{} \\
    ($\geq$4,3)       & 200--275 & 275--325 & 325--375 & 375--475 & 475--575 & 575--675 & 675--775 & 775--875 & $>$875   & \multicolumn{2}{c}{} \\
    ($\geq$4,$\geq$4) & 200--275 & 275--325 & 325--375 & $>$375   & \multicolumn{7}{c}{}                                                        \\
    \hline
    \hline
  \end{tabular}
\end{table}

For bins of $\HT>375\gev$ the leading two jets in the event are required to 
have $\Pt>100\gev$, with all additional jets having half the requirement of
$\Pt>50\gev$. In order to maintain a similar kinematic phase space throughout
the many \HT bins, these jet \Pt requirements are scaled for the lower \HT bins 
as shown in Table~\ref{tab:jet-pt-thresholds}.

\begin{table}[h!]
  \caption{Jet \Et thresholds per \HT bin.\label{tab:jet-pt-thresholds}}
  \centering
  \footnotesize
  \begin{tabular}{ lcccc }
    \hline
    \hline
    \HT bin    & 200--275 & 275--325 & 325--375 & $>$375 \\
    \hline
    Lead jet       & 73.3     & 73.3     & 86.7     & 100.0  \\
    Second jet     & 73.3     & 73.3     & 86.7     & 100.0  \\
    All other jets & 36.7     & 36.7     & 43.3     & 50.0   \\
    \hline
    \hline
  \end{tabular}
\end{table}

\subsection{SM Backgrounds}
Following the generic hadronic signal region selection criteria described in 
Section~\ref{sec:selec_crit}, remaining SM backgrounds are divided into two 
categories, as described in the following sections.

\subsubsection{Genuine EWK MET}
While the lepton and photon vetoes employed in the signal region surpress 
significant amounts of background from events with neutrinos, such events can 
still persist if, for example, the lepton is not identified.
Examples of such processes are \ttbar or W-boson production, where the 
W decays via \wlnu. If the lepton is `lost' and therefore evades our 
lepton vetoes significant missing energy can be produced not only from missing 
the lepton, but with the presence of the neutrino. When such processed are accompanied 
by associated jet production they are able to pass our signal region selection.

Leptons can be `lost' for a variety differnt reasons, but ultimately for failing
our lepton ID criteria. This can be for a number of different reasons, with the 
most prevelant being soft-leptons below ID threshold or non-isolated leptons 
which pass the ID quality cuts but fail the isolation requirement.

SHOW TABLE WITH THE BACKGROUND BREAKDOWN - PUT INTO GRAPHICAL FORM?

Lost-lepton backgrounds can be reduced by vetoing events contain a single 
isolated track (SIT) in the tracker, as defined in Section~\ref{objects_sit}. TALK 
MORE ABOUT THIS CUT!

The dominant EWK source of genuine missing energy comes from a produced Z decaying
via neutrinos, \zinv, with associated jet production.

Following the reduction of such processes using the lepton, photon and SIT vetoes,
any remaining contributions from SM EWK backgrounds are estimated using a 
fully data-driven transfer factor technique, described in detail in Section~\ref
{sec:background_overview}.

what about photons? do they contribute??

\subsubsection{Fake MET}
already partly described above - perhaps move the meaty descriptions here?

can explain here the MHT/MET cut, and how that removes jets under threshold 
which can conspire to give significant met or mht whatever

1. What is the background?
  - how does it manifest itself in our signal region?
2. How do we reduce it?
  - MHT/MET
    - jets under threshold conspiring to make MET
  - alphaT
    - significant jet mis-measurements - stochastic
3. Any left?
- once alphaT cut has been made, QCD is considered entirely negligible.

describe the various chosen alphat thresholds
  - do I include the MHT/MET QCD prediction method?

\subsection{Signal Triggers}

Add emphasis on parked trigger, as it opens up phase space to compressed susy models

Events are collected at the high-level trigger (HLT - may already be declared in
the detector section!) system using a dedicated suite of
signal triggers for our analysis. In order for an event to pass the trigger's 
requirements, it must exceed both a \HT and an \alphat threshold. Trigger rate 
can be maintained by varying the
threshold requirement on each of these independent `legs', as shown in Table~
\ref{tab:sig_trigs}.

\begin{table}[h]
\begin{tabular}{l|l}
HT bin (\gev) & HLT Trigger            \\ \hline
200-275       & HLT\_HT200\_AlphaT0p57 \\
275-325       & HLT\_HT200\_AlphaT0p57 \\
325-375       & HLT\_HT300\_AlphaT0p53 \\
375-475       & HLT\_HT350\_AlphaT0p52 \\
475-575       & HLT\_HT400\_AlphaT0p51 \\
575-675       & HLT\_HT400\_AlphaT0p51 \\
675-775       & HLT\_HT400\_AlphaT0p51 \\
775-875       & HLT\_HT400\_AlphaT0p51 \\
875-975       & HLT\_HT400\_AlphaT0p51 \\
975-1075      & HLT\_HT400\_AlphaT0p51 \\
1075-$\inf$   & HLT\_HT400\_AlphaT0p51 \\                       
\end{tabular}
\label{tab:sig_trigs}
\end{table}

- L1 seed
- say absolute rate of triggers

All triggers were present throughout 2012 (is 8\tev just 2012?), however the 
\verb!HLT_HT200_AlphaT0p57! trigger was used as part of the `Parked' stream of 
data which was reconstructed at a later data, following the active data-taking 
period.

Trigger efficiencies are measured using a muon reference trigger, \verb!
HLT_IsoMu24_eta2p1!, using a muon tag and probe method (EXPAND?). An efficiency 
is determined for each analysis category and for each trigger.

- show trig effs table and maybe some plots?


\subsection{Formula method}

tie in with the split into nb and nj categories - necessary as stats will reduce, so we use
this `trick' to reduce errors



\section{Datasets and MC samples}

can describe the different methods of MC generation