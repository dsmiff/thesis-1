\chapter{The \alphat Analysis}
\label{ch:5}

% **************************** Define Graphics Path **************************
\ifpdf
    \graphicspath{{Chapter5/Figs/Raster/}{Chapter5/Figs/PDF/}{Chapter5/Figs/}}
\else
    \graphicspath{{Chapter5/Figs/Vector/}{Chapter5/Figs/}}
\fi


%********************************** % First Section  *************************************
\section{Analysis Overview}  %Section - 1.1 
\label{sec:selection_analysis_overview}

Analyses searching in the jets and \met final state encounter significant 
backgrounds from SM sources of both genuine and fake \met. Genuine \met 
originates from W and Z boson production, decaying with one or more neutrinos in
the final state. In this analysis background predictions are made for these
processes using 
extrapolations into the signal region from independent, process-specific control
samples, as described in chapter~\ref{ch:6}.
Fake \met predominantly is found due to mismeasurements of QCD 
multijet (MJ) events, which is the dominant process in the hadronic environment
and 
phase space considered in this analysis due to it's very large cross section. 
Small inconsistencies in the handling of such large backgrounds or rare detector
effects can therefore 
have a significant impact on an analysis. This background is reduced to an 
entirely negligible level using the dimensionless kinematic variable, \alphat
(described in section~\ref{sec:alphat}).

% OLD
% Analyses searching in the jets and \met final state encounter significant 
% backgrounds from SM sources of both genuine \met, originating from the leptonic 
% decays of W bosons, 
% and fake \met, originating from jet mis-measurement. The \alphat 
% analysis makes predictions for the former using a data-driven transfer factor 
% technique to extrapolate yields from statistically independent control samples 
% into the signal region. However the latter, sources of fake \met from QCD multijet events,
% are reduced to a negligible level using the kinematic 
% variable \alphat.

In this analysis events are binned in exclusive categories of \HT, 
and the multiplicity of jets and b-tagged jets,  
\nj and \nb, as summarised in table~\ref{tab:ht-bins}. Such binning allows for
targeted interpretations across the vast array of
possible simplified model final states, while reducing background yields to a 
minimum.

\begin{table}[ht!]
  \caption{\HT bin lower bounds used for each \nj and \nb category.\label
  {tab:ht-bins}}
  \centering
  \small
  \begin{tabular}{ lrrrrrrrrrrr }
    \hline
    \hline
    (\nj,\nb)       & \multicolumn{11}{c}{\HT bins (\gev)}                                                                                \\
    \hline
    (2-3,0)           & 200 & 275 & 325 & 375 & 475 & 575 & 675 & 775 & 875 & 975 & $>$1075  \\
    (2-3,1)           & 200 & 275 & 325 & 375 & 475 & 575 & 675 & 775 & 875 & 975 & $>$1075  \\
    (2-3,2)           & 200 & 275 & 325 & 375 & 475 & 575 & 675 & 775 & $>$875   & \multicolumn{2}{c}{} \\
%    (2-3,3)          & 200 & 275 & 325 & 375 & 475 & 575 & 675 & 775 & $>$875   & \multicolumn{2}{c}{} \\
    ($\geq$4,0)       & 200 & 275 & 325 & 375 & 475 & 575 & 675 & 775 & 875 & 975 & $>$1075  \\
    ($\geq$4,1)       & 200 & 275 & 325 & 375 & 475 & 575 & 675 & 775 & 875 & 975 & $>$1075  \\
    ($\geq$4,2)       & 200 & 275 & 325 & 375 & 475 & 575 & 675 & 775 & $>$875   & \multicolumn{2}{c}{} \\
    ($\geq$4,3)       & 200 & 275 & 325 & 375 & 475 & 575 & 675 & 775 & $>$875   & \multicolumn{2}{c}{} \\
    ($\geq$4,$\geq$4) & 200 & 275 & 325 & $>$375   & \multicolumn{7}{c}{}                                                        \\
    \hline
    \hline
  \end{tabular}
\end{table}

The main analysis signal region is described in the remainder of this chapter.


\subsection{The \alphat kinematic variable}
\label{sec:alphat}

% QCD multijet (MJ) events dominate the SM background in any search with multiple jets 
% in the final state. Jet energy mis-measurements in a 
% purely QCD MJ event can lead to non-negligible amounts of \mht, therefore 
% passing the signal selection.
Attempting to accurately measure the QCD contribution to the total background is
made very difficult given the hadronic environment of the LHC, and the lack of 
precise measurements and calculations of the large multijet cross sections. As an 
alternative approach, the goal of this analysis is to reduce QCD down to an
entirely negligible level through the use of the dimensionless kinematic
variable, \alphat. The di-jet variable
$\alpha$ was first proposed by Randall et al. in 2008 \cite{Randall:2008rw},
and later translated into the transverse plane for use with LHC analyses
\cite{CMS:2008vya, CMS-PAS-SUS-09-001}.
\alphat is defined for di-jets as:
% 
\begin{equation}
\alphat = \frac{ \sqrt{E_T^{j_2}/E_T^{j_1}} }{ \sqrt{2(1-cos(\Delta \phi))} }
\label{eq:alphat_di-jet}
\end{equation}
% 
where $E_T^{j_1}$ and $E_T^{j_2}$ are the reconstructed transverse energies of 
the first and second jets respectively, and $\Delta \phi$ is the seperation 
between the two jets in the $\phi$ plane.

A perfectly measured di-jet event containing back to back jets in $\phi$ of equal energy will
have an \alphat value of 0.5, whereas 
events with \met originating from jet energy mismeasurements will have values of $\alphat<0.5$.
Only events containing sources of genuine \met, whether from SM or BSM sources,
can have values of $\alphat > 0.5$.
% 
% \begin{equation}
% \alphat = \frac{E_T^{j_2}}{M_T} = \frac{\sqrt{}}
% \label{eq:alphat_di-jet}
% \end{equation}
% 
% where $E_T^{j_2}$ is the transverse energy of the less energetic jet and $M_T$ 
% is the transverse mass of the di-jet system, defined as:
% 
% \begin{equation}
% M_T = \sqrt{\bigg(\sum^2_{i=1}{E_T^{j_i}}\bigg)^2 - \bigg(\sum^2_{i=1}{p_x^{j_i}}\bigg)^2 - \bigg(\sum^2_{i=1}{p_y^{j_i}}\bigg)^2}
% \label{eq:mt}
% \end{equation}
% 
% where $E_T^{j_i}$, $p_x^{j_i}$ and $p_y^{j_i}$ are the transverse energy and 
% transverse momentum in the $x$ and $y$ planes, for the jet $j_i$.

The \alphat variable can be generalised to an n-jet case by considering the event as a 
pseudo-di-jet system, constructing each pseudo-jet such that the difference in \HT
between each pseudo-jet system, \deltaHT, is minimised. \alphat then takes the 
form:
% 
\begin{equation}
\alphat = \frac{1}{2} \times \frac{\HT-\deltaHT}{\sqrt{\HT^2 - \mht^2}} = 
\frac{1}{2} \times \frac{1-\frac{\deltaHT}{\HT}}{\sqrt{1 - \big(\frac{\mht}
{\HT}\big)^2}}.
\label{eq:alphat_njet}
\end{equation}
% 

The \alphat variable introduces a correlation between the two variables of \mht
and \deltaHT. This relationship is demonstrated in figure~\ref{fig:alphat_corr}
by the black contours of constant \alphat. The different physics sources
will populate various regions of this plane. Most notably QCD exhibits a strong
correlation between the two variables at low \alphat values
(figure~\ref{fig:alphat_corr_qcd}), while processes with real \met break this
correlation, sitting also at higher values of \alphat
(figures~\ref{fig:alphat_corr_zinv} and \ref{fig:alphat_corr_t2cc}).

\begin{figure}[h!]
  \centering
  \begin{subfigure}[b]{.46\textwidth}
    \includegraphics[width=\textwidth]{Figs/alphat/alphat_correlation_QCD.pdf}
    \caption{QCD}
    \label{fig:alphat_corr_qcd}
  \end{subfigure}
  \begin{subfigure}[b]{.46\textwidth}
    \includegraphics[width=\textwidth]{Figs/alphat/alphat_correlation_Zinv.pdf}
    \caption{\zinv}
    \label{fig:alphat_corr_zinv}
  \end{subfigure}\\
  \begin{subfigure}[b]{.46\textwidth}
    \includegraphics[width=\textwidth]{Figs/alphat/alphat_correlation_T2cc_250_240.pdf}
    \caption{\texttt{T2cc} (250, 240)}
    \label{fig:alphat_corr_t2cc}
  \end{subfigure}
  \caption{Plots showing the correlation between \mht and \deltaHT for QCD,
  \zinv and an example SUSY signature \texttt{T2cc} ($m_{\sTop} = 250$ \gev,
  $m_{\chiz} = 240$ \gev). Contours of constant \alphat are shown in black.
  Each axis is normalised according to the event \HT \emph{show all three?}.}
  \label{fig:alphat_corr}
\end{figure}


%********************************** % First Section  *************************************
\section{Hadronic Signal Region}
\label{sec:selection_hadronic}
% Introduce that QCD is a dominant background due to jet mis-measurement.

This section outlines the relevant background sources of this analysis, and
describes the basic hadronic pre-selection made for the signal 
region, as well as the relevant trigger requirements.


\subsection{Standard Model Backgrounds}
\subsubsection{Genuine \met}
The dominant EWK source of genuine missing energy comes from Z-boson 
production where the Z decays
to neutrinos, \zinv, with associated jet production. This source of background
is considered irreducible.

Events containing leptonic decays of W bosons, \wlnu, originating either
from direct W production, or via the decay of a top quark from \ttbar 
production, are sources of genuine 
missing energy, due to the presence of a weakly interacting neutrino which
evades detection. Such events are vetoed in the signal
region due to the presence of a 
lepton, however if the lepton is missed for whatever reason, leptonic W decays 
can pass the signal selection, forming a significant SM background.

% While lepton and photon vetoes employed in the signal region suppress 
% significant amounts of background from events with neutrinos, such events can 
% still persist if, for example, the lepton is not identified.
% Such processes are predominantly from \ttbar or W-boson production, where the 
% W decays via \wlnu. If the lepton is `lost' and evades our 
% lepton vetoes, significant missing energy can be produced not only from missing 
% the lepton, but from the presence of the neutrino. When such processes are accompanied 
% by associated jet production they are then able to pass our signal region selection.

Leptons can be `lost' for a variety different reasons, but ultimately for failing
the lepton ID criteria. There are numerous potential causes, the 
most prevalent being soft-leptons below ID threshold or non-isolated leptons 
which pass the ID quality cuts but fail the isolation requirement.


% \emph{move to selection section}
% Events containing a Single Isolated Track (SIT) are vetoed from the signal 
% region. This tracker based veto is particularly useful for vetoing additional events 
% that contain leptons which have failed our lepton ID requirements entirely, and 
% are therefore not considered by the leptonic vetoes. Additionally, the veto also
% removes background contributions from single-pronged hadronical decays of $\tau$
% leptons.

While originally designed to target hadronically decaying tau leptons, 
this requirement reduces the remaining lost-lepton backgrounds also.

Following the hadronic selection requirements,
any remaining contributions from SM EWK backgrounds are estimated using a 
fully data-driven transfer factor technique, described in detail in
Section~\ref{sec:background_overview}. A breakdown of the EWK background 
composition is shown in Figure~\ref{fig:background_decomp}, split into the main 
categories of \zinv, \wj, \ttbar and remaining residual backgrounds, such as
single top quark, diboson and Drell-Yann processes.

\begin{figure}[hb!]
\centering
\hspace{0cm}\includegraphics[width=0.9\textwidth, trim=0 00 0 0, clip=true]
{Figs/ra1_had_bg_comp_v3.png}
\label{fig:background_decomp}
\caption{The breakdown of the total electroweak background into component
processes as a function of \nj and \nb, for \HT>200 \gev.}
\end{figure}

\subsubsection{Fake \met}

As mentioned previously, the dominant source of background for analyses 
searching for a multijet final state is from QCD. A fully-measured QCD event 
would consist of multiple jets balancing each other in all planes, however in 
order to enter the signal region, an event must contain missing energy, 
\mht (equivalent to \met in all-hadronic events).

The most common way for a balanced multijet (MJ) event to gain \mht is when one 
or more of the jets are mis-measured, such that their vectorial sum then leads to non-zero
\mht. This can occur due to detector issues, or due to stochastic fluctuations
within
the inherant jet-resolution of the 
detector. The former is protected against by the \met filters summarised in 
table~\ref{tab:met_filters} and using a filter to remove events
affected by non-functioning or damaged regions of the ECAL system, where 
events are vetoed if they contain significant energy deposits within a given 
distance from a known problematic region. The latter is dealt with using a
cut on the \alphat variable where 
events with fake missing energy signatures give values $<0.5$.

QCD MJ events can also appear to contain non-zero missing energy 
due to the threshold requirements of jets. If an event contains 
one or more jets below the analysis threshold, then the 
event is measured as imbalanced and containing \mht. Events such as these are largely 
removed with the \alphat requirement, however in addition a requirement is made on the
ratio \mhtmet.

% Finally, it is also possible for rare instrumentation effects to lead to jet 
% energy
% mismeasurements. To protect against this, a suite of MET filters are defined by 
% the JetMET \emph{POG}, and are applied to all selections.


\subsection{Signal Triggers}

Events are collected at the HLT using a dedicated suite of
signal triggers. For an event to pass the trigger
requirements, it must exceed both a \HT and an \alphat threshold. Trigger rate 
can be maintained by varying the
threshold requirement on each of these independent requirements, as shown in
Table~\ref{tab:sig_trigs}. Each \HT bin in the analysis is seeded by a
particular signal
trigger, with a 25 \gev offset in online and offline \HT, with the exception of
the 200 \gev bin.

Exclusively for this analysis, the additional `Parked' trigger 
\\\verb!HT200_AlphaT0p57! is included, seeding the new \HT> 200 \gev bin. Such a low
threshold allow sensitivity to be maintained for softer physics signatures, such
as those expected from compressed spectra SUSY decays.

\begin{table}[!ht]
  \caption{Signal triggers, the L1 seed triggers and their efficiencies measured
  for per \HT and \nj category.}
  \label{tab:sig_trigs}
  \centering
  \scriptsize
  \begin{tabular}{ cccccc }
    \hline
    \hline
    Offline \HT       & Offline \alphat & L1 seed (\verb!L1_?!)         & Trigger (\verb!HLT_?!)  & \multicolumn{2}{c}{Efficiency (\%)}          \\ [0.5ex]
    region (\gev)         & threshold       & (highest thresholds)          &                         & $2 \leq \nj \leq 3$ & $\nj \geq 4$       \\ [0.5ex]
    \hline
    $200 < \HT < 275$ & 0.65            & \verb!DoubleJetC64!           & \verb!HT200_AlphaT0p57! & $81.8^{+0.4}_{-0.4}$  & $78.9^{+0.3}_{-0.4}$ \\
    $275 < \HT < 325$ & 0.60            & \verb!DoubleJetC64!           & \verb!HT200_AlphaT0p57! & $95.2^{+0.3}_{-0.4}$  & $90.0^{+1.2}_{-1.3}$ \\
    $325 < \HT < 375$ & 0.55            & \verb!DoubleJetC64 OR HTT175! & \verb!HT300_AlphaT0p53! & $97.9^{+0.3}_{-0.3}$  & $95.6^{+0.9}_{-1.0}$ \\
    $375 < \HT < 475$ & 0.55            & \verb!DoubleJetC64 OR HTT175! & \verb!HT350_AlphaT0p52! & $99.2^{+0.2}_{-0.2}$  & $98.7^{+0.5}_{-0.7}$ \\
    $\HT > 475$       & 0.55            & \verb!DoubleJetC64 OR HTT175! & \verb!HT400_AlphaT0p51! & $99.8^{+0.1}_{-0.3}$  & $99.6^{+0.3}_{-0.7}$ \\
    \hline
    \hline
  \end{tabular}
\end{table}

Trigger efficiencies are measured using an unbiased single muon reference
trigger,
\\\verb!HLT_IsoMu24_eta2p1!, using a muon tag and probe method where a
single muon is selected and then subsequently ignored from the analysis when 
calculating event level variables such as \HT, \mht and \alphat. Efficiencies 
are measured for each \HT bin and for each \nj category, as summarised in 
table~\ref{tab:sig_trigs}. Example trigger `turn-on' curves are shown for the 3 
lowest \HT bins in figures~\ref{fig:eff_alphat_le3j} and \ref{fig:eff_alphat_ge4j}.
The curves are shown both differentially and cumulatively, to show the
efficiency for events of a given \alphat value and above an \alphat threshold
respectively.
Across the higher \HT 
bins the triggers are fully efficient. Inefficiencies in the low \HT bins are
understood as being due to the relatively high threshold L1 seed trigger used
for this region, in order to maintain
low rates in the high PU environment encountered throughout \runone. Lower 
efficiencies are also observed in the \njhigh category attributed to the presence of 
softer jets, as an increased number of jets must equate to the same total \HT 
requirement of the bin.

\afterpage{%
\begin{figure}[!ht]
  \centering
    \begin{subfigure}[b]{0.48\textwidth}
      \includegraphics[width=\textwidth,page=11, trim=0 0 0 20, clip=true]{figures/trigger/HT200_275_73_73_36_AlphaT_le3j_RunAtFNAL}
      \caption{Differential, $200 < \HT < 275 $ \gev}
    \end{subfigure}
    \begin{subfigure}[b]{0.48\textwidth}
      \includegraphics[width=\textwidth,page=18, trim=0 0 0 20, clip=true]{figures/trigger/HT200_275_73_73_36_AlphaT_le3j_RunAtFNAL}
      \caption{Cumulative, $200 < \HT < 275 $ \gev}
    \end{subfigure} \\
    \vspace{0.5cm}\begin{subfigure}[b]{0.48\textwidth}
      \includegraphics[width=\textwidth,page=11, trim=0 0 0 20, clip=true]{figures/trigger/HT275_325_73_73_36_AlphaT_le3j_RunAtFNAL}
      \caption{Differential, $275 < \HT < 325 $ \gev}
    \end{subfigure}
    \begin{subfigure}[b]{0.48\textwidth}
      \includegraphics[width=\textwidth,page=18, trim=0 0 0 20, clip=true]{figures/trigger/HT275_325_73_73_36_AlphaT_le3j_RunAtFNAL}
      \caption{Cumulative, $275 < \HT < 325 $ \gev}
    \end{subfigure} \\
    \vspace{0.5cm}\begin{subfigure}[b]{0.48\textwidth}
      \includegraphics[width=\textwidth,page=11, trim=0 0 0 20, clip=true]{figures/trigger/HT325_375_86_86_43_AlphaT_le3j_RunAtFNAL}
      \caption{Differential, $325 < \HT < 375 $ \gev}
    \end{subfigure}
    \begin{subfigure}[b]{0.48\textwidth}
      \includegraphics[width=\textwidth,page=18, trim=0 0 0 20, clip=true]{figures/trigger/HT325_375_86_86_43_AlphaT_le3j_RunAtFNAL}
      \caption{Cumulative, $325 < \HT < 375 $ \gev}
    \end{subfigure} \\
  
    \caption{\label{fig:eff_alphat_le3j}
      Differential (left) and Cumulative (right) efficiency turn-on curves for 
      the signal triggers, for the three lowest \HT bins and \njlow.}
\end{figure}
\clearpage
}

\afterpage{%
\begin{figure}[!ht]
  \centering
    \begin{subfigure}[b]{0.48\textwidth}
      \includegraphics[width=\textwidth,page=11]{figures/trigger/HT200_275_73_73_36_AlphaT_ge4j_RunAtFNAL}
      \caption{Differential, $200 < \HT < 275 $ \gev}
    \end{subfigure}
    \begin{subfigure}[b]{0.48\textwidth}
      \includegraphics[width=\textwidth,page=18]{figures/trigger/HT200_275_73_73_36_AlphaT_ge4j_RunAtFNAL}
      \caption{Cumulative, $200 < \HT < 275 $ \gev}
    \end{subfigure} \\
    \begin{subfigure}[b]{0.48\textwidth}
      \includegraphics[width=\textwidth,page=11]{figures/trigger/HT275_325_73_73_36_AlphaT_ge4j_RunAtFNAL}
      \caption{Differential, $275 < \HT < 325 $ \gev}
    \end{subfigure}
    \begin{subfigure}[b]{0.48\textwidth}
      \includegraphics[width=\textwidth,page=18]{figures/trigger/HT275_325_73_73_36_AlphaT_ge4j_RunAtFNAL}
      \caption{Cumulative, $275 < \HT < 325 $ \gev}
    \end{subfigure} \\
    \begin{subfigure}[b]{0.48\textwidth}
      \includegraphics[width=\textwidth,page=11]{figures/trigger/HT325_375_86_86_43_AlphaT_ge4j_RunAtFNAL}
      \caption{Differential, $325 < \HT < 375 $ \gev}
    \end{subfigure}
    \begin{subfigure}[b]{0.48\textwidth}
      \includegraphics[width=\textwidth,page=18]{figures/trigger/HT325_375_86_86_43_AlphaT_ge4j_RunAtFNAL}
      \caption{Cumulative, $325 < \HT < 375 $ \gev}
    \end{subfigure} \\
  
    \caption{\label{fig:eff_alphat_ge4j}
      Differential (left) and Cumulative (right) efficiency turn-on curves for 
      the signal triggers, for the three lowest \HT bins and \njhigh.}
\end{figure}
\clearpage
}

All triggers were present throughout \runone, however the 
\\\verb!HLT_HT200_AlphaT0p57! trigger was used as part of the `Parked' stream of 
data which was reconstructed at a later date, following the active data taking 
period. During data taking triggers may have `prescale' factors applied to them 
such that only every $n$ triggered events are actually recorded, however all of
the signal triggers remained unprescaled for the entirety of the 8 \tev
data taking.


\subsection{Selection Criteria}
\label{sec:selec_crit}

Event selection requirements for the hadronic signal region are chosen with an
aim to
maintain sensitivity to hadronically decaying sparticle production, while 
rejecting as many QCD-type processes as possible. To do so, requirements are 
made on:

\begin{description}
\item[Jets]
Events are required to contain at least two jets with at least
$\HT > 200$ \gev, to ensure the presence of significant hadronic activity. As 
mentioned previously, events are categorised by \HT, with jet \Pt
requirements on the two leading and the remaining additional jets seperately.
The jet \Pt thresholds vary as a function of the \HT bin of the event, as shown in
Table~\ref{tab:jet_pt_thresholds}, in order to maintain a similar kinematic
phase space throughout the \HT range.

\begin{table}[ht!]
  \caption{Jet \Et thresholds per \HT bin.\label{tab:jet_pt_thresholds}}
  \centering
  \footnotesize
  \begin{tabular}{ lcccc }
    \hline
    \hline
    \HT bin        & 200--275 & 275--325 & 325--375 & $>$375 \\
    \hline
    Lead jet       & 73.3     & 73.3     & 86.7     & 100.0  \\
    Second jet     & 73.3     & 73.3     & 86.7     & 100.0  \\
    All other jets & 36.7     & 36.7     & 43.3     & 50.0   \\
    \hline
    \hline
  \end{tabular}
\end{table}

\item[Leptons]
Any events containing leptons are vetoed to ensure hadronic events
are considered, thereby suppressing events with genuine \met from leptonic decays to
neutrinos such as \wlnu.

\item[Photons]
Events containing photons are vetoed, for similar reasons as the leptonic 
vetoes, in order to maintain a purely hadronic environment.

\item[Single Isolated Tracks]
Events containing a Single Isolated Track (SIT) are vetoed from the signal 
region. This tracker based veto is particularly useful for vetoing additional events 
that contain leptons which have failed our lepton ID requirements entirely, and 
are therefore not considered by the leptonic vetoes. Additionally, the veto also
removes background contributions from single-pronged hadronical decays of $\tau$
leptons.

\item[Event]
The topology of the event is required to pass a threshold of $\alphat > 0.55$, 
a requirement which itself varies as a function of the \HT category being
considered, always
chosen such that QCD MJ events contribute a negligible amount to the overall 
background UPDATE. While no absolute \met requirement is made, the cut on 
\alphat imposes an implied threshold, which maintains the analysis'
sensitivity to very low regions of \met. \emph{give an example of the actual met
threshold for a given HT bin.}

\end{description}

\subsection{Residual QCD cleaning}

While the \alphat requirement removes many order of magnitude of QCD events,
there still exists scenarios in which events may pass the signal region
selection. Accordingly, further requirements are made to ensure the search
region is free of any residual QCD contamination.

\begin{figure}[ht!]
\centering
\includegraphics[width=0.6\textwidth]
{Figs/datamc/had/v1/Stacked_MHTovMET_all_200_upwards.png}
\caption{The \mhtmet distribution of MC events following the hadronic selection
criteria, minus the \mhtmet requirement.The MC yields are stacked,
with the QCD contribution shown in cyan. The plot is for a fully inclusive
selection of $\nb \geq 0$, $\nj \geq 2$ and \HT > 200 \gev.}
\label{fig:full_mhtmet_distro}
\end{figure}

Events can acquire non-negligible amounts of \mht without the presence of
real \met if multiple jets are below the analysis threshold, where their
configuration can conspire to give a topology passing the \alphat thresholds.
Such events will contain a disparity between the energy deposit based \met and
jet object based \mht variables. This effect is seen in
figure~\ref{fig:full_mhtmet_distro} where the presence of significant QCD
contamination at high values of \mhtmet is observed, despite an \alphat
requirement. To protect against this scenario, events are required to have a low
ratio of the two variables, specifically \mhtmet < 1.25.

Fake \mht may also be produced if jets overlap with areas of the calorimeter 
system which are damaged or known to be faulty, where jets can be mis-measured or 
lost as a result. To protect against this, for a given jet $j$, the angular separation
between the event \mht, calculated excluding jet $j$, and the jet itself is used, defined
as:
% 
\begin{equation}
\dphistar_j = \Delta \phi\big(\overrightarrow{\Pt}_j,-\sum_{i\neq j}{\overrightarrow{\Pt}_i}\big)
\label{eq:biasdphi}
\end{equation}
% 
An advantage of this variable with respect to the often used
$\Delta R(jet, MHT)$ is it's detection of spurious missing energy vectors caused
by both under-measurements and over-measurements of a jet's transverse momentum.
A small value of $\dphistar_j$ indicates that the momentum vector of $j$
is aligned with the \mht vector, implying the jet is mis-measured. Events are 
vetoed if a jet with \dphistar< 0.5 is within $\Delta R < 0.3$ of a known
`dead' region of the ECAL.

Multiple event filters are applied to remove any jet mismeasurements arising due
to instrumental effects, however previously un-discovered and
therefore rare detector effects may hypothetically be present. To check
for this, the
jet giving the minimum \dphistar value in an event, \mindphistar, is found,
and a single entry of the $\eta$ and $\phi$ direction of the jet's axis is entered
into a map of the detector, figure~\ref{fig:hotspots}. Any areas of instrumental issue would
be visible as clusters of high event counts. Figures~\ref{fig:hotspots_2d_nodeadECAL} and
\ref{fig:hotspots_1d_nodeadECAL} shows the detector map and the 1d distribution
of counts are shown before the dead ECAL filter is applied. Areas of
potential instrumental defects are clearly visible, notably as outliers in
figure~\ref{fig:hotspots_1d_nodeadECAL}. Following the application of the dead
ECAL filter the hotspot areas and the corresponding outliers removed, as seen in
figures~\ref{fig:hotspots_2d_withdeadECAL} and \ref{fig:hotspots_1d_withdeadECAL}.

\begin{figure}[h!]
  \begin{center}
    \begin{subfigure}[b]{0.46\textwidth}
      \includegraphics[width=\textwidth]{Figs/dphi/HT_dependent_AlphaT_thresholds/th2d_denom_summed_ge2j_ge0b_200.pdf}
      \caption{No ``dead ECAL filter''.}
      \label{fig:hotspots_2d_nodeadECAL}
    \end{subfigure}
    \begin{subfigure}[b]{0.46\textwidth}
      \includegraphics[width=\textwidth]{Figs/dphi/HT_dependent_AlphaT_thresholds/th1d_denom_summed_ge2j_ge0b_200.pdf}
      \caption{No ``dead ECAL filter''.}
      \label{fig:hotspots_1d_nodeadECAL}
    \end{subfigure} \\ 
    \begin{subfigure}[b]{0.46\textwidth}
      \includegraphics[width=\textwidth]{Figs/dphi/Nominal_AlphaT_thresholds/th2d_numer_summed_ge2j_ge0b_200.pdf}
      \caption{With ``dead ECAL filter''.}
      \label{fig:hotspots_2d_withdeadECAL}
    \end{subfigure}
    \begin{subfigure}[b]{0.46\textwidth}
      \includegraphics[width=\textwidth]{Figs/dphi/Nominal_AlphaT_thresholds/th1d_numer_summed_ge2j_ge0b_200.pdf}
      \caption{With ``dead ECAL filter''.}
      \label{fig:hotspots_1d_withdeadECAL}
    \end{subfigure} \\ 
    \caption{Distribution of jets in ($\eta$,$\phi$)-space that are
      responsible for an event satisfying the requirement $\dphistar <
      0.3$, with (a,b) and without (c,d) the ``dead ECAL filter''
      requirement applied as part of the signal region selection.}
    \label{fig:hotspots}
  \end{center}
\end{figure}

\emph{also jets with real MET in their decay}

% The \dphistar variable can be used to identify QCD events with \met
% originating from mismeasured jets. The jet most responsible for the \met
% vector can be isolated by finding that which has the minimum \dphistar value.
Jets can also appear to be mis-measured if they contain leptonic decays of heavy
flavour mesons (see appendix~\ref{ch:app_dphistar}). 

\emph{ratio of R(MHT/MET) can be used as a function of alphaT to check for
these events. MHT roughly equal to MET, ratio roughly 1. would expect a steeply
falling distribution as function of alphaT, but then a constant pedestal from
these events, which pass alphaT}.

\emph{can study these events using control region and un-prescaled HT triggers}

\emph{what if we make the >0.3 requirement? they seem isolated below 0.3 -
robin's plots?}

\emph{R(MHTMET) plots}

\emph{dphistar pass/fail plots}

Events are required to have \dphistar > 0.3, in
order to clean any remaining events with jets containing sources of genuine \met jet (see
appendix~\ref{ch:app_dphistar}).

% For bins of $\HT>375$ \gev the leading two jets in the event are required to 
% have $\Pt>100$ \gev, with all additional jets having half the requirement of
% $\Pt>50$ \gev. In order to maintain a similar kinematic phase space throughout
% the many \HT bins, these jet \Pt requirements are scaled for the lower \HT bins 
% as shown in Table~\ref{tab:jet_pt_thresholds}.

\subsection{Example distributions and cutflow}

Example distributions from MC simulations of \alphat, \HT, \mht and jet \Pt  are
shown in figure~\ref{fig:datamc_had_inc}.

\begin{figure}[!ht]
  \centering
    \begin{subfigure}[b]{0.48\textwidth}
      \includegraphics[width=\textwidth]{Figs/datamc/had/v1/AlphaT_all_200_upwards}
      \caption{\alphat}
    \end{subfigure}
    \begin{subfigure}[b]{0.48\textwidth}
      \includegraphics[width=\textwidth]{Figs/datamc/had/v1/HT_all_200_upwards}
      \caption{\HT}
    \end{subfigure} \\
    \begin{subfigure}[b]{0.48\textwidth}
      \includegraphics[width=\textwidth]{Figs/datamc/had/v1/MHT_all_200_upwards}
      \caption{\mht}
    \end{subfigure}
    \begin{subfigure}[b]{0.48\textwidth}
      \includegraphics[width=\textwidth]{Figs/datamc/had/v1/CommonJetPt_all_200_upwards}
      \caption{Jet \Pt}
    \end{subfigure} \\
    \caption{\label{fig:datamc_had_inc}
    MC distributions for the full hadronic signal selection. The 
    sum of the individual sample contributions is shown in green. Plots
    are for $\HT>200$ \gev, $\nj\geq2, \nb\geq0$.
    }
\end{figure}
  
The cutflow yields are shown in table~\ref{tab:had_data_cutflow} for data in the
\HT > 375 \gev region, with an inclusive selection on \nj and \nb.

\subsection{Predicting high-\nb event yields}
\label{sec:formula_method}

Determining yields for high-\nb analysis categries directly from simulation
becomes
inherently reliant on not only MC modelling, but also on the physics
distribution of jet-flavours. For EWK samples, given the underlying abundance of
low b-jet multiplicity events, high-\nb categories become dominated by
mis-tagging of light-flavour jets, leading to large statistical uncertainties.
In order to reduce this reliance on direct MC yields, a method has been
developed REF using flavour tagging efficiencies and the underlying quark
flavour distribution, both measured directly from simulation in order to
statistically determine more precise yields, particularly for the high-\nb
categories.

This method and it's validation are described in detail in REF. The approach can
be summarised by:
% 
\begin{equation}
\begin{split}
N(n) = \sum_{n_b^{gen} + n_c^{gen} + n_{\text{light}}^{gen} = n_{jet}^{cat}}\:
\sum_{n_b^{tag} + n_c^{tag} + n_{\text{light}}^{tag} = n}
N(n_b^{gen}, n_c^{gen}, n_q^{gen}) \times \\
P(n_b^{tag}, n_b^{gen}, \epsilon_{b}) \times
P(n_c^{tag}, n_c^{gen}, \epsilon_{c}) \times
P(n_{\text{light}}^{tag}, n_{\text{light}}^{gen}, \epsilon_{\text{light}})
\end{split}
\end{equation}
% 
where $N(n)$ is represents the predicted yield of $n$ b-tagged events for a
given
analysis category and \HT bin. The jet-flavour tagging probability
terms $P(n_b^{tag}, n_b^{gen}, \epsilon_{b})$,
$P(n_c^{tag}, n_c^{gen}, \epsilon_{c})$ and 
$P(n_{\text{light}}^{tag}, n_{\text{light}}^{gen}, \epsilon_{\text{light}})$ are measured for each \HT bin and
analysis category for each simulated sample, including the flavour tagging
effiencies $\epsilon_b$, $\epsilon_c$ and $\epsilon_{\text{light}}$ as also measured from
each sample. Finally, $N(n_b^{gen}, n_c^{gen}, n_q^{gen})$ represents the
distribution of actual generator-level jet flavours in the events.

In the large-statistic sample limit this technique would not be necessary,
as enough high-\nb events would be present to sufficiently reduce statistical
uncertainties. However, in the absence of this un-feasible scenario,
this technique
makes uses of all events in a sample, thereby reducing the uncertainties
and ultimately delivering more precise yields from simulation.




\begin{table}[ht!]
  \caption{The cutflow of the hadronic selection in data. The subscript `fail'
  indicates an object which did not meet all ID requirements, and so is not
  considered as a `common object'. The event selection follows an inclusive
  selection in data of $\HT > 375$ \gev, $\nj \geq 2$ and $\nb \geq 0$. }
  \label{tab:had_data_cutflow}
  \centering
  \footnotesize
  \begin{tabular}{ lcc }
    \hline
    \hline
    Cut Name    & Eff (\%) & N \\
    \hline
  Event Count  & 100.00  & 161866108.00 \\
  Good Event JSON Filter  & 94.97  & 153716716.00 \\
  $\nj \geq 2$  & 98.68  & 151689664.00 \\
  MET Filters & 98.81  & 149885013.00 \\
  Vertex Noise Filter & 99.91  & 149746017.00 \\
  DeadECAL Filter & 35.44  & 53066597.00 \\
  Select seed Trigger & 3.76  & 1995549.00 \\
  $n_{e} = 0$ & 98.48  & 1965269.00 \\
  $n_{\gamma} = 0$  & 97.70  & 1920124.00 \\
  $n_{\mu} = 0$ & 98.36  & 1888711.00 \\
  $\text{EMF}_{max}$ for all jets > 0.1 & 99.99  & 1888576.00 \\
  Leading jet \Pt > 100 \gev  & 97.83  & 1847501.00 \\
  Leading jet $\eta$ < 2.5  & 97.31  & 1797777.00 \\
  Sub-Leading jet \Pt > 100 \gev  & 81.75  & 1469654.00 \\
  $n_{jet, fail} = 0$ & 81.73  & 1201096.00 \\
  $\Delta R(\mu^i_{fail}, jet^j) < 0.5$ & 98.77  & 1186370.00 \\
  $(\sum_{}^{n_{vertices}}{\Pt}$) / \HT & 100.00  & 1186368.00 \\
  recHitCut & 98.44  & 1167861.00 \\
  $n_{SIT} = 0$ & 85.77  & 1001646.00 \\
  \mindphistar > 0.3  & 19.50  & 195361.00 \\
  \HT > 375 \gev  & 73.33  & 143255.00 \\
  \mhtmet < 1.25  & 9.26  & 13263.00 \\
  \alphat > 0.55  & 38.25  & 5073.00 \\
    \hline
    \hline
  \end{tabular}
\end{table}






