\chapter{Background Estimation}
\label{ch:background}

% **************************** Define Graphics Path **************************
\ifpdf
    \graphicspath{{Chapter6/Figs/Raster/}{Chapter6/Figs/PDF/}{Chapter6/Figs/}}
\else
    \graphicspath{{Chapter6/Figs/Vector/}{Chapter6/Figs/}}
\fi

Predictions of EWK SM backgrounds are made using independent control regions
designed to mimic a specific background process observed in the signal region.
These regions are orthogonal to the signal region due to the
selection of leptons or photons - particles which are subsequently ignored such 
that the selection kinematics of each event is kept similar to the corresponding
process. For each sample, Transfer Factors (TF) are 
constructed from yields in MC, to extract a prediction in the signal 
region for a given process. The total EWK prediction is comprised of a prediction
of lost-lepton backgrounds originating predominantly from \ttbar + jets and \wj processes,
and a prediction of \zj, with the latter being treated differently according to \nb.
This process is described in detail in
Section\ref{sec:background_overview}. The validity of this procedure is
extensively tested using a suite of Closure Tests, probing multiple
characteristics of the prediction technique while allowing for \HT-dependent
systematic errors to be derived on the prediction.

Predictions made using this technique alone are considered as ``primitive'' 
predictions and are used only in analysis development and the derivation of 
background systematics. In order to determine final background counts for
interpretation and
limit-setting, a fit is made across all signal and control regions, using the 
likelihood model as described later in chapter~\ref{ch:results}. The derived 
transfer factors and individual yields enter as terms in the likelihood, where 
all related systematics, potential correlations and signal contamination are
accounted for.

Residual contamination from QCD processes are removed through additional
cleaning cuts. Studies performed in a QCD enriched sideband region in \alphat
lead to the definition of these cuts, as described in
Section\ref{sec:qcd_cleaning}, ensuring no significant QCD background remains.

% Thresholds of the \alphat variable are determined such that the QCD MJ 
% contribution to the SM background is negligible. These are found using a
% data-driven method, employing sidebands in both the \mhtmet and \alphat variables, 
% described in Section\ref{sec:background_qcd}. Following the 
% application of these thresholds, any QCD contributions are considered negligible
% and are therfore not considered in the likelihood model.

%********************************** % First Section  *************************************
\section{Overview of Electroweak background prediction method}  %Section - 1.1 
\label{sec:background_overview}

% Contributions from Standard Model background processes are estimated using a data-driven 
% prediction technique, employing dedicated control samples.
A transfer factor is constructed from MC samples as the ratio 
of the yield in the signal region, \\$N_{MC}^{signal}\big(\HT, \nj, \nb\big)$,
and the yield of a given control region, \\$N_{MC}^{control}\big(\HT, \nj,
\nb\big)$, as a function of the analysis binning, \HT, \nj and \nb, defined as:
% 
\begin{equation}
TF = \frac{N_{MC}^{signal}\big(\HT, \nj, \nb\big)}{N_{MC}^{control}\big(\HT, \nj, \nb\big)} .
\label{eq:transfer_factor}
\end{equation}
% 
For a given \HT, \nj and \nb bin, the TF is used to extrapolate a yield in data from
the control region, $N_{data}^{control}\big(\HT, \nj, \nb\big)$,
to give a prediction in the signal region $N_{pred}^{signal}\big(\HT, \nj,
\nb\big)$, using:
%
\begin{equation}
N_{pred}^{signal}\big(\HT, \nj, \nb\big) = N_{data}^{control}\big(\HT, \nj, \nb\big)
\times TF .
\label{eq:transfer_equation}
\end{equation}
% 
The control samples are statistically independent and each used for predicting 
specific background processes.

MC yields, following the application of all MC correction factors described in
chatper~\ref{ch:samples}, are defined by process from the following samples:
W + jets (\numw),
\ttbar + jets (\numtt), DY + jets (\numdy), $\gamma$ + jets(\numgam),
single top + jets (\numtop), WW + jets, WZ + jets and ZZ + jets (\numdibo), and
\zinv + jets (\numzinv).

% Predictions made using this technique alone are considered as ``na\"{i}ve'' 
% predictions and are used only in analysis development and the derivation of 
% background systematics. In order to determine final yields for interpretation and 
% limit-setting, a fit is made across all signal and control regions, using the 
% full likelihood model, described later in chapter~\ref{ch:7}. The derived 
% transfer factors and individual yields enter as terms in the likelihood, where 
% all related systematics, potential correlations and signal contamination are
% accounted for in the fit. 

The denominator of each transfer factor is constructed using the sum of
\textit{all} MC sample yields, for a given control region and
analysis category:
% 
\begin{equation}
N_{MC}^{control}\big(\HT, \nj, \nb\big) = \numw + \numtt + \numdy + \numgam + 
\numtop + \numdibo + \numzinv .
\label{eq:trans_fact_denom}
\end{equation}
% 
The numerator is constructed according to the b tag multiplicity being 
considered. Table~\ref{tab:control_prediction_summary} summarises the control
samples used to determine a background prediction as a function of each analysis
category. For $\nb \leq 1$, the \mj control region is used to predict 
lost-lepton background, e.g. \ttbar + jets and W + jets. All MC samples
are therefore used with the exception of \zinv + jets:
% 
\begin{equation}
N_{MC}^{signal}\big(\HT, \nj, \nb \leq 1\big) = \numw + \numtt + \numdy + \numgam + 
\numtop + \numdibo .
\label{eq:trans_fact_num_le1b_noz}
\end{equation}
% 
In this \nb region the \zinv + jets component of the background is predicted
with the \mmj and \gj control samples, using only the \zinv MC yields:
% 
\begin{equation}
N_{MC}^{signal}\big(\HT, \nj, \nb \leq 1 \big) = \numzinv .
\label{eq:trans_fact_num_le1b_z}
\end{equation}
%
For $\nb \geq 2$, the \mj sample is used to produce a prediction for all 
SM processes, including \zinv, and therefore the numerator of the TF is defined as:
% 
\begin{equation}
N_{MC}^{signal}\big(\HT, \nj, \nb \geq 2 \big) = \numw + \numtt + \numdy + \numgam + 
\numtop + \numdibo + \numzinv ,
\label{eq:trans_fact_num_geq2b}
\end{equation}
% 
or equivalently:
% 
\begin{equation}
N_{MC}^{signal}\big(\HT, \nj, \nb \geq 2 \big) = N_{MC}^{signal}\big(\HT, \nj, \nb \leq 1\big) + \numzinv .
\end{equation}

The \mmj and \gj control samples are not used beyond the $\nb \leq 1$ categories
due to the statistical limitations of such samples at high b tag multiplicities.

It should be noted that although two separate control samples are 
used to estimate the \zinv background contribution, the result of each is
considered by the global likelihood fit to produce the final background
prediction (described in Section\ref{sec:results_likelihood}).

\begin{table}[ht!]
  \caption{The control samples used to produce SM background predictions for each 
  analysis category (\nb, \nj).}
  \label{tab:control_prediction_summary}
  \centering
  \small
  \begin{tabular}{ lll }
    \hline
    \hline
    \nj     & \nb     & Control samples \\ [1.0ex]
    \hline
    2--3    & 0       & \mj, \mmj, \gj  \\
    2--3    & 1       & \mj, \mmj, \gj  \\
    2--3    & 2       & \mj             \\
    $\geq$4 & 0       & \mj, \mmj, \gj  \\
    $\geq$4 & 1       & \mj, \mmj, \gj  \\
    $\geq$4 & 2       & \mj             \\
    $\geq$4 & 3       & \mj             \\
    $\geq$4 & $\geq4$ & \mj             \\
    \hline
    \hline
  \end{tabular}
\end{table}

By employing a technique that uses a ratio of MC yields, direct dependence on
MC modelling is 
greatly reduced. Sources of error inherent to MC samples, such as mismodelling effects, 
will approximately cancel in the ratio. These errors can potentially include kinematic
mismodelling, which would affect analysis acceptance, and mismodelling of 
instrumental effects, which could affect object 
reconstruction efficiencies. However, any remaining systematics such as these
and others are probed using dedicated Closure Tests (CT), described in detail
in Section\ref{sec:closure_tests}.


%********************************** % Second Section  *************************************
\section{Control samples}  %Section - 1.2
\label{sec:background_control}

Control sample definitions are designed such that they are as kinematically
similar as possible
to the signal region, with the exception of a selected `tag' muon or 
photon. The `tagged' particle is then subsequently ignored for the calculation of all analysis 
variables, such as \HT, \met, \alphat etc.
Other differences include mass-window
and minor kinematic cuts, aimed at enriching the control samples in certain processes.
% The samples themselves are statistically independent, and orthogonal to the 
% signal region
Due to the selection of the tagged lepton or photon, the control samples are
orthogonal to the signal region and therefore minimise any
possible signal contamination. However, a full treatment of the
signal-contamination and sample cross-correlation is taken into account in the
background fit and final limit-setting. 

The following sections will describe the control regions in more detail,
including their targeted background estimations, selection cuts specific to each
and their trigger requirements.

\subsection{\mj}
\label{sec:mujets_control_sample}

The \mj control sample is constructed by selecting a single muon with associated 
jets. This region is used to predict backgrounds from processes such as \wj and
\ttj. This covers not only the leptonic decays of such productions, where the 
lepton is not identified for whatever reason, but also the hadronic decays of tau 
leptons [from high-\Pt W bosons]. The event selection therefore is optimised to 
select W bosons decaying to a muon and a neutrino in the phase-space of the 
signal.

\subsubsection{Triggers}
\label{sec:mujets_control_trigger}
Events are collected using the\verb!HLT_IsoMu24_eta2p1! trigger, which was in
place throughout the 8~\tev
data-taking period. The efficiency of this trigger was measured by the muon
\emph{POG} in bins of the muon \Pt and $\eta$, as summarised 
in Table~\ref{tab:muon-trig-effs}. Statistical uncertainties are at the
per-mille level, and systematics are taken to be 1~\%.

% \begin{table}[!ht]
%   \caption{Muon trigger efficiencies (\%) for the \mj selection listed by \HT bin and
%   \nj category.}
%   \label{tab:muon-trig-effs}
%   \centering
%   \small
%   \begin{tabular}{ cccc }
%     \hline
%     \hline
%     \HT (GeV) & 2-3 & $\geq$4 \\ [0.5ex]
%     \hline
% %    200--275  & 89.1 & 89.8  \\
% %    275--325  & 89.3 & 89.8  \\
% %    325--375  & 89.5 & 90.0  \\
% %    375--475  & 89.7 & 90.3  \\
% %    475--575  & 89.8 & 90.5  \\
% %    575--675  & 90.0 & 90.6  \\
% %    675--775  & 90.1 & 90.7  \\
% %    775--875  & 90.2 & 90.8  \\
% %    875--975  & 90.4 & 90.6  \\
% %    975--1075 & 90.3 & 90.6  \\
% %    $>$1075   & 90.0 & 91.2  \\
%     150--200  & 87.2 & 88.1  \\
%     200--275  & 87.5 & 88.1  \\
%     275--325  & 87.8 & 88.2  \\
%     325--375  & 87.9 & 88.4  \\
%     375--475  & 88.1 & 88.6  \\
%     475--575  & 88.2 & 88.8  \\
%     575--675  & 88.4 & 88.9  \\
%     675--775  & 88.5 & 89.0  \\
%     775--875  & 88.6 & 89.1  \\
%     875--975  & 88.8 & 89.0  \\
%     975--1075 & 88.7 & 89.0  \\
%     $>$1075   & 88.4 & 89.6  \\
%     \hline
%     \hline
%   \end{tabular}
% \end{table}

\begin{table}[h!]
  \caption{Muon trigger efficiencies (\%) for the \mj selection listed by \HT bin and
  \nj category.}
  \label{tab:muon-trig-effs}
  \centering
  \footnotesize
  \begin{tabular}{ l|cccccccccccc }
    \hline
    \hline
    \multirow{2}{*}{\nj} & \multicolumn{10}{c}{HT bin low edge (GeV)} \\
    & 150 & 200 & 275 & 325 & 375 & 475 & 575 & 675 & 775 & 875 & 975 & 1075 \\
    \hline
    2--3 & 87.2 & 87.5 & 87.8 & 87.9 & 88.1 & 88.2 & 88.4 & 88.5 & 88.6 & 88.8 &
    88.7 & 88.4 \\
    $\geq$4 & 88.1 & 88.1 & 88.2 & 88.4 & 88.6 & 88.8 & 88.9 & 89.0 & 89.1 &
    89.0 & 89.0 & 89.6 \\
    \hline
    \hline
  \end{tabular}
\end{table}

\subsubsection{Selection Criteria}
\label{sec:mujets_control_selection}
A single tight isolated muon is selected, with \Pt > 30~\gev and $|\eta| <$2.1.
The transverse mass of the W, reconstructed by the 
$\mu$ and the \met (originating from the $\nu_{\mu}$), is required to be in a 
loose window around $m_W$, $30 < M(\mu, \met) < 125$~\gev. Events are vetoed if
$\Delta R(\mu, jet_{i}) < 0.5$,
for all jets $i$ in the event. To keep the selection as close to the signal 
region as possible, other cuts such as the single isolated track veto and 
\mhtmet cuts are also applied.

\begin{figure}[t]
  \centering
    \begin{subfigure}[b]{0.48\textwidth}
      \includegraphics[width=\textwidth]{Figs/datamc/mu/Stacked_MuPt_all_OneMuon_200_upwards}
      \caption{$\mu$ \Pt}
    \end{subfigure}
    \begin{subfigure}[b]{0.48\textwidth}
      \includegraphics[width=\textwidth]{Figs/datamc/mu/Stacked_MET_Corrected_all_OneMuon_200_upwards}
      \caption{\met (corrected for $\mu$)}
    \end{subfigure} \\
    \caption{\label{fig:datamc_mu_inc}
    Comparison of data with MC for the \mj control selection. Plots 
    are for $\HT>200$~\gev, $\nj\geq2, \nb\geq0$.
    }
\end{figure}

Specifically for the \mj (and \mmj) control samples, no \alphat requirement is
made in order to increase the statistics and therefore the predictive power of 
the samples. This is possible as other requirements, in particular the 
requirement of
a single muon and a specific invariant mass window, greatly reduce any potential
contamination from QCD MJ events. The viability of this is specifically tested 
by dedicated closure tests described later in Section\ref{sec:closure_tests}.

Example distributions of $\mu$ \Pt and $\mu\text{-corrected}$ \met for this selection
are shown in Figure~\ref{fig:datamc_mu_inc}.


\subsection{\mmj}
The \mmj sample is constructed to predict background contributions from \zinv 
decays, mimicking this decay via the kinematically similar $Z\to\mu\mu + jets$
process where both muons are subsequently ignored.
The sample is used to provide low \HT coverage for the \zinv background 
prediction, where the \gj sample (Section\ref{sec:gjets_control_sample})
is unable to do so.

\subsubsection{Triggers}
The trigger used is the same as for the \mj sample, as described in
Section\ref{sec:mujets_control_trigger}. Trigger efficiencies are significantly
improved for the dimuon selection given that either of the muons 
can cause a positive trigger decision, as shown in Table~\ref{tab:dimuon-trig-effs}. 
Systematic errors are considered of the same magnitude as for the \mj trigger
efficiencies.

% \begin{table}[!ht]
%   \caption{Muon trigger efficiencies (\%) for the \mmj selection listed by
%   \HT bin and \nj category.}
%   \label{tab:dimuon-trig-effs}
%   \centering
%   \small
%   \begin{tabular}{ cccc }
%     \hline
%     \hline
%     \HT (GeV) & 2-3 & $\geq$4 \\ [0.5ex]
                                       
%     \hline
%     150--200  & 98.4 & 98.4  \\
%     200--275  & 98.5 & 98.4  \\
%     275--325  & 98.5 & 98.4  \\
%     325--375  & 98.6 & 98.6  \\
%     375--475  & 98.6 & 98.5  \\
%     475--575  & 98.6 & 98.6  \\
%     575--675  & 98.6 & 98.6  \\
%     675--775  & 98.7 & 98.6  \\
%     775--875  & 98.6 & 98.6  \\
%     875--975  & 98.7 & 98.6  \\
%     975--1075 & 98.7 & 98.8  \\
%     $>$1075   & 98.7 & 98.7  \\
%     \hline
%     \hline
%   \end{tabular}
% \end{table}

\begin{table}[!h]
  \caption{Muon trigger efficiencies (\%) for the \mmj selection listed by \HT bin and
  \nj category.}
  \label{tab:dimuon-trig-effs}
  \centering
  \footnotesize
  \begin{tabular}{ l|cccccccccccc }
    \hline
    \hline
    \multirow{2}{*}{\nj} & \multicolumn{10}{c}{HT bin low edge (GeV)} \\
    & 150 & 200 & 275 & 325 & 375 & 475 & 575 & 675 & 775 & 875 & 975 & 1075 \\
    \hline
    2--3 & 98.4 & 98.5 & 98.5 & 98.6 & 98.6 & 98.6 & 98.6 & 98.7 & 98.6 & 98.7 &
    98.7 & 98.7 \\
    $\geq 4$ & 98.4 & 98.4 & 98.4 & 98.6 & 98.5 & 98.6 & 98.6 & 98.6 & 98.6 & 98.6 & 98.8 &
    98.7 \\
    \hline
  \end{tabular}
\end{table}

\subsubsection{Selection Criteria}

\begin{figure}[t]
  \centering
    \begin{subfigure}[b]{0.48\textwidth}
      \includegraphics[width=\textwidth]{Figs/datamc/mumu/Stacked_DiMuon_Mass_all_DiMuon_200_upwards}
      \caption{$M_T(\mu, \mu)$}
    \end{subfigure}
    \begin{subfigure}[b]{0.48\textwidth}
      \includegraphics[width=\textwidth]{Figs/datamc/mumu/Stacked_SecondMuPt_all_DiMuon_200_upwards}
      \caption{Second $\mu$ \Pt}
    \end{subfigure} \\
    \caption{\label{fig:datamc_mumu_inc}
    Comparison of data with MC for the \mmj control selection. Plots 
    are for $\HT>200$~\gev, $\nj\geq2, \nb\geq0$.}
\end{figure}

The selection for the \mmj sample is very similar to that of the \mj sample, 
described in Section\ref{sec:mujets_control_selection}, with differences chosen
to enrich the sample in Z bosons decaying to pairs of muons in the kinematic 
phase space of the signal region. Two tight isolated muons are selected, each 
with $\Pt > 30$~\gev and $|\eta| < 2.1$. Their invariant mass is required to be
tight around $m_Z$, $m_Z - 25 < M_{\mu_1\mu_2} < m_Z + 25$~\gev. Furthermore, a 
veto is made on events satisfying $\Delta R(\mu_i, jet_j) < 0.5$, for every muon 
$i$ and every jet $j$. Similarly as in the \mj sample selection, no \alphat
requirement is made.

Example distributions of transverse mass of the muon pair, $M_T(\mu\mu)$, and
$\mu_2$ \Pt for this selection are shown in Figure~\ref{fig:datamc_mumu_inc}.

\subsection{\gj}
\label{sec:gjets_control_sample}
The \gj sample is used to predict the \zinv background contribution, given it's 
similar kinematics when the $\gamma$ is ignored from the event, as well as a
larger
production cross section relative to \mmj. Due to trigger thresholds, the \gj
sample cannot make predictions for $\HT<375$~\gev, and so is complimentary to
the \mmj sample prediction.

\subsubsection{Triggers}
Events are collected using the \verb!HLT_Photon150! trigger. The trigger's 
efficiency is measured using
a procedure equivalent to the method used for the signal triggers, described in
Section\ref{sec:signal_triggers}, with the \verb!HLT_Photon90! trigger as a reference.
The trigger
is found to be 100$\%$ efficient for $E_T^{\gamma}>165$~\gev and $\HT>375$~\gev,
as shown by the turn on curves in Figure~\ref{fig:photon_control_trigeff}.

\begin{figure}[t]
  \centering
  \begin{subfigure}[b]{0.35\textwidth}
    \includegraphics[width=\textwidth, page=3,trim=40 40 160 120,clip=true]{Figs/trigger/g_barrel_375_caloJet_le3j.pdf}
    \caption{\njlow}
    \label{fig:photon_control_trigeff_le3j}
  \end{subfigure}
  \begin{subfigure}[b]{0.35\textwidth}
    \includegraphics[width=\textwidth, page=3,trim=40 40 160 120,clip=true]{Figs/trigger/g_barrel_375_caloJet_ge4j.pdf}
    \caption{\njhigh}
    \label{fig:photon_control_trigeff_ge4j}
  \end{subfigure}
  \caption{Efficiency turn-on curves for the photon trigger, based on the \gj 
  selection, for \HT > 375~\gev, with \njlow (Left) and \njhigh(Right).}
  \label{fig:photon_control_trigeff}
\end{figure}

\subsubsection{Selection Criteria}
Exactly one photon satisfying tight isolation criteria is required, with 
$\Pt > 165$~\gev and $|\eta|<1.45$. In addition, events are vetoed if
$\Delta R(\gamma, jet_i)<1.0$ is satisfied, for all jets $i$ in the event.

An example distribution of the $\gamma$ \Pt for this selection is shown
in Figure~\ref{fig:datamc_pho_inc}.

\begin{figure}[!ht]
  \centering
    \begin{subfigure}[b]{0.48\textwidth}
      \includegraphics[width=\textwidth]{Figs/datamc/pho/Stacked_PhotonPt_all_Photon_375_upwards}
      \caption{$\gamma$ $\Pt$}
    \end{subfigure}
    \caption{\label{fig:datamc_pho_inc}
    Comparison of data with MC for the \gj control selection. Plot 
    is for $\HT>200$~\gev, $\nj\geq2, \nb\geq0$.}
\end{figure}

%********************************** % Third Section  *************************************
% \section{Estimating multijet backgrounds}  %Section - 1.3
% \label{sec:background_qcd}

% A data-driven technique has been developed to measure any remaining QCD events
% in the signal region. This is used to determine a \HT dependent \alphat
% requirement such that QCD is at the sub-percent level with respect to the total
% EWK background.

% \subsection{Multijet control sample}
% In order to predict the contamination of QCD MJ events in the signal region a 
% control region of hadronic events is constructed.

% \subsubsection{Triggers}
% Events are collected using a suite of triggers requiring various thresholds of 
% \HT. By using single-object \HT triggers it is possible to study events
% across a spectrum of \alphat values. The high rates expected through these
% triggers were maintained throughout
% the 2012 run with a variety of prescales. Each trigger seeds a single \HT bin in
% the analysis, with a 25 \gev offset between the online trigger requirement and 
% the offline threshold. Efficiencies are measured with the same technique as the
% signal triggers, using the \verb!HLT_IsoMu24_eta2p1! trigger as a reference. 
% The efficiencies of these triggers are summarised in Table~\ref{tab:ht-triggers}.

% \begin{table}[!ht]
%   \caption{List of \texttt{HTxxx} triggers and their efficiencies
%     (\%), as measured in data for each \HT bin and \nj category. Also listed are
%     the typical prescales applied and the L1 seed triggers.}
%   \label{tab:ht-triggers}
%   \centering
%   \scriptsize
%   \begin{tabular}{ ccccll }
%     \hline
%     \hline
%     Offline \HT & L1 seed (\verb!L1_?!) & Trigger (\verb!HLT_?!) &    Typical & \multicolumn{2}{c}{Efficiency (\%)}\\ [0.5ex]
%    region (\gev) & (highest thresholds) &  & prescale & \multicolumn{1}{c}{$2 \leq \nj \leq 3$} & \multicolumn{1}{c}{$\nj \geq 4$} \\ [0.5ex]

%     \hline
%     $200 < \HT < 275$  & \verb!DoubleJetC64!           & \verb!HT250! & 4800     & $\phantom{1}66.4 \pm 14.1$                & $154.3 \pm 154.3$                     \\
%     $275 < \HT < 325$  & \verb!DoubleJetC64 OR HTT175! & \verb!HT250! & 2400     & $\phantom{1}97.3 \pm 23.0$                & $\phantom{1}91.7 \pm \phantom{1}53.1$ \\
%     $325 < \HT < 375$  & \verb!DoubleJetC64 OR HTT175! & \verb!HT300! & 1200     & $\phantom{1}79.5 \pm 20.6$                & $198.1 \pm \phantom{1}81.2$           \\
%     $375 < \HT < 475$  & \verb!DoubleJetC64 OR HTT175! & \verb!HT350! & 600      & $108.7 \pm 18.7$                          & $\phantom{1}54.5 \pm \phantom{1}31.6$ \\
%     $475 < \HT < 575$  & \verb!DoubleJetC64 OR HTT175! & \verb!HT450! & 150      & $110.6 \pm 15.9$                          & $106.4 \pm \phantom{1}26.8$           \\
%     $575 < \HT < 675$  & \verb!DoubleJetC64 OR HTT175! & \verb!HT550! & 70       & $\phantom{1}96.1 \pm 14.7$                & $104.4 \pm \phantom{1}23.1$           \\
%     $675 < \HT < 775$  & \verb!DoubleJetC64 OR HTT175! & \verb!HT650! & 25       & $\phantom{1}94.3 \pm 15.4$                & $101.2 \pm \phantom{1}21.5$           \\
%     $775 < \HT < 875$  & \verb!DoubleJetC64 OR HTT175! & \verb!HT750! & 1        & $\phantom{1}96.9 \pm \phantom{1}6.1$      & $\phantom{1}94.4 \pm \phantom{11}8.3$ \\
%     $875 < \HT < 975$  & \verb!DoubleJetC64 OR HTT175! & \verb!HT750! & 1        & $100.0 \pm \phantom{1}8.4$                & $100.0 \pm \phantom{1}12.6$           \\
%     $975 < \HT < 1075$ & \verb!DoubleJetC64 OR HTT175! & \verb!HT750! & 1        & $100.0 \pm 11.2$                          & $100.0 \pm \phantom{1}15.3$           \\
%     $\HT > 1075$       & \verb!DoubleJetC64 OR HTT175! & \verb!HT750! & 1        & $100.0 \pm 15.0$                          & $100.0 \pm \phantom{1}22.9$           \\
%     \hline
%     \hline
%   \end{tabular}
% \end{table}

% \subsubsection{Selection Criteria}
% The selection of this sample matches that of the hadronic signal region, with 
% the exception that both the \alphat and \mhtmet requirements are removed 
% ensuring a very high yield of QCD MJ events.


% \subsection{Prediction Technique}

% Events from the hadronic control sample are used to populate a plane of \mhtmet 
% and \alphat. A prediction of the EWK contribution to this sample is determined 
% from the \mj control sample with the \mhtmet requirement removed, using the TF 
% factor method described in Section\ref{sec:background_overview}. The prediction
% is subtracted from the hadronic control sample yields as a function of \mhtmet
% and \alphat, leaving a pure sample of QCD MJ events (SHOW PLOTS?).

% By considering events both above and below the nominal \mhtmet threshold of
% 1.25, the ratio \rmhtmet is constructed, defined as:
% % 
% \begin{equation}
% \label{eq:rmhtmet}
% \rmhtmet = \frac{N(\mhtmet<1.25)}{N(\mhtmet>1.25)}
% \end{equation}

% A fit is made to this variable as a function of \alphat, using an exponential 
% functional form:
% % 
% \begin{equation}
% \label{eq:fit_exp}
% \rmhtmet(\alphat) = e^{{-(a+b.\alphat)}^n}
% \end{equation}
% % 
% both for $n=0$ and also when $n$ is allowed to float as a free 
% parameter within the range $0-2$ in order to span the scenarios from flat to a
% Gaussian form. Example distributions and fits are shown in FIGURE.

% QCD yield predictions are made and compared with the relevant EWK background 
% contributions in Table~\ref{tab:qcd-pred-data}. \alphat thresholds are chosen 
% for each \HT bin such that the ratio of QCD to EWK is at the sub-percent level. 
% The chosen \alphat threshold values are summarised in
% Table~\ref{tab:alphat_thresholds_qcd}.

% \begin{table}[h!]
% \centering
% \scriptsize
%   \caption{QCD multijet background contribution prediction summarised for the 
%   main analysis categories of \nb, \nj and \HT as a function of \alphat. The 
%   predicted EWK background contribution is included for comparison, and a 
%   the ratio of QCD/EWK is also shown.}
% \label{tab:qcd-pred-data}
% \begin{tabular}{ccccrrr}
% \hline
% \hline
% \nj & \nb & \HT (GeV) & Bkgd & \multicolumn{3}{c}{\alphat threshold} \\
% \cline{5-7}
%  & & & & \multicolumn{1}{c}{0.550}   & \multicolumn{1}{c}{0.600}   & \multicolumn{1}{c}{0.650} \\
% \hline
% 2--3 & 0 & 200--275 & QCD  & $\left(3.8 \pm 1.3 \pm 1.2 \right) \times 10^{3}$ & $\left(4.1 \pm 2.4 \pm 3.0 \right) \times 10^{1}$ & $\left(0.9 \pm 0.8 \pm 1.5 \right) \times 10^{0}$\\
% 2--3 & 0 & 200--275 & EWK  & $\left(2.1 \pm 0.1\right) \times 10^{4}$ & $\left(1.5 \pm 0.0\right) \times 10^{4}$ & $\left(1.2 \pm 0.0\right) \times 10^{4}$\\
% 2--3 & 0 & 200--275 & Ratio  & $0.2 \pm 0.1$ & $0.003 \pm 0.003$ & $0.0001 \pm 0.0001$\\ [1.0ex]
% 2--3 & 0 & 275--325 & QCD  & $\left(1.0 \pm 0.3 \pm 1.5 \right) \times 10^{4}$ & $\left(0.2 \pm 0.1 \pm 0.7 \right) \times 10^{0}$ & $\left(0.8 \pm 0.3 \pm 4.8 \right) \times 10^{-3}$\\
% 2--3 & 0 & 275--325 & EWK  & $\left(7.9 \pm 0.2\right) \times 10^{3}$ & $\left(5.3 \pm 0.2\right) \times 10^{3}$ & $\left(4.0 \pm 0.2\right) \times 10^{3}$\\
% 2--3 & 0 & 275--325 & Ratio  & $1 \pm 2$ & $0.0000 \pm 0.0001$ & $\left(0 \pm 1\right) \times 10^{-6}$\\ [1.0ex]
% 2--3 & 0 & 325--375 & QCD  & $\left(2.8 \pm 0.4 \pm 2.1 \right) \times 10^{1}$ & $\left(0.9 \pm 0.2 \pm 1.3 \right) \times 10^{-1}$ & $\left(0.6 \pm 0.4 \pm 1.2 \right) \times 10^{-3}$\\
% 2--3 & 0 & 325--375 & EWK  & $\left(3.4 \pm 0.1\right) \times 10^{3}$ & $\left(2.2 \pm 0.1\right) \times 10^{3}$ & $\left(1.7 \pm 0.1\right) \times 10^{3}$\\
% 2--3 & 0 & 325--375 & Ratio  & $0.008 \pm 0.006$ & $\left(4 \pm 6\right) \times 10^{-5}$ & $\left(4 \pm 8\right) \times 10^{-7}$\\ [1.0ex]
% $\geq 4$ & 0 & 200--275 & QCD  & $\left(2.8 \pm 1.5 \pm 1.3 \right) \times 10^{3}$ & $\left(1.1 \pm 0.7 \pm 0.2 \right) \times 10^{1}$ & $\left(0.2 \pm 0.2 \pm 0.0 \right) \times 10^{0}$\\
% $\geq 4$ & 0 & 200--275 & EWK  & $\left(4.3 \pm 0.3\right) \times 10^{2}$ & $\left(2.0 \pm 0.1\right) \times 10^{2}$ & $\left(1.0 \pm 0.1\right) \times 10^{2}$\\
% $\geq 4$ & 0 & 200--275 & Ratio  & $7 \pm 5$ & $0.06 \pm 0.04$ & $0.002 \pm 0.002$\\ [1.0ex]
% $\geq 4$ & 0 & 275--325 & QCD  & $\left(1.5 \pm 1.2 \pm 1.0 \right) \times 10^{4}$ & $\left(1.5 \pm 1.7 \pm 0.3 \right) \times 10^{0}$ & $\left(0.1 \pm 0.2 \pm 0.0 \right) \times 10^{-1}$\\
% $\geq 4$ & 0 & 275--325 & EWK  & $\left(1.2 \pm 0.0\right) \times 10^{3}$ & $\left(5.3 \pm 0.2\right) \times 10^{2}$ & $\left(2.9 \pm 0.1\right) \times 10^{2}$\\
% $\geq 4$ & 0 & 275--325 & Ratio  & $\left(1 \pm 1\right) \times 10^{1}$ & $0.003 \pm 0.003$ & $\left(4 \pm 7\right) \times 10^{-5}$\\ [1.0ex]
% $\geq 4$ & 0 & 325--375 & QCD  & $\left(0.7 \pm 0.1 \pm 0.8 \right) \times 10^{0}$ & $\left(2.5 \pm 0.6 \pm 6.5 \right) \times 10^{-5}$ & $\left(0.7 \pm 0.3 \pm 2.8 \right) \times 10^{-8}$\\
% $\geq 4$ & 0 & 325--375 & EWK  & $\left(4.8 \pm 0.3\right) \times 10^{2}$ & $\left(2.0 \pm 0.1\right) \times 10^{2}$ & $\left(1.1 \pm 0.1\right) \times 10^{2}$\\
% $\geq 4$ & 0 & 325--375 & Ratio  & $0.002 \pm 0.002$ & $\left(1 \pm 3\right) \times 10^{-7}$ & $\left(1 \pm 3\right) \times 10^{-10}$\\ [1.0ex]
% %2--3 & $\geq 1$ & 200--275 & QCD  & $\left(2.2 \pm 1.1 \pm 4.5 \right) \times 10^{2}$ & $\left(0.5 \pm 0.2 \pm 3.2 \right) \times 10^{0}$ & $\left(0.3 \pm 0.1 \pm 4.5 \right) \times 10^{-2}$\\
% %2--3 & $\geq 1$ & 200--275 & EWK  & $\left(3.7 \pm 0.1\right) \times 10^{3}$ & $\left(2.5 \pm 0.1\right) \times 10^{3}$ & $\left(1.9 \pm 0.1\right) \times 10^{3}$\\
% %2--3 & $\geq 1$ & 200--275 & Ratio  & $0.1 \pm 0.1$ & $0.000 \pm 0.001$ & $\left(0 \pm 2\right) \times 10^{-5}$\\ [1.0ex]
% \hline
% \hline
% \end{tabular}
% \end{table}

% \emph{RELATED SYSTEMATICS?}

% \begin{table}[!ht]
%   \caption{\alphat thresholds for each analysis \HT bin as determined from the 
%   QCD MJ prediction method, such that QCD is at the sub-percent level.}
%   \label{tab:alphat_thresholds_qcd}
%   \centering
%   \small
%   \begin{tabular}{ cc }
%     \hline
%     \hline
%     \HT (GeV) & \alphat threshold \\ [0.5ex]
                                       
%     \hline
%     200--275  & 0.65 \\
%     275--325  & 0.60 \\
%     >325  & 0.55 \\
%     \hline
%     \hline
%   \end{tabular}
% \end{table}


%********************************** % Fifth Section  *************************************
\section{Systematic uncertainties on SM background predictions}  %Section - 1.5
\label{sec:background_systematics}

In order to probe the levels at which the transfer factors are sensitive to 
relevant uncertainties, a statistically powerful ensemble of Closure Tests
(CT's) have been designed. The CT method works by constructing a TF to
extrapolate from one sub-region of a particular control sample into another 
control sample sub-region. This prediction method is identical to that described
in Section\ref{sec:background_overview} with the difference that a prediction is
made from a control region to another control region, rather than to the signal
region. In doing so, tests can be
designed to specifically probe any potential sources of bias in the transfer factors.

\subsection{Closure tests}
\label{sec:closure_tests}

Closure tests are performed as a function of \HT, in the two \nj categories,
\njlow and \njhigh. The level of closure is represented by the statistical 
consistency between predicted and observed yields for each test, in the absence 
of any assumed systematic uncertainty. The test statistic is defined as $(N_{obs}
- N_{pred}) / N_{pred}$, with any bias being observed as a statistically 
significant deviation from zero, or trend in \HT.

\begin{figure}[ht!]
  \centering
  \begin{subfigure}[b]{0.7\textwidth}
    \includegraphics[width=\textwidth]{Figs/syst/v0/le3j/summary_plot_img}
    \caption{$2 \leq \nj \leq 3$}
    \label{fig:closure_summary_le3j}
  \end{subfigure}             
  \begin{subfigure}[b]{0.7\textwidth}
    \includegraphics[width=\textwidth]{Figs/syst/v0/ge4j/summary_plot_img}
    \caption{$\nj \geq 4$}
    \label{fig:closure_summary_ge4j}
  \end{subfigure}
  \caption{The results of the eight core closure tests (open symbols), shown 
  over the systematic uncertainty bands for each of the five \HT regions
  (shaded grey), for the two jet multiplicity regions (a) \njlow and (b) \njhigh.}
  \label{fig:closure_summary}
\end{figure}

Figure~\ref{fig:closure_summary} shows a summary of the eight closure tests 
considered as `core' tests for the analysis, split into both
\njlow (Figure~\ref{fig:closure_summary_le3j}) 
and \njhigh (Figure~\ref{fig:closure_summary_ge4j}). It should be noted that 
numerous other tests are also considered, but that these eight represent those 
deemed most important to the background prediction and are therefore used to
derive related uncertainties.

The first test, represented by open circles, tests the modelling of the \alphat 
variable in the \mj control sample. In the analysis a prediction is made 
between the \mj sample, which has no \alphat requirement, and the 
signal region, with it's tight \alphat requirement. This particular test probes
the validity of the prediction between the `bulk' of the \alphat distribution in
the control sample
and the `tail' of the distribution in the signal region. A similar test, not
shown here, is performed for the \mmj control sample.

The next two tests, represented by crosses and open squares, probe the
different b tag multiplicities in the \mj control sample. The b tag 
requirements greatly changes the relative admixture of, for example, \wj (\nb=0)
and \ttj (\nb=1) events.
It is important to note that this test is 
considered conservative, given that the admixture of \wj to \ttj events 
varies minimally between control and signal regions, where this extrapolation is 
made between identical \nb categories in the analysis. Given the focus on b
tagging, these tests also investigate the modelling of b quark jets in the
simulated data.

Represented by open triangles, a similar test is made for the relative
admixture of \zj to \wj and \ttj, by  predicting between the \mj and \mmj
control regions. Again, this test is considered conservative, and also probes
the muon reconstruction and trigger efficiencies between the different muon
multiplicities. These are however already well studied by the muon \emph{POG}
using data-driven techniques.

As described in Section\ref{sec:background_overview}, the \zinv prediction 
is made from both the \gj and \mmj samples, and so a test is constructed to 
predict between these two orthogonal control regions, as shown by the open 
crosses.

The final three tests, indicated by open stars, triangles and diamonds, make 
predictions between the two different jet multiplicity categories in each 
control sample, thereby 
testing jet reconstruction and modelling. These tests are also considered 
conservative as the analysis only predicts between identical \nj categories in 
the control and signal regions.

Summary plots of these eight tests are shown in
Figure~\ref{fig:closure_summary}, indicating no statistically significant biases
or \HT dependencies. Figures~\ref{fig:closure_fit_le3j_pol0} and \ref{fig:closure_fit_ge4j_pol0} show
zeroeth order polynomial fits (blue lines) made to each individual test to assess the 
level of any potential bias present. In addition, first order polynomial fits 
(red lines) are made to assess any potential \HT dependence present in the
tests, as shown
in figures~\ref{fig:closure_fit_le3j_pol1} and \ref{fig:closure_fit_ge4j_pol1}.
The best-fit values, $\chi^2$ and $p$-values 
obtained from both fits are summarised for each \nj category in
tables~\ref{tab:syst-fits-le3j}, \ref{tab:syst-fits-ge4j} and
\ref{tab:syst-fits-njet}.

The fits show no significant biases or trends and therefore indicate good
closure. The only exception is the 0 b-jets \ra 1 b-jet (\mj) test for the
\njhigh category which has a sub-optimal goodness of fit value. This is 
attributed to upwards and downwards fluctuations in the adjacent 475-575~\gev
and 575-675~\gev bins respectively. Also shown in Table~\ref{tab:syst-fits-ge4j},
when the same fit is made after summing these two bins significantly improved
fit parameters are observed. This leads to the conclusion that these two
bins contain a statistical fluctuation as opposed to a systematic bias.

\begin{figure}[p]
  \centering
  \begin{subfigure}[b]{0.9\textwidth}
    \includegraphics[width=\textwidth]{Figs/syst/v0/le3j/summary_plot_pol0_img}
    \caption{Constant function}
    \label{fig:closure_fit_le3j_pol0}
  \end{subfigure}
  \begin{subfigure}[b]{0.9\textwidth}
    \includegraphics[width=\textwidth]{Figs/syst/v0/le3j/summary_plot_pol1_img}
    \caption{Linear function}
    \label{fig:closure_fit_le3j_pol1}
  \end{subfigure}
  \caption{The results of the eight core closure tests (open symbols), shown 
  over the systematic uncertainty bands for each of the five \HT regions
  (shaded grey), for \njlow. Included are zeroeth order (left column, blue)
  and first order (right column, red) fits to each individual closure test.}
  \label{fig:closure_fits_le3j}
\end{figure}

\begin{figure}[p]
  \begin{subfigure}[b]{0.9\textwidth}
    \includegraphics[width=\textwidth]{Figs/syst/v0/ge4j/summary_plot_pol0_img}
    \caption{Constant function}
    \label{fig:closure_fit_ge4j_pol0}
  \end{subfigure}
  \begin{subfigure}[b]{0.9\textwidth}
    \includegraphics[width=\textwidth]{Figs/syst/v0/ge4j/summary_plot_pol1_img}
    \caption{Linear function}
    \label{fig:closure_fit_ge4j_pol1}
  \end{subfigure}
  \caption{The results of the eight core closure tests (open symbols), shown 
  over the systematic uncertainty bands for each of the five \HT regions
  (shaded grey), for \njhigh. Included are zeroeth order (left column, blue)
  and first order (right column, red) fits to each individual closure test.}
  \label{fig:closure_fits_ge4j}
\end{figure}

\begin{table}[!ht]
  \caption{Results of zeroeth (i.e. constant) and first order (i.e. linear) fits
  for five sets of closure tests, performed in the \njlow category.}
  \label{tab:syst-fits-le3j}
  \centering
  \tiny
  \begin{tabular}{ llrccccrc }
    \hline
    \hline
                                              &          & \multicolumn{4}{c}{Constant fit} &          & \multicolumn{2}{c}{Linear fit}                        \\
    \cline{3-6}\cline{8-9}                                                                  
    Closure test                              & Symbol   & Best fit value                   & $\chi^2$ & d.o.f. & $p$-value &  & Slope ($10^{-4}$) & $p$-value \\
    \hline                                                                                                                                  
    $\alphat < 0.55 \ra \alphat > 0.55$ (\mj) & Circle   & $-0.02 \pm 0.01$                 & 11.3     & 10     & 0.34      &  & $-2.9 \pm 1.1$    & 0.83      \\ 
    0 b-jets \ra 1 b-jet (\mj)                & Times    & $ 0.04 \pm 0.01$                 & 5.8      & 10     & 0.83      &  & $-1.5 \pm 0.9$    & 0.97      \\ 
    1 b-jet \ra 2 b-jets (\mj)                & Square   & $-0.03 \pm 0.02$                 & 5.3      & 10     & 0.87      &  & $-3.0 \pm 1.7$    & 0.99      \\ 
    \mj \ra \mmj                              & Triangle & $ 0.03 \pm 0.02$                 & 12.3     & 10     & 0.27      &  & $-1.3 \pm 1.1$    & 0.28      \\ 
    \gj \ra \mmj                              & Cross    & $-0.02 \pm 0.03$                 & 3.0      & 7      & 0.88      &  & $ 0.0 \pm 2.7$    & 0.81      \\ 
    \hline
    \hline
  \end{tabular}
\end{table}

\begin{table}[!ht]
  \caption{Results of zeroeth (i.e. constant) and first order (i.e. linear) fits
  for five sets of closure tests, performed in the \njhigh category. An
  additional test marked with a $\dag$ is listed, as described in the text.}
  \label{tab:syst-fits-ge4j}
  \centering
  \tiny
  \begin{tabular}{ llrccccrc }
    \hline
    \hline
                                              &          & \multicolumn{4}{c}{Constant fit} &          & \multicolumn{2}{c}{Linear fit}                        \\
    \cline{3-6}\cline{8-9}                                                                  
    Closure test                              & Symbol   & Best fit value                   & $\chi^2$ & d.o.f. & $p$-value &  & Slope ($10^{-4}$) & $p$-value \\
    \hline                                                                                                                                 
    $\alphat < 0.55 \ra \alphat > 0.55$ (\mj) & Circle   & $-0.02 \pm    0.02$              & 17.6     & 10     & 0.06      &  & $-3.1 \pm 1.7$    & 0.11      \\ 
    0 b-jets \ra 1 b-jet (\mj)                & Times    & $-0.06 \pm 0.02$                 & 31.2     & 10     & 0.00      &  & $-4.1 \pm 1.2$    & 0.02      \\ 
    0 b-jets \ra 1 b-jet (\mj)$^{ \dag}$      & Times    & $-0.05 \pm 0.02$                 & 13.4     & 9      & 0.15      &  & $-3.9 \pm 1.3$    & 0.78      \\ 
    1 b-jet \ra 2 b-jets (\mj)                & Square   & $ 0.06 \pm    0.02$              & 13.7     & 10     & 0.19      &  & $ 2.5 \pm 1.6$    & 0.28      \\ 
    \mj \ra \mmj                              & Triangle & $ 0.11 \pm    0.05$              & 4.8      & 10     & 0.90      &  & $ 0.4 \pm 2.7$    & 0.85      \\ 
    \gj \ra \mmj                              & Cross    & $-0.00 \pm 0.07$                 & 2.3      & 7      & 0.94      &  & $-5.3 \pm 4.7$    & 0.99      \\ 
    \hline
    \hline
  \end{tabular}
\end{table}

\begin{table}[!ht]
  \caption{Results of zeroeth (i.e. constant) and first order (i.e. linear) fits
  for the three sets of closure tests probing the accuracy of jet multiplicity 
  modelling in MC, for each control sample.} 
  \label{tab:syst-fits-njet}
  \centering
  \footnotesize
  \begin{tabular}{ llrccccrc }
    \hline
    \hline
           &                   & \multicolumn{4}{c}{Constant fit} &          & \multicolumn{2}{c}{Linear fit}                        \\
    \cline{3-6}\cline{8-9}
    Sample & Symbol            & Best fit value                   & $\chi^2$ & d.o.f. & $p$-value &  & Slope ($10^{-4}$) & $p$-value \\
    \hline                                                                                                            
    \mj    & Star              & $-0.08 \pm 0.01$                 & 9.3      & 10     & 0.50      &  & $0.6 \pm 0.7$     & 0.48      \\ 
    \gj    & Inverted triangle & $ 0.09 \pm 0.04$                 & 3.7      & 7      & 0.82      &  & $5.1 \pm 3.2$     & 0.98      \\ 
    \mmj   & Diamond           & $-0.00 \pm 0.05$                 & 4.7      & 10     & 0.91      &  & $2.5 \pm 2.9$     & 0.92      \\ 
    \hline
    \hline
  \end{tabular}
\end{table}

\subsection{Background systematic uncertainty summary}
Under the assumption of closure for the eight core tests,
systematic errors on the EWK background prediction are derived for each \nj
category in seven regions of \HT.
Values are calculated by summing in quadrature the weighted mean and sample 
variance for all eight tests in a given \HT region, in order to ensure coverage of all
test points in that region. These values are summarised 
in Table~\ref{tab:syst_values} and also in the summary plots of Figure~\ref{fig:closure_summary},
shown as grey bands.

\begin{table}[!ht]
  \caption{Summary of the magnitude of systematic uncertainties (\%) on the EWK
  background prediction, derived 
  from the eight core closure tests, for each \nj category and \HT region.}
  \label{tab:syst_values}
  \centering
  \footnotesize
  \begin{tabular}{ cccccccc }
    \hline
    \hline
            & \multicolumn{7}{c}{\HT region (GeV)}                                \\
    \cline{2-8}
    \nj   & 200--275 & 275--325 & 325--375 & 375--575 & 575--775 & 775-975 & $>975$ \\
    \hline                                                                                                                                  
    2--3    & 4        & 6        & 6        & 8        & 12       & 17      & 19     \\
    $\geq$4 & 6        & 6        & 11       & 11       & 18       & 20      & 26     \\
    \hline                                                                                                                                  
    \hline
  \end{tabular}
\end{table}

Systematic uncertainties are considered as fully uncorrelated between the 
different analysis categories and the \HT regions. As some correlation between,
for example, adjacent \HT bins may be expected, this is again considered as a
conservative approach as the fit gains more freedom to accommodate the background
prediction shape within the data observation.


%********************************** % blah Section  *************************************

\section{Eliminating QCD events}
\label{sec:qcd_cleaning}

While the \alphat requirement removes many orders of magnitude of QCD events,
there still exist scenarios in which these events may pass the signal region
selection. Accordingly, further requirements are made to ensure the search
region is free of any residual QCD contamination, such that it is negligible
when compared to the uncertainty on the EWK background prediction.

\subsection{Multiple jets below threshold}
\label{sec:qcd_cleaning_below_thresh}

Events are able to acquire non-negligible amounts of \mht without the presence
of
real \met if multiple jets are below the jet \Pt threshold and their
configuration conspires to form a topology that gives high values of \alphat.
Such events will contain a disparity between the \met and
\mht variables, given that the former is reconstructed using energy deposits
and particle flow techniques, while the latter is reconstructed with jet objects.
Figure~\ref{fig:full_mhtmet_distro} shows the contribution of QCD at
high \mhtmet values, even following the \alphat requirement.
To protect against this scenario, events are required to have \mhtmet < 1.25.

\begin{figure}[t]
\centering
\includegraphics[width=0.6\textwidth]
{Figs/datamc/had/v1/Stacked_MHTovMET_all_200_upwards.png}
\caption{The \mhtmet distribution of MC events following the hadronic selection
criteria, minus the nominal \mhtmet requirement.The MC yields are stacked,
with the QCD contribution shown in cyan. The plot is for a fully inclusive
selection of $\nb \geq 0$, $\nj \geq 2$ and \HT > 200~\gev.}
\label{fig:full_mhtmet_distro}
\end{figure}

\subsection{Instrumental effects}

It is possible to observe fake \mht if an event's jet is mismeasured. Such
mismeasurement can occur due to instrumental effects, such as if a jet
deposits energy in a region of the calorimeter system known to be damaged or
faulty (hereby referred to as `dead'). To protect against this, a filter is
employed that finds such a mismeasured jet, rejecting the event if
the jet overlaps with a dead region. This filter is known as the `Dead ECAL
Filter'. To detect such mismeasured jets, the 
variable \dphistar is used, defined for a jet $j$ as:
% 
\begin{equation}
\dphistar_j = \Delta \phi\big(\overrightarrow{\Pt}_j,-\sum_{i\neq j}
{\overrightarrow{\Pt}_i}\big) .
\label{eq:biasdphi}
\end{equation}
%
This quantity is the difference in $\phi$ between the jet and the missing
hadronic energy vector calculated using all other jets in the event. An advantage
of this variable with respect to the often used
$\Delta \phi(\overrightarrow{\Pt}_j, \mhtvect)$ is it's detection of spurious missing
energy vectors caused
by either under-measurements and over-measurements of a jets energy. In the case
of an over-measured jet the nominal \mhtvect would point in the opposite direction
to that jet, whereas \mhtvect calculated from all other jets would point in
the same direction, thereby giving a low \dphistar.

Events are vetoed if a jet with \dphistar< 0.5 is within $\Delta R < 0.3$ of a known
dead region of the calorimeter.



% Fake \mht may also be produced if jets overlap with areas of the calorimeter 
% system which are damaged or known to be faulty (hereby referred to as `dead'),
% and jets are mismeasured or lost as a result. To protect against this, for a
% given jet $j$ the angular separation between the event \mht, calculated
% excluding jet $j$, and the jet itself is defined as:
% % 
% \begin{equation}
% \dphistar_j = \Delta \phi\big(\overrightarrow{\Pt}_j,-\sum_{i\neq j}
% {\overrightarrow{\Pt}_i}\big) .
% \label{eq:biasdphi}
% \end{equation}
% % 
% An advantage of this variable with respect to the often used
% $\Delta R(\overrightarrow{\Pt}_j, \mht)$ is it's detection of spurious missing
% energy vectors
% caused
% by either under-measurements and over-measurements of a jets energy.
% A small value of $\dphistar_j$ indicates that the momentum vector of the jet $j$
% is aligned with the \mht vector, implying the jet to be mismeasured. Events are
% vetoed if a jet with \dphistar< 0.5 is within $\Delta R < 0.3$ of a known
% dead region of the calorimeter. This requirement shall hereby be described as
% the `Dead ECAL filter'.

To protect against further jet mismeasurements arising from
instrumental effects, multiple event filters (see Table~\ref{tab:met_filters})
are applied to account for well-understood detector issues. However, previously
undiscovered and therefore rare detector effects may still be
present. To check for such issues the
jet giving the minimum \dphistar value in an event, \mindphistar, is found
and a single entry of the $\eta$ and $\phi$ direction of the jet's axis is entered
into a map of the detector, as shown in Figure~\ref{fig:hotspots}. Any areas of
instrumental issue would
be visible as clusters of high event counts.
Figures~\ref{fig:hotspots_2d_nodeadECAL} and \ref{fig:hotspots_1d_nodeadECAL}
show the detector map and the 1D distribution
of counts before the Dead ECAL filter is applied, with areas of
potential instrumental defects clearly visible, notably as outliers in
1D distribution. Following the application of the Dead
ECAL filter the hotspot areas and the corresponding outliers are removed, as
seen in figures~\ref{fig:hotspots_2d_withdeadECAL} and
\ref{fig:hotspots_1d_withdeadECAL}. The lack of localised high-count regions or
the presence of a tail in the 1D distribution indicate there to be
no significant instrumental issues remaining.

\begin{figure}[h!]
  \begin{center}
    \begin{subfigure}[b]{0.46\textwidth}
      \begin{overpic}[width=\textwidth]{Figs/dphi/HT_dependent_AlphaT_thresholds/th2d_denom_summed_ge2j_ge0b_200.pdf}
        \put(10,60){\includegraphics[width=1.6cm]{Figs/results/v0/ht_white_cmsprelim_cover.png}}
      \end{overpic}
      \caption{No ``Dead ECAL filter''.}
      \label{fig:hotspots_2d_nodeadECAL}
    \end{subfigure}
    \begin{subfigure}[b]{0.46\textwidth}
      \includegraphics[width=\textwidth]{Figs/dphi/HT_dependent_AlphaT_thresholds/th1d_denom_summed_ge2j_ge0b_200.pdf}
      \caption{No ``Dead ECAL filter''.}
      \label{fig:hotspots_1d_nodeadECAL}
    \end{subfigure} \\ 
    \begin{subfigure}[b]{0.46\textwidth}
      \begin{overpic}[width=\textwidth]{Figs/dphi/Nominal_AlphaT_thresholds/th2d_numer_summed_ge2j_ge0b_200.pdf}
        \put(10,60){\includegraphics[width=1.6cm]{Figs/results/v0/ht_white_cmsprelim_cover.png}}
      \end{overpic}
      \caption{With ``Dead ECAL filter''.}
      \label{fig:hotspots_2d_withdeadECAL}
    \end{subfigure}
    \begin{subfigure}[b]{0.46\textwidth}
      \includegraphics[width=\textwidth]{Figs/dphi/Nominal_AlphaT_thresholds/th1d_numer_summed_ge2j_ge0b_200.pdf}
      \caption{With ``Dead ECAL filter''.}
      \label{fig:hotspots_1d_withdeadECAL}
    \end{subfigure} \\ 
    \caption{Distribution of jets in ($\eta$, $\phi$)-space that are
      responsible for the \mindphistar value of an event, with (a, b) and
      without (c, d) the ``Dead ECAL filter''
      requirement applied as part of the signal region selection.}
    \label{fig:hotspots}
  \end{center}
\end{figure}

\subsection{Heavy-flavour jet decays}

Jets can also appear to be mismeasured if the parton
shower contains heavy flavour mesons which decay semi-leptonically. In rare
circumstances, these decays can give the largest fraction of the available
momenta to the subsequent neutrino, leading to significant amounts of real \met
and soft-leptons which can evade the lepton vetoes.
This effect is compounded when multiple neutrinos are produced in the decay,
with a significant fraction of the jet's energy therefore evading detection.

An event display of a typical event is shown in Figure~\ref{fig:event_display_QCD}.
It is important to note the high amount, 185~\gev, of generator level
\met (`genmetP4True', thin pink arrow) pointing along the axis of a reconstructed
jet with $\Pt = 119$~\gev (bold blue arrow) - a configuration which,
coupled with multiple additional jets, can conspire to give
large values of \alphat. The sum of the generator level \met and
the \Pt of the jet equate to the \Pt of the generator level jet (bold black
arrow),
indicating that the missing energy of the event comes almost entirely from the
neutrinos in the decay of the jet. Furthermore, this implies the ratio
\mhtmet to be near unity, therefore allowing events to also evade the \mhtmet <
1.25 requirement of the signal region.

To better study events of this type a study region is defined, populated by
the the single-object \HT trigger, \verb!HLT_HT750!, which remained unprescaled
throughout \runone. As opposed to a typical signal trigger, the lack of an
\alphat requirement allows events with lower values of \alphat to be studied.
The region is
therefore defined by \HT > 775~\gev and 0.507 < \alphat < 0.55. Due to the
intrinsic correlation of \HT and \mht within the
\alphat variable (Figure~\ref{fig:alphat_mht_corr}), this selection provides
an effective \mht requirement similar to that of the low \HT categories of the
nominal analysis.

\begin{figure}[t]
  \centering
  % \includegraphics[width=0.6\textwidth]
  % {Figs/datapred/qcd_study_region/ge2j_ge0b_775_upwards/Prediction_ComMinBiasDPhi_acceptedJets_all_775_upwards_QCD}
  \begin{overpic}[width=0.6\textwidth]{Figs/datapred/qcd_study_region/ge2j_ge0b_775_upwards/Prediction_ComMinBiasDPhi_acceptedJets_all_775_upwards_QCD}
    \put(62,70){\includegraphics[width=1.3cm]{Figs/results/v0/ht_white_cmsprelim_cover.png}}
  \end{overpic}
  \caption{Data (black points) against the EWK background prediction 
  (stacked, yellow and purple) as a function of \mindphistar. The expected yield
  from QCD MC (cyan) is stacked on top of the EWK prediction, but not included
  in the ratio plot. The plot represents
  the QCD control study region, with $\nb \geq 0$, $\nj \geq 2$, $\HT > 775
 ~\gev$ and $\alphat > 0.507$.}
  \label{fig:qcd_region_pred_dphistar_incl}
\end{figure}

Jets containing a \met source appear as mismeasured and therefore
populate a region of low \dphistar (Equation~\ref{eq:biasdphi}).
Figure~\ref{fig:qcd_region_pred_dphistar_incl} shows the data compared to the
EWK background prediction
as a function of the \mindphistar value of each event. The disagreement observed
at low \mindphistar is well accounted for by the yield from QCD MC. However, it
should be noted that while simulated MC can provide a qualitative understanding
of the QCD
contamination, it should not be relied upon to determine a quantitative
understanding of the phenomenon.

As motivated by Figure~\ref{fig:qcd_region_pred_dphistar_incl}, the residual QCD
events appear to be well isolated in the region $\mindphistar < 0.3$. The effect
of applying this threshold in the QCD control study region is shown in
Figure~\ref{fig:data_pred_dphistar_eff}.
% screw you latex.
\begin{figure}[t]
  \centering
  \begin{subfigure}[b]{0.46\textwidth}
    \begin{overpic}[width=\textwidth]{Figs/dphi/chris2/qcd_mc/dphi_incl/v2/dphi_eq3j_ge0b_775}
      \put(63,43){\includegraphics[width=1.7cm]
      {Figs/dphi/chris2/dphi_acc_legend}}
    \end{overpic}
    \caption{$\nj = 3$, QCD MC simulation}
    \label{fig:dphi_acceptance_sim_3j}
  \end{subfigure}
  \begin{subfigure}[b]{0.46\textwidth}
    \begin{overpic}[width=\textwidth]{Figs/dphi/chris2/data/dphi_incl/v2/dphi_eq3j_ge0b_775}
      \put(63,43){\includegraphics[width=1.7cm]{Figs/dphi/chris2/dphi_acc_legend}}
    \end{overpic}
    \caption{$\nj = 3$, data}
    \label{fig:dphi_acceptance_data_3j}
  \end{subfigure}\\
  \begin{subfigure}[b]{0.46\textwidth}
    \begin{overpic}[width=\textwidth]{Figs/dphi/chris2/qcd_mc/dphi_incl/v2/dphi_eq4j_ge0b_775}
      \put(63,43){\includegraphics[width=1.7cm]
      {Figs/dphi/chris2/dphi_acc_legend}}
    \end{overpic}
    \caption{$\nj = 4$, QCD MC simulation}
    \label{fig:dphi_acceptance_sim_4j}
  \end{subfigure}
  \begin{subfigure}[b]{0.46\textwidth}
    \begin{overpic}[width=\textwidth]{Figs/dphi/chris2/data/dphi_incl/v2/dphi_eq4j_ge0b_775}
      \put(63,43){\includegraphics[width=1.7cm]{Figs/dphi/chris2/dphi_acc_legend}}
    \end{overpic}
    \caption{$\nj = 4$, data}
    \label{fig:dphi_acceptance_data_4j}
  \end{subfigure}\\
  \caption{The \alphat distribution for events with no \mindphistar
    requirement (red circles) and with the $\mindphistar > 0.3$
    requirement (blue circles) as determined from QCD multijet
    simulation (left column) or data (right column) and the exclusive
    $\nj =3$
    (top row) or $\nj = 4$ (bottom row). Note that negligible QCD contamination
    is seen in the $\nj = 2$ category and consequently is not shown here. The
    QCD control study region requirements have been applied, $\HT > 775$~\gev
    and $\alphat > 0.507$, with $\nb \geq 0$.}
    \label{fig:data_pred_dphistar_eff}
\end{figure}
% 
When considering simulation (figures~\ref{fig:dphi_acceptance_sim_3j} and
\ref{fig:dphi_acceptance_sim_4j}), the requirement of \mindphistar > 0.3
removes all QCD events. However care
must be taken when interpreting the same plots for data
(figures~\ref{fig:dphi_acceptance_data_3j} and
\ref{fig:dphi_acceptance_data_4j}). To extract the expected QCD MJ contribution
from data, observed event counts are corrected to subtract the expected
contribution from
EWK background processes, as estimated using the standard EWK background
prediction method. Given this subtraction, the remaining number of events is
expected to be compatible with zero within errors as opposed to exactly zero, as
is seen in the data distributions. This supports the conclusion found in
simulation that the heavy flavour QCD
events are removed by the application of the \mindphistar > 0.3 requirement.


% Events with jets containing real \met sources are likely to have
% $\mht \approx \met$ by definition, and would therefore not be protected against
% by the \mhtmet threshold
% described in Section\ref{sec:qcd_cleaning_below_thresh}. Consider the ratio
% of events passing and failing the \mhtmet requirement, as
% % 
% \begin{equation}
% \label{eq:rmhtmet}
% \rmhtmet = \frac{N(\mhtmet<1.25)}{N(\mhtmet>1.25)}.
% \end{equation}
% % 
% Another method to determine the QCD MJ contamination is to study this variable
% plotted as a function of \alphat after subtracting the
% predicted EWK background, leaving only QCD multijet events.
% In the absence of events with jets containing sources of real \met, a strong
% exponential decrease as a function of \alphat would be expected. However, given
% these events populate the
% \mhtmet < 1.25 region, a constant pedestal in \alphat is observed, as is shown
% in Figure~\ref{fig:rmhtmet_dphi_data_sim}.

%   \begin{figure}[p!]
%     \centering
%     \begin{subfigure}[b]{0.46\textwidth}
%       \includegraphics[width=\textwidth]{Figs/dphi/chris2/qcd_mc/dphi_lt0p3/v2/ratio_eq3j_ge0b_775}
%       \caption{Simulation, $\nj = 3$, \mindphistar < 0.3}
%       \label{fig:rdphi_sim_j3_lt0p3}
%     \end{subfigure}
%     \begin{subfigure}[b]{0.46\textwidth}
%       \includegraphics[width=\textwidth]{Figs/dphi/chris2/data/dphi_lt0p3/v2/ratio_eq3j_ge0b_775}
%       \caption{Data, $\nj = 3$, \mindphistar < 0.3}
%       \label{fig:rdphi_data_j3_lt0p3}
%     \end{subfigure} \\

%     \begin{subfigure}[b]{0.46\textwidth}
%       \includegraphics[width=\textwidth]{Figs/dphi/chris2/qcd_mc/dphi_gt0p3/v2/ratio_eq3j_ge0b_775}
%       \caption{Simulation, $\nj = 3$, \mindphistar > 0.3}
%       \label{fig:rdphi_sim_j3_gt0p3}
%     \end{subfigure}
%     \begin{subfigure}[b]{0.46\textwidth}
%       \includegraphics[width=\textwidth]{Figs/dphi/chris2/data/dphi_gt0p3/v2/ratio_eq3j_ge0b_775}
%       \caption{Data, $\nj = 3$, \mindphistar > 0.3}
%       \label{fig:rdphi_data_j3_gt0p3}
%     \end{subfigure} \\

%     \begin{subfigure}[b]{0.46\textwidth}
%       \includegraphics[width=\textwidth]{Figs/dphi/chris2/qcd_mc/dphi_lt0p3/v2/ratio_eq4j_ge0b_775}
%       \caption{Simulation, $\nj = 4$, \mindphistar < 0.3}
%       \label{fig:rdphi_sim_j4_lt0p3}
%     \end{subfigure}
%     \begin{subfigure}[b]{0.46\textwidth}
%       \includegraphics[width=\textwidth]{Figs/dphi/chris2/data/dphi_lt0p3/v2/ratio_eq4j_ge0b_775}
%       \caption{Data, $\nj = 4$, \mindphistar < 0.3}
%       \label{fig:rdphi_data_j4_lt0p3}
%     \end{subfigure} \\

%     \begin{subfigure}[b]{0.46\textwidth}
%       \includegraphics[width=\textwidth]{Figs/dphi/chris2/qcd_mc/dphi_gt0p3/v2/ratio_eq4j_ge0b_775}
%       \caption{Simulation, $\nj = 4$, \mindphistar > 0.3}
%       \label{fig:rdphi_sim_j4_gt0p3}
%     \end{subfigure}
%     \begin{subfigure}[b]{0.46\textwidth}
%       \includegraphics[width=\textwidth]{Figs/dphi/chris2/data/dphi_gt0p3/v2/ratio_eq4j_ge0b_775}
%       \caption{Data, $\nj = 4$, \mindphistar > 0.3}
%       \label{fig:rdphi_data_j4_gt0p3}
%     \end{subfigure}
%     \caption{Ratio of events passing and failing the \mhtmet < 1.25
%     requirement, both for simulation (left column) and data (right column).
%     Individual plots are shown for all combinations of $\nj = 3$ or $\nj = 4$,
%     and \mindphistar > 0.3 or \mindphistar < 0.3. Note that data plots
%     represent events counts after the EWK background prediction has been
%     subtracted.}
%     \label{fig:rmhtmet_dphi_data_sim}
%   \end{figure}

% When the QCD-enriched \mindphistar < 0.3 region is considered, for low \alphat
% values the ratio \rmhtmet is seen to fall exponentially. However at
% higher \alphat values, where the bulk of QCD events are rejected, the
% constant pedestal arises due to heavy-flavour QCD jets. Inverting
% the \mindphistar requirement reduces the pedestal to entirely negligible
% levels. Similarly to Figure~\ref{fig:data_pred_dphistar_eff}, data counts are
% corrected to remove the EWK background prediction and so gain a reasonably
% large uncertainty, however still agree statistically with zero.

\begin{figure}[t]
  \centering
  \begin{subfigure}[b]{0.46\textwidth}
    \begin{overpic}[width=\textwidth]{Figs/datapred/qcd_study_region/ge2j_ge0b_775_upwards/Prediction_MHT_all_775_upwards_QCD.pdf}
      \put(60,70){\includegraphics[width=1.1cm]{Figs/results/v0/ht_white_cmsprelim_cover.png}}
    \end{overpic}
    \caption{\mindphistar < 0.3}
    \label{fig:datapred_mht_lt0p3}
  \end{subfigure}
  \begin{subfigure}[b]{0.46\textwidth}
    % \includegraphics[width=\textwidth]
    % {Figs/datapred/qcd_study_region/ge2j_ge0b_775_upwards/gt0p3/Prediction_MHT_all_775_upwards_QCD.pdf}
    \begin{overpic}[width=\textwidth]{Figs/datapred/qcd_study_region/ge2j_ge0b_775_upwards/gt0p3/Prediction_MHT_all_775_upwards_QCD.pdf}
      \put(64,70){\includegraphics[width=0.91cm]{Figs/results/v0/ht_white_cmsprelim_cover.png}}
    \end{overpic}
    \caption{\mindphistar > 0.3}
    \label{fig:datapred_mht_gt0p3}
  \end{subfigure}
  \caption{Observations in data compared to the EWK background prediction
  as a function of the \mht variable.
  An inclusive selection is made in the QCD study region, \HT > 775~\gev,
  \alphat > 0.507, $\nj \geq 2$ and $\nb \geq 0$. Plots are shown for
  both the
  QCD-enriched \mindphistar < 0.3 and QCD-cleaned \mindphistar > 0.3 regions.
  Yields from QCD MC (cyan) are shown stacked on the total EWK background
  prediction, but not included in the ratio.}
  \label{fig:datapred_mht_before_after_dphi}
\end{figure}

Figure~\ref{fig:datapred_mht_before_after_dphi} shows data observations
compared to the EWK background prediction as a
function of \mht for the QCD study region, with both the
\mindphistar < 0.3 (Figure~\ref{fig:datapred_mht_lt0p3}) and
\mindphistar > 0.3 (Figure~\ref{fig:datapred_mht_gt0p3}) requirements applied.
Following the removal of the low \mindphistar events, observations agree well
with the EWK prediction. As a consequence of these studies, the analysis
additionally requires that all events have \mindphistar > 0.3 in the signal
region.


\clearpage
\begin{sidewaysfigure}
    \centering
    \includegraphics[width=0.95\textwidth]
    {Figs/eventDisplays/Had_QCD_MG_MC_HT375_skim_displays_singleEvent_2_noPF.pdf}
    \caption{An event display of a typical heavy flavour QCD event, with a jet
    decaying semi-leptonically with high-\Pt neutrinos and therefore
    considerable generator-level \met.}
    \label{fig:event_display_QCD}
\end{sidewaysfigure}

% For bins of $\HT>375$~\gev the leading two jets in the event are required to 
% have $\Pt>100$~\gev, with all additional jets having half the requirement of
% $\Pt>50$~\gev. In order to maintain a similar kinematic phase space throughout
% the many \HT bins, these jet \Pt requirements are scaled for the lower \HT bins 
% as shown in Table~\ref{tab:analysis_thresholds}.

% \emph{NOTE: have removed the `naive' predictions section}

% \section{Primitive predictions from Transfer Factors}
% \label{sec:background_predictions}
% Predictions are made using the transfer factor method alone for the EWK
% background processes. These predictions precede the full predictions made using
% the more sophisticated simultaneous fit method, as described later in
% chapter~\ref{ch:7}.


% \clearpage
% NOTE - All plots currently from:

% \begin{verbatim}
%         out_dict["28Jan_fullLatestReRun_dPhi_gt0p3_v0"] = {

%                 "path_name": "rootfiles/Root_Files_28Jan_fullLatestReRun_dPhi_gt0p3_v0",

%                 # All Runs
%                 "had_lumi": 18.493,
%                 "mu_lumi": 19.131,
%                 "ph_lumi": 19.12,

%                 # taken from parked final (change if necessary)
%                 "wj_corr": 0.94,
%                 "dy_corr": 0.94,
%                 "tt_corr": 1.17,

%         }
% \end{verbatim}
