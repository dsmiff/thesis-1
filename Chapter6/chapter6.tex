\chapter{Background Estimation}
\label{ch:6}

% **************************** Define Graphics Path **************************
\ifpdf
    \graphicspath{{Chapter6/Figs/Raster/}{Chapter6/Figs/PDF/}{Chapter6/Figs/}}
\else
    \graphicspath{{Chapter6/Figs/Vector/}{Chapter6/Figs/}}
\fi

Maybe some stuff here introducing the chapter...

%********************************** % First Section  *************************************
\section{Overview of TF method}  %Section - 1.1 
\label{sec:background_overview}

Contributions from Standard Model background processes are estimated using a data-driven 
prediction technique, using of dedicated control samples. The 
technique uses a transfer factor (TF), constructed from MC samples as the ratio 
of the MC yield in the signal region, $N_{MC}^{signal}\big(\HT, \nj, \nb\big)$,
and the MC yield of a given control region, \\$N_{Data}^{control}\big(\HT, \nj, \nb\big)$,
as a function of the analysis binning, \HT, \nj and \nb, as shown by
equation~\ref{eq:transfer_factor}.

\begin{equation}
TF = \frac{N_{MC}^{signal}\big(\HT, \nj, \nb\big)}{N_{MC}^{control}\big(\HT, \nj, \nb\big)}
\label{eq:transfer_factor}
\end{equation}

For a given \HT, \nj and \nb bin, the TF is used to extrapolate a yield in data from
the control region, $N_{Data}^{control}\big(\HT, \nj, \nb\big)$
to the signal region $N_{Data}^{signal}\big(\HT, \nj, \nb\big)$, as shown in
equation~\ref{eq:transfer_equation}.

\begin{equation}
N_{Data}^{signal}\big(\HT, \nj, \nb\big) = N_{Data}^{control}\big(\HT, \nj, \nb\big)
\times TF
\label{eq:transfer_equation}
\end{equation}

The control samples are statistically independent and each used for predicting 
specific background processes, the specific details of which are described in 
the following section, section~\ref{sec:background_control}.

To construct the MC yields, the following samples are considered: W + jets (\numw),
\ttbar + jets (\numtt), DY + jets (\numdy), $\gamma$ + jets(\numgam),
single top + jets (\numtop), WW + jets, WZ + jets and ZZ + jets (\numdibo), and
\zinv + jets (\numzinv).

The predictions made using this technique alone are considered as ``na\"{i}ve'' 
predictions and are only used in analysis development and background systematics
derivation. In order to determine final yields for use in interpretation and 
limit-setting, a fit is made across all signal and control regions, using the 
full likelihood model, as described later in chapter~REF. For this, the derived 
transfer factors and individual yields enter as terms in the likelihood, where 
all related systematics and potential correlations are accounted for in the fit. 

The denominator of each transfer factor is constructed using the sum of all MC
sample yields, for a given control region and category:

\begin{equation}
N_{MC}^{control}\big(\HT, \nj, \nb\big) = \numw + \numtt + \numdy + \numgam + 
\numtop + \numdibo + \numzinv
\label{eq:trans_fact_denom}
\end{equation}

However, the numerator is constructed according to the b-tag mutliplicity being 
considered. For $\nb \leq 1$, the \mj control region is used to predict 
predominantly \ttbar + jets and W + jets, however an estimate for all other 
residual backgrounds is produced. All MC samples are used with the exception of
\zinv:

\begin{equation}
N_{MC}^{signal}\big(\HT, \nj, \nb \leq 1\big) = \numw + \numtt + \numdy + \numgam + 
\numtop + \numdibo
\label{eq:trans_fact_num_le1b_noz}
\end{equation}

Whereas the \zinv + jets component of the background is predicted using the \mmj
and \gj control samples, using only the \zinv MC yields:

\begin{equation}
N_{MC}^{signal}\big(\HT, \nj, \nb \leq 1 \big) = \numzinv
\label{eq:trans_fact_num_le1b_z}
\end{equation}

Again, it should be noted here that although two seperate control samples are 
used to estimate this background, but result of each is considered by the global fit
used to produce the final background prediction.

For $\nb \geq 2$, the \mj sample is used to produce a prediction for all 
SM processes, including \zinv, therefore the numerator of the TF is defined as:

\begin{equation}
N_{MC}^{signal}\big(\HT, \nj, \nb \geq 2 \big) = \numw + \numtt + \numdy + \numgam + 
\numtop + \numdibo + \numzinv
\label{eq:trans_fact_num_geq2b}
\end{equation}

The \mmj and \gj control samples are not used beyond the $\nb \leq 1$ categories
due to the statistical limitations of such samples at high b-tag multiplicities.
A full summary of the control regions used for predictions per analysis category
is shown in table~\ref{tab:control_prediction_summary}.

\begin{table}[h!]
  \caption{Summary of control samples used to predict the SM
    background for each event category. REWORD}
  \label{tab:control_prediction_summary}
  \centering
  \begin{tabular}{ lll }
    \hline
    \hline
    \nj     & \nb     & Control samples \\ [1.0ex]
    \hline
    2--3    & 0       & \mj, \mmj, \gj  \\
    2--3    & 1       & \mj, \mmj, \gj  \\
    2--3    & 2       & \mj             \\
    $\geq$4 & 0       & \mj, \mmj, \gj  \\
    $\geq$4 & 1       & \mj, \mmj, \gj  \\
    $\geq$4 & 2       & \mj             \\
    $\geq$4 & 3       & \mj             \\
    $\geq$4 & $\geq4$ & \mj             \\
    \hline
    \hline
  \end{tabular}
\end{table}

Aimed to remove dependency on MC as much as possible - (hopefully) everything cancels in 
the ratio!
- MC mismodelling
    - kinematics (affecting acceptances)
    - instrumental affects (affecting object reconstruction efficiencies)


%********************************** % Second Section  *************************************
\section{Control samples}  %Section - 1.2
\label{sec:background_control}

control sample selection is very close to signal region. only differences are `
tags' used to get mu, mumu, pho, which are then ignored for kin variables. also
some mass windows  and minor kinematic cuts used to enrich samples in certain
processes. they are statistically independent, no correlation, due to 
requirement of the tagged particles, therefore surpressing signal contamination.

\subsection{mu + jets}
Overview
Targetted backgrounds
\subsubsection{Triggers}
\subsubsection{Selection Criteria}

\subsection{mumu + jets}
Overview
Targetted backgrounds
\subsubsection{Triggers}
\subsubsection{Selection Criteria}

\subsection{photon + jets}
Overview
Targetted backgrounds
\subsubsection{Triggers}
\subsubsection{Selection Criteria}

\subsection{jets}
Overview
Targetted backgrounds
\subsubsection{Triggers}
\subsubsection{Selection Criteria}

\subsection{HT sideband normalisation}

%********************************** % Third Section  *************************************
\section{Estimating multijet backgrounds}  %Section - 1.3
\label{sec:background_qcd}
More complex method
\subsection{MHT/MET sideband}


%********************************** % Fourth Section  *************************************
\section{Naive predictions from Transfer Factors}  %Section - 1.4
\label{sec:background_predictions}
Naive predictions only purely from transfer factors and yields.


%********************************** % Fifth Section  *************************************
\section{Uncertainties on SM background predictions}  %Section - 1.5
\label{sec:background_systematics}

Introduction to closure test method
Designed to probe the various areas of uncertainty in the analysis
`statistically powerfull'

\subsection{Closure tests}
Show closure test summary plots as well as some specific tests on their own

may also include SITV closure tests

\subsection{Background uncertainty summary}
split into HT regions

summary of determined systematics
