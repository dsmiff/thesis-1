\chapter{Background Estimation}
\label{ch:6}

% **************************** Define Graphics Path **************************
\ifpdf
    \graphicspath{{Chapter6/Figs/Raster/}{Chapter6/Figs/PDF/}{Chapter6/Figs/}}
\else
    \graphicspath{{Chapter6/Figs/Vector/}{Chapter6/Figs/}}
\fi

Maybe some stuff here introducing the chapter...

%********************************** % First Section  *************************************
\section{Overview of TF method}  %Section - 1.1 
\label{sec:background_overview}

Contributions from Standard Model background processes are estimated using a data-driven 
prediction technique which makes use of control samples in the analysis. The 
technique uses a transfer factor (TF), constructed from MC samples as the ratio 
of the MC yield in the signal region, $N_{MC}^{signal}\big(\HT, \nj, \nb\big)$,
and the MC yield of a given control region, $N_{Data}^{control}\big(\HT, \nj, \nb\big)$,
as a function of the analysis binning, \HT, \nj and \nb, as shown by
equation~\ref{eq:transfer}.

\begin{equation}
TF = \frac{N_{MC}^{signal}\big(\HT, \nj, \nb\big)}{N_{MC}^{control}\big(\HT, \nj, \nb\big)}
\label{eq:transfer_factor}
\end{equation}

For a given \HT, \nj and \nb bin, the TF is used to extrapolate yields in data from
the control region, $N_{Data}^{control}\big(\HT, \nj, \nb\big)$
to the signal region $N_{Data}^{signal}\big(\HT, \nj, \nb\big)$, as shown in
equation~\ref{eq:transfer_equation}.

\begin{equation}
N_{Data}^{signal}\big(\HT, \nj, \nb\big) = N_{Data}^{control}\big(\HT, \nj, \nb\big)
\times TF
\label{eq:transfer_equation}
\end{equation}


Introduce the control samples and which is used for what background. ``Will 
describe in more detail in the following section''

outline all of the MC yields used (i.e. which sample where)

explain that the predictions calculated here are not used for results and 
interpretation, but instead the TF's are constructed as part of the likelihood 
function which performs a simultaneous fit of all control regions to produce 
final yields.

for 0, 1 b, use mu for everything, except Zinv where we use dimu and gamma.
we use dimu and gamma together, but later everything is fit.

for >= 2b, we use mu for everything, including zinv. in this region, dimu and 
gamma will have negligible events because of the high b-tag multiplicity.

summary table of what's predicted from where!

Aimed to remove dependency on MC as much as possible - (hopefully) everything cancels in 
the ratio!
- MC mismodelling
    - kinematics (affecting acceptances)
    - instrumental affects (affecting object reconstruction efficiencies)


%********************************** % Second Section  *************************************
\section{Control samples}  %Section - 1.2
\label{sec:background_control}

control sample selection is very close to signal region. only differences are `
tags' used to get mu, mumu, pho, which are then ignored for kin variables. also
some mass windows  and minor kinematic cuts used to enrich samples in certain
processes. they are statistically independent, no correlation, due to 
requirement of the tagged particles, therefore surpressing signal contamination.

\subsection{mu+jets}
Overview
Targetted backgrounds
\subsubsection{Triggers}
\subsubsection{Selection Criteria}

\subsection{mumu+jets}
Overview
Targetted backgrounds
\subsubsection{Triggers}
\subsubsection{Selection Criteria}

\subsection{photon+jets}
Overview
Targetted backgrounds
\subsubsection{Triggers}
\subsubsection{Selection Criteria}

\subsection{jets}
Overview
Targetted backgrounds
\subsubsection{Triggers}
\subsubsection{Selection Criteria}

\subsection{HT sideband normalisation}

%********************************** % Third Section  *************************************
\section{Estimating multijet backgrounds}  %Section - 1.3
\label{sec:background_qcd}
More complex method
\subsection{MHT/MET sideband}


%********************************** % Fourth Section  *************************************
\section{Naive predictions from Transfer Factors}  %Section - 1.4
\label{sec:background_predictions}
Naive predictions only purely from transfer factors and yields.


%********************************** % Fifth Section  *************************************
\section{Uncertainties on SM background predictions}  %Section - 1.5
\label{sec:background_systematics}

Introduction to closure test method
Designed to probe the various areas of uncertainty in the analysis
`statistically powerfull'

\subsection{Closure tests}
Show closure test summary plots as well as some specific tests on their own

may also include SITV closure tests

\subsection{Background uncertainty summary}
split into HT regions

summary of determined systematics
