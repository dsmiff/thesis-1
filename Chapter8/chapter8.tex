\chapter{Analysis Crosschecks}

% **************************** Define Graphics Path **************************
\ifpdf
    \graphicspath{{Chapter8/Figs/Raster/}{Chapter8/Figs/PDF/}{Chapter8/Figs/}}
\else
    \graphicspath{{Chapter8/Figs/Vector/}{Chapter8/Figs/}}
\fi

%********************************** % First Section  *************************************
\section{Overview}  %Section - 1.1 
\label{sec:crosschecks_overview}
Explain the fact that there is an excess. Numerous features and areas of the 
analysis were probed

\section{Remove SITV}
\label{sec:crosschecks_nositv}
Is the excess still there if we remove the SITV? Yes

\section{What other cross checks have we performed?}
\label{sec:crosschecks_other}
Going lower in alphaT...see more excess!

\section{bDPhiStar}
\label{sec:crosschecks_dphi}
Can use this variable to check for jet mis-measurements due to jet-
mismeasurement. We noticed the excess manifests itself at low-values of dphi and
so numerous investigations have been made.

\subsection{Apply >0.3 requirement}

Might be a nice idea, but it kills a lot of signal too...

\subsection{Signal dphi distro}
Show the T2cc plots - the deadECAL's effect on these (removes peak around zero)

in this respect, signal can look like QCD to this variable!

\subsection{Soft muon}
Requirement a soft muon, thereby nearly entirely removing QCD from the 
selection

\subsection{Object veto checks}
Can probe how well the various vetoes are working by allowing a soft [relevant 
object] to pass the selection (pT less than the object selection requirement, 
but greater than the ID requirement)


we expect an under prediction to explain the excess we see. 
take the example of the soft SIT. we want the prediction we see within the 
analysis to be too low. this will only happen if N(signal, MC) is too low, in 
the transfer factor eqn.

\begin{equation}
N_{pred}^{sig} = N_{MC}^{sig} \times \frac{N_{Data}^{cont}}{N_{MC}^{cont}}
\end{equation}

The control sample doesn't change in data or MC, so only a change in the signal 
MC can affect the prediction. We want this to be lower than we expect - in other
words we want there to be more soft SITs (which will be vetoed by the SITV) in 
the MC, thereby lowering the MC signal yield that enters the above equation. If 
we actually require soft SITs, we want to see there are more in MC than data, in
other words the N MC sig will be high, and therefore we want to see an over 
prediction. However, in the latest numbers we see an underprediction...oh shit 
yo!


\subsubsection{Muon}
quite impressive closure of this test! No excess seen.

\subsubsection{Electron}
No excess seen.

\subsubsection{Photon}
No excess seen.

\subsubsection{SIT}
Excess seen.

Are currently implementing a dPhi>0.3 cut to see whether it remains.

Potential explanation is the isolation used with the analysis objects. Can 
replace this with the tracker isolation found in the SIT (to do!) 

\section{Any more we do...}
