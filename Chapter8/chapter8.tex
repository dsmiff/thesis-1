\chapter{Analysis Crosschecks}

% **************************** Define Graphics Path **************************
\ifpdf
    \graphicspath{{Chapter8/Figs/Raster/}{Chapter8/Figs/PDF/}{Chapter8/Figs/}}
\else
    \graphicspath{{Chapter8/Figs/Vector/}{Chapter8/Figs/}}
\fi

%********************************** % First Section  *************************************
\section{Overview}  %Section - 1.1 
\label{sec:crosschecks_overview}
Explain the fact that there is an excess. Numerous features and areas of the 
analysis were probed

\section{Remove SITV}
\label{sec:crosschecks_nositv}
Is the excess still there if we remove the SITV? Yes

\section{What other cross checks have we performed?}
\label{sec:crosschecks_other}
Going lower in alphaT...see more excess!

\section{bDPhiStar}
\label{sec:crosschecks_dphi}
Can use this variable to check for jet mis-measurements due to jet-
mismeasurement. We noticed the excess manifests itself at low-values of dphi and
so numerous investigations have been made.

\subsection{Apply >0.3 requirement}

Might be a nice idea, but it kills a lot of signal too...

\subsection{Signal dphi distro}
Show the T2cc plots - the deadECAL's effect on these (removes peak around zero)

in this respect, signal can look like QCD to this variable!

\subsection{Soft muon}
Requirement a soft muon, thereby nearly entirely removing QCD from the 
selection

\subsection{Object veto checks}
Can probe how well the various vetoes are working by allowing a soft [relevant 
object] to pass the selection (pT less than the object selection requirement, 
but greater than the ID requirement)

\subsubsection{Muon}
quite impressive closure of this test! No excess seen.

\subsubsection{Electron}
No excess seen.

\subsubsection{Photon}
No excess seen.

\subsubsection{SIT}
Excess seen.

Are currently implementing a dPhi>0.3 cut to see whether it remains.

Potential explanation is the isolation used with the analysis objects. Can 
replace this with the tracker isolation found in the SIT (to do!) 

\section{Any more we do...}
