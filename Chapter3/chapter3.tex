\chapter{The LHC and the CMS Detector}

% **************************** Define Graphics Path **************************
\ifpdf
    \graphicspath{{Chapter3/Figs/Raster/}{Chapter3/Figs/PDF/}{Chapter3/Figs/}}
\else
    \graphicspath{{Chapter3/Figs/Vector/}{Chapter3/Figs/}}
\fi


%********************************** % First Section  *************************************
\section{The Large Hadron Collider}  %Section - 1.1 
\label{sec:detector_lhc}

The Large Hadron Collider (LHC) is a 27km circumference proton-proton 
synchrotron accelerator with a maximum design centre of mass energy,
$\sqrt{s} = 14$ \tev. It is the highest energy component of the CERN accelerator
complex (figure~REF). The LHC accelerates counter-rotating beams of protons 
using 400 MHz radio frequency
(RF) cavities, focusses the beam using multiple quadrapole and higher order magnets,
and maintains the beams trajectory using super-conducting, niobium-titanium 
dipole magnets, capable of producing an 8.4 Tesla magnetic field. This high 
magnetic field is achieved by cooling the magnets to 1.9K and applying a 11.85 
kA electric current.

The beams, consisting of numbers of bunches of protons, are brought into collision
at four points around the LHC ring within specialised detector systems. During the
first run of the LHC, `Run I', bunches were spaced in 50ns intervals, leading to
a bunch crossing rate of 20 MHz. Simultaneous interactions happening at the time
of each bunch crossing are possible, and referred to as in-time pile up (PU), 
causing experimental challenges for detector readout and event reconstruction. 

The four detectors of the LHC consist of \texttt{ALICE}, \texttt{ATLAS},
\texttt{CMS} and \texttt{LHCb}. Of these, both \texttt{ATLAS} and \texttt{CMS} 
are considered as general purpose detectors, optimised for the investigation of
high-\Pt phenomena, making them ideal detectors for new physics.

The work in this thesis uses the \texttt{CMS} detector, which is discussed in 
the following section.


%********************************** % Second Section  *************************************
\section{Compact Muon Solenoid}  %Section - 1.2 
\label{sec:detector_overview}

Overview of CMS. What it is? What are it's general goals? Some history?


%********************************** % Third Section  *************************************
\section{Detector Subsystems}  %Section - 1.3
\label{sec:detector_subsystems}

\subsection{Pixels and tracker}

\subsection{Electromagenetic Calorimeter}

\subsection{Hadronic Calorimeter}

\subsection{Muon Systems}

\subsection{Solenoid Magnet}


%********************************** % Fourth Section  *************************************
\section{Trigger and Data Acquisition}  %Section - 1.4
\label{sec:detector_daq}

\subsection{Trigger system}

\subsubsection{Level-1 Hardware Trigger}

\subsubsection{High-level Trigger}

\subsection{DAQ system}

