%*******************************************************************************
%*********************************** First Chapter *****************************
%*******************************************************************************

\chapter{Introduction}  %Title of the First Chapter

\ifpdf
    \graphicspath{{Chapter1/Figs/Raster/}{Chapter1/Figs/PDF/}{Chapter1/Figs/}}
\else
    \graphicspath{{Chapter1/Figs/Vector/}{Chapter1/Figs/}}
\fi


%********************************** % First Section  *************************************
%Section - 1.1 
\label{sec:introduction_intro}

\emph{Introduce the goal of science. The scientific method.}

Our instinctive method of understanding is reductive - the cliched image of a
young inventor attempting to make sense of the things around them, is
that of endless toasters, clocks radios, or even gearboxes broken down into
their smallest constituent parts to better understand what they're made of and,
ultimately, what's happening inside of them. This simple idea translates almost
too perfectly to explain what Particle Physicists have been attempting to do
with not simple everyday household objects, but the Universe.

Since the beginnings of the field in YEAR, physicists have questioned
what is
around us, and what it's made of. From Fermi developing his ground-breaking
model of the atom, to Rutherford discovering it's structure, the basic building
blocks of the universe have been searched for. For Particle Physicits, this is
the ultimate goal - a complete understanding of the nature of the Universe on
the most
fundamental level. If we can describe to smallest possible constituents of
matter and how these constituents interact, then ultimately we can understand
the Universe from the ground up.

Currently, our best attempt comes from the intersection of two very different
fields: Quantum Mechanics, the study of microscopic objects, and Relativity, the
study of very fast \emph{(better wording!)} objects. The combination of these
two different \emph{thesaurus} worlds gave birth to the field of Relativistic
Quantum Mechanics, and ultimately, the Standard Model.

First proposed in the 1960s, the Standard Model attempts to describe the
properties of the fundamental consituents of matter, and outline their
interactions through three of the fundamental forces of nature. Through it's
accurate predictions of the existence of previously undiscovered particles and
precision experimentally measured values (can do better than that...!), the
Standard Model has sincerely (right word?) proven itself to be a triumph of
modern physics.

Depsite it's continued accuracy in the face of substantial scrutiny, however,
the Standard Model fails to explain large facts of the Universe (better than
that), and as such is widely considered as a piece of the puzzle (too
simplistic?) of some greater theory. 

\emph{would be nice to reference the constant back and forth between theorists
and experimentatlistlistilstsss.}
particle physics has developed through a cat and mouse game of theorists and
experimentalists trying to keep up with each others progress

\section{SUSY}
one theory we're looking to find evidence of is susy. can fill many of the holes
of the standard model.
of the many ways to search for susy, hadronic channels prove to be an excellent
arena for early discovery at hadronic machines like the LHC.

intersection of cosmology (huge shit) and us (tiny shit) is becoming more and
more apparent.

while perhaps the theories and techniques discussed and described in this thesis
may become outdated, the spirit of discovery and the quest for a deeper
understanding of the underlying nature of the universe must never change.

we now find ourselves in a historically familiar position in that, thanks t
o the LHC, we're standing
at the edge of the energy frontier, peering into the unknown.
