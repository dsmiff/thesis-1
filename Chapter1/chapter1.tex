%*******************************************************************************
%*********************************** First Chapter *****************************
%*******************************************************************************

\chapter{Introduction}  %Title of the First Chapter

\ifpdf
    \graphicspath{{Chapter1/Figs/Raster/}{Chapter1/Figs/PDF/}{Chapter1/Figs/}}
\else
    \graphicspath{{Chapter1/Figs/Vector/}{Chapter1/Figs/}}
\fi


%********************************** % First Section  *************************************
%Section - 1.1 
\label{sec:introduction_intro}

Our instinctive method of understanding is reductive - the idea of an
inventor attempting to make sense of the things around them, is
that of toasters, clocks radios, or even gearboxes broken down into
their smallest constituent parts to better understand what they're made of and,
ultimately, what's happening inside of them. This simple idea translates
perfectly to explain the work Particle Physicists have been undertaking with
not simple everyday household objects, but the Universe.

Since the beginnings of the field, physicists have questioned
what is
around us, and what it's made of. From the early research of scientists such as
Ernest Rutherford and Niels Bohr, the basic building
blocks of the universe have been searched for. For Particle Physicists, this is
the ultimate goal - a complete understanding of the nature of the Universe
at it's most
fundamental level. If a description can be written of the smallest
building blocks of
nature and how they interact, then ultimately we can understand
the Universe from the bottom up.

Currently, our best attempt of such a description comes from the intersection of
two very different
fields: Quantum Mechanics, the study of microscopic objects, and Relativity, the
study of very fast objects. The combination of these
two contrasting worlds gave birth to the field of Relativistic
Quantum Mechanics, and ultimately, the Standard Model.

First proposed in the 1960s, the Standard Model describes the
properties of the fundamental constituents of matter, and outline their
interactions through three of the four fundamental forces of nature, as
described in chapter~\ref{ch:theory}. Through
it's accurate description and predictions of not only particles such as
the electrons, quarks and gluons of which atoms are composed,
as well as fundamental constants of nature such as the fine structure constant
$\alpha$, the
Standard Model has exceptionally proven itself to be a triumph of
modern physics.

Despite its continued accuracy in the face of substantial scrutiny, however,
the Standard Model fails to account for significant observations, for example
the absent description of gravity and the `other' 96\% of the
Universe - the so-called `Dark Sector'. Similarly the Standard Model fails to
unify the forces at some higher mass scale, or predict the mass of
the newly discovered Higgs boson. As such, given how successfully it describes
what it can, the SM is widely considered as a smaller component of some greater,
all-encompassing theory. This belief has led many to use this distinct theory as
a building block on which extensions can be introduced to create Beyond the
Standard Model (BSM) theories - the goal being to fill the gaps in knowledge
left by the SM.

Of the plethora of BSM theories, one particularly well motivated extension is
Supersymmetry (SUSY). The Supersymmetric extension to the Standard
Model introduces supersymmetryic particles (or `sparticles') differing from
their SM partners by half a unit of spin - described in greater detail in
chapter~\ref{ch:theory}. The addition of sparticles fixes
many of the Standard Model's problems, most notably that of the
divergent Higgs Mass, the Unification of the fundamental gauge couplings and,
crucially, provides a feasible candidate particle for the elusive Dark Matter
observed in abundance in the Universe.

Particle physics as a field has developed largely via a cat and mouse game,
played out between theorists and experimentalists. While theoretical physicists
have developed new theories and proposed new particles, experimentalists have
raced to build
greater and more advanced experiments with which to test them, and vice-versa.
Today, thanks to the incredibly hard work of thousands of physicists and
engineers, we are able to test such BSM theories with the Large
Hadron Collider (LHC) in Geneva, Switzerland.

The LHC, described in detail in chapter~\ref{ch:detector}, is a 27~km
circumference proton synchrotron, capable of colliding protons
at centre-of-mass energies far exceeding any previous experiment.
Using the LHC, it is possible to reproduce the conditions observed a short time
after the Big Bang, where the incredibly high energies present were able to
produce countless particles that wouldn't be visible in today's universe. If
such exotic particles are created in the proton-proton collisions performed at
the LHC, it is the goal of experiments such as the Compact Muon Solenoid (CMS)
to detect them.

The CMS detector, also described in chapter~\ref{ch:detector}, is comprised of
multiple sub-detector systems, each designed
specifically to make precise measurements of the characteristics of the
different particles created. The detector is built such that it provides almost
complete hermetic coverage to detect particles travelling in all directions
from the point of collision. Energy deposits throughout the detector subsystems
are used to form physics objects, as summarised in
chapter~\ref{ch:objects}, themselves allowing the events to be categorised by
the physics processes present.

A key technique
for searches for SUSY is the study of global momentum imbalances,
which allows for the presence of weakly interacting particles to be inferred.
If sparticles were to be produced at the LHC they would likely decay immediately
to many well-understood SM particles, and finally the weakly interacting
`Lightest Supersymmetric Particle' (LSP) - the theory's Dark Matter candidate.
Of the many different possible supersymmetric decays, hadronic channels offer
the largest branching fractions, although can present some of the most
difficult experimental challenges. Hadronic searches are of interest given their
relatively simple and quick ability to provide sensitivity to early discovery at
hadronic colliders such as the LHC.

The work described in this thesis consists of such a search for SUSY particles
decaying to quarks and missing-energy. The analysis makes use of the full
`Parked' data sample collected throughout 2012, where experimental
triggers collected events with lower thresholds than the standard data stream.
Lower energy thresholds greatly improve analysis sensitivity to soft physics
processes, such as those expected from SUSY scenarios with nearly degenerate
mass hierarchies. The data samples and corresponding MonteCarlo simulated
samples used in this analysis are detailed in chapter~\ref{ch:samples}.

The biggest obstacle to
searching for hadronic signatures of new physics, is dealing with the
overwhelming hadronic backgrounds from Quantum Chromodynamics (QCD). In this
analysis a powerful kinematic variable, \alphat, is used to differentiate
between such QCD and potential signal events. The analysis backgrounds,
selection and
the \alphat variable are described at length in chapter~\ref{ch:analysis}. The
remaining SM sources of background
are estimated using a data-driven transfer technique from dedicated control
samples, orthogonal to the search region, as outlined in
chapter~\ref{ch:background}.

Given the vast number of potential manifestations of SUSY, sparticles may
decay in many different ways if produced. Accordingly, chapter~\ref{ch:results}
gives the results of this analysis, categorised into events in dimensions of
three different event-level variables: the total sum of visible transverse
hadronic energy, the number of jets and the number of b-tagged jets.
Specific analysis categories are used to produce targeted
interpretations of different potential final states, thereby maintaining a broad
sensitivity to SUSY production, however it may present itself in nature. As
motivated by Dark Matter observations, the interpretations presented in this
thesis represent two different decay channels of the \sTop
sparticle in the so-called `compressed' region, where the mass of the \sTop is
near-degenerate with that of the LSP. Accordingly, upper limits on the mass of
the \sTop and the lightest neutralino are produced. These interpretations
and discussions of the related efficiencies and systematics are presented in
chapter~\ref{ch:interpretation}, before final conclusions are made in
chapter~\ref{ch:conclusion}.

% The analysis presented in this thesis is one of many such searches for
% Supersymmetry performed at the LHC, each with the goal of understanding better
% the nature of the largest scale objects of the Universe, by studying its
% smallest.