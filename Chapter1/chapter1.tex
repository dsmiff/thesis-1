%*******************************************************************************
%*********************************** First Chapter *****************************
%*******************************************************************************

\chapter{Introduction}  %Title of the First Chapter

\ifpdf
    \graphicspath{{Chapter1/Figs/Raster/}{Chapter1/Figs/PDF/}{Chapter1/Figs/}}
\else
    \graphicspath{{Chapter1/Figs/Vector/}{Chapter1/Figs/}}
\fi


%********************************** % First Section  *************************************
%Section - 1.1 
\label{sec:introduction_intro}

Our instinctive method of understanding is reductive - the typical image of a
young inventor attempting to make sense of the things around them, is
that of endless toasters, clocks radios, or even gearboxes broken down into
their smallest constituent parts to better understand what they're made of and,
ultimately, what's happening inside of them. This simple idea translates
perfectly to explain the work Particle Physicists have been undertaking with
not simple everyday household objects, but the Universe.

Since the beginnings of the field, physicists have questioned
what is
around us, and what it's made of. From Fermi developing his ground-breaking
model of the atom, to Rutherford discovering it's structure, the basic building
blocks of the universe have been searched for. For Particle Physicits, this is
the ultimate goal - a complete understanding of the nature of the Universe
at it's most
fundamental level. If a description can be written of the smallest possible
building blocks of
nature and how they interact, then ultimately we can understand
the Universe from the ground up.

Currently, our best attempt comes from the intersection of two very different
fields: Quantum Mechanics, the study of microscopic objects, and Relativity, the
study of very fast objects. The combination of these
two contrasting worlds gave birth to the field of Relativistic
Quantum Mechanics, and ultimately, the Standard Model.

First proposed in the 1960s, the Standard Model attempts to describe the
properties of the fundamental consituents of matter, and outline their
interactions through three of the four fundamental forces of nature. Through
it's accurate description and predictions of not only particles
but also fundamental constants of nature, the
Standard Model has exceptionally proven itself to be a triumph of
modern physics.

Depsite it's continued accuracy in the face of substantial scrutiny, however,
the Standard Model fails to significant observations, for example
the absent description of gravity and the `other' 96\% of particles in the
Universe. As such, the SM is widely considered as a smaller component of some
greater, all-encompassing theory.
\emph{talk more of the problems of the standard model?}
This belief has led many to use this distinct theory as a building
block, on which extensions can be introduced to create Beyond the Standard Model
(BSM) theories - the goal being to fix the gaps in knowledge left by the SM.

Of the plethora of BSM theories, one particularly well motivated extension is
Supersymmetry (SUSY). The so-called `Supersymmetric' extension to the Standard
Model, introducing supersymmetryic particles (or `sparticles'), fixes many of
the original theories problems, most notably that of the
Standard Models divergent Higgs Mass and, crucially, provides a strong
candidate particle for the elusive Dark Matter observed in abundance in the
Universe.

Particle physics as a field has developed largely via a cat and mouse game,
played out between theorists and experimentalists. While theoreticians have
developed new theories and new particles, experimentalists have raced to build
greater and more advanced experiments with which to test them, and vice-versa.
Today, thanks to the incredibly hard work of thousands of physicists, engineers
and other professionals, we are able to test such BSM theories with the Large
Hadron Collider (LHC) in Geneva, Switzerland.

Using the LHC, we can reproduce the conditions observed a short time
after the Big Bang, where the incredibly high energies present were capable of
producing countless particles that wouldn't be visible in todays universe. If
such exotic particles are created in the proton-proton collisions performed at
the LHC, it is the goal of experiments such as the Compact Muon Solenoid (CMS)
to detect them.

The CMS detector is comprised of multiple sub-detector systems, each designed
specifically to make precise measurements of the characteristics of the
different particles created. The detector is built such that it provides almost
complete hermetic coverage to detect particles travelling in all directions
from the point of collision. Through the study of global momentum imbalances,
the presence of weakly interacting particles can be inferred - a key technique
for searches for SUSY.

If sparticles were to be produced at the LHC they would immediately decay to
many well-understood SM particles, and finally the weakly interacting `Lightest
Supersymmetric Particle' (LSP), the theory's Dark Matter candidate.
\emph{motivate why jets+MET is the best channel - high branching fractions?}
\emph{early discovery - blunt, quick, simple}

The work described in this thesis consists of such a search for SUSY particles
decaying to a jets and missing-energy final state. The biggest obstacle to
searching for hadronic signatures of new physics, is dealing with the huge
(better word) hadronic backgrounds from Quantum Chromodynamics (QCD). In this
analysis a powerful kinematic variable, \alphat, is used to differentiate
between QCD and potential signal events. The remaining SM sources of background
are estimated using a data-driven transfer technique from dedicated control
samples, orthogonal to the search region.

Given the vast number of potential manifestations of SUSY, many different forms
could theoretically be realised in nature. Accordingly, this analysis
categorises events into dimensions of three different event-level variables.
Specific analysis categories can therefore be used to produce targetted
interpretaions of different potential final states, thereby maintaining a broad
sensitivity to SUSY production, however it presents itself. In this thesis,
interpretations are shown for two different decay channels of the \sTop
sparticle, and limits produced in terms of the \sTop mass.


\emph{intersection of cosmology (huge stuff) and us (tiny stuff) is becoming more
and
more apparent.}

% while perhaps the theories and techniques discussed and described in this thesis
% may become outdated, the spirit of discovery and the quest for a deeper
% understanding of the underlying nature of the universe must never change.
\emph{little more here...}
We find ourselves in a historically familiar position in that, thanks to the
LHC, we are at the edge of the energy frontier, peering into the unknown.
Whatever we see, it's going to be exciting.
