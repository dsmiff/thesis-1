%*******************************************************************************
%****************************** Second Chapter *********************************
%*******************************************************************************

\chapter{Theory}

\ifpdf
    \graphicspath{{Chapter2/Figs/Raster/}{Chapter2/Figs/PDF/}{Chapter2/Figs/}}
\else
    \graphicspath{{Chapter2/Figs/Vector/}{Chapter2/Figs/}}
\fi


%********************************** % First Section  *************************************
\section{Standard Model}  %Section - 1.1 
\label{sec:theory_standard_model_intro}

Overview of Particle Physics and the Standard Model

\subsection{Fundamental Forces of Nature}

\subsection{Particle Content}
Go through the normal table of particles, explaining each section and how the relate 
to historic experimental observations.

\subsection{Local Gauge Invariance}
\subsubsection{QED and U(1)}
\subsubsection{The Electroweak Sector}
\subsubsection{QCD and SU(N)}

\subsection{Electroweak Symmetry Breaking and the Higgs}

\subsection{Failings of the Standard Model}
Overview
\subsubsection{Fine tuning/heirarchy problem}
\subsubsection{Dark Matter/Energy}


%********************************** % Second Section  *************************************
\section{Supersymmetry}  %Section - 1.2
\label{sec:theory_supersymmetry}
Overview

\subsection{The Minimally Supersymmetric Standard Model}
R-parity here?
`Complete models'

\subsection{Simplified Model Spectra}
Overview
\subsubsection{vs `Complete' models}
\subsubsection{Stop quark production and decay}
\subsubsection{Gluino production and decay}
\subsubsection{Current Limits}

\subsection{Mass-degeneracy in SUSY}
Overview
\subsubsection{Relevant Models}
Make link to the above section about stop decay channels
Why only relevant here?
\subsubsection{Phenomenology}
briefly cover this with some poinient plots
\subsubsection{Current Limits}

