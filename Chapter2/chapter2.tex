%*******************************************************************************
%****************************** Second Chapter *********************************
%*******************************************************************************

\chapter{Supersymmetry}

\ifpdf
    \graphicspath{{Chapter2/Figs/Raster/}{Chapter2/Figs/PDF/}{Chapter2/Figs/}}
\else
    \graphicspath{{Chapter2/Figs/Vector/}{Chapter2/Figs/}}
\fi


%********************************** % First Section  *************************************
\section{Introduction}  %Section - 1.1 
\label{sec:supersymmetry_intro}

Brief intro to what will be covered in this chapter

\section{Current understanding}
\label{sec:supersymmetry_current}
\subsection{Fundamental Forces of Nature}

\subsection{Particle Content of the Standard Model}
Go through the normal table of particles, explaining each section and how the relate 
to historic experimental observations.

\subsection{Failings of the Standard Model}
Overview

Main focus of section. Need to motivate beyond the standard model theories

\subsubsection{Fine tuning/heirarchy problem}
\subsubsection{Dark Matter/Energy}


%********************************** % Second Section  *************************************
\section{Overview of Supersymmetry}  %Section - 1.2
\label{sec:supersymmetry_overview}
Overview. A basic introduction.

Super-multiplets.

a broken symmetry

R-parity

solution to the heirarchy problem - loop corrections etc

\subsection{The Minimally Supersymmetric Standard Model}
talk about `Complete models'

mention the free parameters of the MSSM (hint at how they're eventually 
constrained within other models)

briefly mention the various symmetry breaking models - relation to visible/hidden sector 
etc.


\subsection{Simplified Model Spectra}
Overview. Signature vs model interpretations.

allow for theorists to easily interpret within their favoured complete model

allow easy comparison between various experimental searches.

\subsubsection{vs `Complete' models}
% \subsubsection{Gluino production and decay - only if needed for T1ttcc}

\subsection{Stop quark production and decay}
go through the various decay channels of the stop in the mStop vs mLSP plane
- which are relevant where

\subsection{Mass-degeneracy in SUSY}
Overview of the small deltaM region. Different area of phase space than many searches due to soft decay 
products.

currently largely inacessible

stop mass degeneracy line/gap

\subsubsection{Relevant Models}
Make link to the above section about stop decay channels

Why only relevant here?
- dark matter constraints - DM co-annihilation etc
- off-shell W

relic density is fulfilled in this region. omegah2

\subsubsection{Phenomenology}
briefly cover this with some poinient plots from early T2cc studies

MAY HAVE THIS HERE OR LATER
\subsubsection{Current Limits}

\subsection{Targetted models}
Full list with some minor comments and feynman diagrams

Currently: T2cc, T2degen, (maybe T1ttcc)