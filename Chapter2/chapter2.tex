%*******************************************************************************
%****************************** Second Chapter *********************************
%*******************************************************************************

\chapter{Theory}
\label{ch:2}

\ifpdf
    \graphicspath{{Chapter2/Figs/Raster/}{Chapter2/Figs/PDF/}{Chapter2/Figs/}}
\else
    \graphicspath{{Chapter2/Figs/Vector/}{Chapter2/Figs/}}
\fi


%********************************** % First Section  *************************************
\section{Introduction}  %Section - 1.1 
\label{sec:theory_intro}

The Standard Model (SM), proposed in 1970s, has long been the most prominent and 
successful description of fundamental particles and their interactions at the
energies currently experimentally accessible. It's predictive power has been proven with the 
theoretical description of particles such as the charm and top quarks, and the W and
Z bosons, prior to their experimental observation. Furthermore, precision 
electroweak measurement \emph{SUCH AS...} have shown impressive levels of 
agreement with SM predictions \emph{REF}.

However, despite it's great success, it is known to only be valid for low 
energies, and is missing significant details of both experimentally 
observed and theoretically predicted physical phenomena. As such, extensions to
the SM have been extensively studied, assuming the SM to be a low-energy 
regime within a greater theory. One such strongly theoretically motivated extension 
to the SM is Supersymmetry (SUSY).

This chapter outlines the SM model, including it's particle content and basic 
mechanisms, before going on to describe it's shortcomings - the theoretical 
motivation for theories Beyond the Standard Model (BSM). SUSY will then be 
introduced, before moving on to it's simplest forms, and finally a discussion of
the specific models and frameworks used for interpretation within this analysis.

\section{The Standard Model}
\label{sec:theory_current}
The Standard Model is a Quantum Field Theory describing the fundamental 
matter particles and their interactions via the fundamental strong, weak and 
electromagnetic forces. It was collaboratively developed over the 1960's, with
it's current form being finalised in the mid-1970s.

talk about development
based on fundamental symmetries of space time
great theory
mention last missing piece of the puzzle - scalar particle responsible to EWSB

\subsection{Fundamental Forces of Nature}
self explanatory

\subsection{Particle Content of the Standard Model}
Go through the normal table of particles, explaining each section and how the relate 
to historic experimental observations.

\subsection{Failings of the Standard Model}
Overview

Main focus of section. Need to motivate beyond the standard model theories

Here mention gravity? No dedicated subsubsection?

\subsubsection{The Hierarchy Problem}
\subsubsection{Dark Matter/Energy}


%********************************** % Second Section  *************************************
\section{Overview of Supersymmetry}  %Section - 1.2
\label{sec:theory_overview}
Overview. A basic introduction.

Super-multiplets.

a broken symmetry

R-parity

solution to the heirarchy problem - loop corrections etc

\subsection{The Minimally Supersymmetric Standard Model}
talk about `Complete models'

mention the free parameters of the MSSM (hint at how they're eventually 
constrained within other models)

briefly mention the various symmetry breaking models - relation to visible/hidden sector 
etc.


\subsection{Simplified Model Spectra}
Overview. Signature vs model interpretations.

allow for theorists to easily interpret within their favoured complete model

allow easy comparison between various experimental searches.

\subsubsection{vs `Complete' models}
% \subsubsection{Gluino production and decay - only if needed for T1ttcc}

\subsection{Stop quark production and decay}
go through the various decay channels of the stop in the mStop vs mLSP plane
- which are relevant where

\subsection{Mass-degeneracy in SUSY}
Overview of the small deltaM region. Different area of phase space than many searches due to soft decay 
products.

currently largely inacessible

stop mass degeneracy line/gap

\subsubsection{Relevant Models}
Make link to the above section about stop decay channels

Why only relevant here?
- dark matter constraints - DM co-annihilation etc
- off-shell W

relic density is fulfilled in this region. omegah2

\subsubsection{Phenomenology}
briefly cover this with some poinient plots from early T2cc studies

MAY HAVE THIS HERE OR LATER
\subsubsection{Current Limits}

\subsection{Targetted models}
Full list with some minor comments and feynman diagrams

Currently: T2cc, T2degen, (maybe T1ttcc)