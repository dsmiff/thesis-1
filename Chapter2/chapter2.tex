%*******************************************************************************
%****************************** Second Chapter *********************************
%*******************************************************************************

\chapter{Theory}
\label{ch:2}

\ifpdf
    \graphicspath{{Chapter2/Figs/Raster/}{Chapter2/Figs/PDF/}{Chapter2/Figs/}}
\else
    \graphicspath{{Chapter2/Figs/Vector/}{Chapter2/Figs/}}
\fi


%********************************** % First Section  *************************************
\section{Introduction}  %Section - 1.1 
\label{sec:theory_intro}

The Standard Model (SM), proposed in 1970s, has long been the most prominent and 
successful description of fundamental particles and their interactions at the
energies currently experimentally accessible. It's predictive power has been proven with the 
theoretical description of particles such as the charm and top quarks, and the W and
Z bosons, prior to their experimental observation. Furthermore, precision 
electroweak measurement \emph{SUCH AS...} have shown impressive levels of 
agreement with SM predictions \emph{REF}.

However, despite it's great success, it is known to only be valid for low 
energies, and is missing significant details of both experimentally 
observed and theoretically predicted physical phenomena. As such, extensions to
the SM have been extensively studied, assuming the SM to be a low-energy 
regime within a greater theory. One such strongly theoretically motivated extension 
to the SM is Supersymmetry (SUSY).

This chapter outlines the SM model, including it's particle content and basic 
mechanisms, before going on to describe it's shortcomings - the theoretical 
motivation for theories Beyond the Standard Model (BSM). SUSY will then be 
introduced, before moving on to it's simplest forms, and finally a discussion of
the specific models and frameworks used for interpretation within this analysis.

\section{The Standard Model}
\label{sec:theory_current}

The Standard Model is a renormalisable Quantum Field Theory describing
fundamental matter particles and their interactions via the fundamental strong,
weak and 
electromagnetic forces. It was collaboratively developed over the 1960's, with
it's current form being finalised in the mid-1970s.

Imposing local gauge invariance of the SM Lagrangian, $\mathcal{L}_{SM}$, under
transformations in universal space-time symmetry groups,
interactions can be described through conserved quantities. The $SU(3)$ group is
describes strong force interactions via colour charge, $SU(2)$ and $U(1)$ to
describe electroweak interactions via weak isospin and hypercharge respectively.

The requirement that $\mathcal{L}_{SM}$ be invariant under local gauge
transformations dictates the structure, and crucially the particle content of
the SM.

Matter particles are described by spin 1/2 fermions, divided into the six quarks
and six leptons. The quarks, are arrange into three generations: the up and
down, the charm and strange, and the top and bottom, each couplet carrying
individual electric charges of +2/3 and -1/2 respectively. The quarks also carry
a colour charge, where combinations of quarks forming colour-neutral composite
hadron particles are seen in nature.

% matter particles are described by spin 1/2 fermions, divided into the six quarks and six 
% leptons. the quarks are arranged in three generations, the up and down, the charm
% and strange, and the top and bottom, carrying electric charges of +2/3 and -1/2 
% respectively. the quarks also carry a colour charge, where only colour-neutral 
% composite hadron particles, combinations of quarks, are seen in nature.

The leptons are also arranged into three generations, constructed from $SU(2)$
doublets of weak isospin $(\nu_L, l)_L$ of left-handed chiral states, and
singlets of $l_R$ right-handed chiral states, where $l$ represents either the
electron ($e$), muon ($\mu$) or tau ($\tau$), each carrying a -1 electrical
charge, with $\nu_L$ being their corresponding electrically neutral neutrinos.

% the leptons again are arrange in three generations, constructed from SU(2) 
% doublets of weak isospin (neutrino L, l)L of left-handed states, and singlets of
% lR right-handed states, where l represents the electron, muon and tau, each with
% -1 electric charge. with vl being their corresponding electrically neutral neutrinos.

Interactions between the fermions are mediated by spin-1 bosons. The strong 
force, $SU(3)$, is represented by eight electrically neutral, massive gluons 
($g$), each carrying a colour charge. The electroweak force, $SU(2)\times U(1)$
is
mediated by the electrically neutral, massless photon ($\gamma$), the massive,
neutral $Z^0$, and the massive, electrically charged $W^{\pm}$ bosons.

Following the local gauge invariance enforcement an $SU(2)\times U(1)$
transformation,
the bosonic mediator of the $U(1)$ symmetry, the $\gamma$, remains massless as
expected, however so do the $W^{\pm}$ and $Z^0$, in disagreement with their
experimentally observed masses. To remedy this, the $SU(2)\times U(1)$ symmetry
is
spontaneously broken the Higgs, $H^0$, a massive, electrically neutral, spin-0
scalar particle, thereby completing the formalism of the Standard Model.

% finally, the electroweak sector, SU(2)xU(1) is spontaneously broken by the 
% massive, neutral, spin-0 scalar boson H0, allowing for massive particles to 
% attain mass. couples in proportion to mass.

% \emph{does it give masses to every massive particle, or just W/Z and this 
% propogates mass?}

\subsection{Failings of the Standard Model}

While the Standard Model has provided many accurate predictions and descriptions
which have been extensively experimentally verified, the theory is considered
incomplete due to some gaps in it's description of the Universe.

The SM provides a QFT description of how the strong, weak and electromagnetic
forces work on a quantum level, however it fails to include any description of
the final fundamental force, gravity. A quantum description of gravity has long
been saught after by particle theorists, in particular to provide an explanation
for it's relative weakness at the EWK scale with respect to the other
fundamental forces.

Following the Big Bang it is believed matter and anti-matter were created in
equal amounts, however given that we exist in a matter dominated universe, there
must exist some physical process by which an asymmetry came to be between the
two types of matter - the so-called `Baryogenesis' process. This could
potentially be described by violation of the Charge-Parity (CP) symmetry of the
universe, however the amount of CP violation described by the Standard Model
does not appear to be significant enough to achieve this.

In the SM, following the local gauge invariance requirement of the electroweak
sector, $SU(2)\times U(1)$, the $SU(2)$ doublets of weak isospin contain
massless neutrino particles. However, experimental observations of neutrino's
oscillating between their various flavours indicate them to have mass, and
therefore be mixing between different mass eigenstates. This is only described
by an extension to the Standard Model - the PMNS mixing matrix REF.

Furthermore, and of particular relevance to the work in this this thesis, are
the Hierarchy Problem and the existence of Dark Matter - both of which will be
described in further detail in the following sections.

\subsubsection{The Hierarchy Problem}

When calculating higher than tree-level contributions to the Higgs mass, $m_h$,
divergent terms arise which make the mass increase up the the Planck mass,
$M_{planck}$ - clearly in disagreement with the experimentally observed
$m_h \approx 125 \gev$.

\subsubsection{Dark Matter/Energy}


%********************************** % Second Section  *************************************
\section{Overview of Supersymmetry}  %Section - 1.2
\label{sec:theory_overview}
Overview. A basic introduction.

Super-multiplets.

simplest form of SUSY would have all particle and sparticles pairs with the same
mass, but due to lack of experimental observation of sparticles, we know the
SUSY must be a broken symmetry

a broken symmetry

R-parity

solution to the heirarchy problem - loop corrections etc

\subsection{The Minimally Supersymmetric Standard Model}
talk about `Complete models'

mention the free parameters of the MSSM (hint at how they're eventually 
constrained within other models)


\subsection{Simplified Model Spectra}
Overview. Signature vs model interpretations.

allow for theorists to easily interpret within their favoured complete model

allow easy comparison between various experimental searches.

include some comments about their comparison with complete models

% \subsubsection{vs `Complete' models}
% \subsubsection{Gluino production and decay - only if needed for T1ttcc}

\subsection{Stop quark production and decay}
go through the various decay channels of the stop in the mStop vs mLSP plane
- which are relevant where

show overview plot from CMS? not limits, just the different decay regions.

\subsection{Mass-degeneracy in SUSY}
Overview of the small deltaM region. Different area of phase space than many
searches - due to soft decay products. experimentally challenging.

currently largely inacessible

stop mass degeneracy line/gap - `the diagonal' (this term is constantly
referenced later, so define here.)

\subsubsection{Relevant Models}
Make link to the above section about stop decay channels

Why only relevant here?
- dark matter constraints - DM co-annihilation etc
- off-shell W

reference the early 4body/compressed region paper

relic density is fulfilled in this region. omegah2

\subsubsection{Phenomenology}
briefly cover this with some poinient plots from early T2cc studies

show plots indicating how soft gen charms are near the diagonal, so are
invisible

will talk later in the signal section about how we gain experimental sensitivity
to this topology (ISR!)

\subsubsection{Current Limits}
\emph{(maybe no need to include, now there are quite a few that are comparable
to
ours...!)}

\subsection{Targetted models}
\emph{Full list with some minor comments and feynman diagrams}

This analysis will search for two such compressed, stop production models,
namely \texttt{T2cc} (\Ttwocc) and \texttt{T2degen} (\Ttwodegen).