%*******************************************************************************
%****************************** Second Chapter *********************************
%*******************************************************************************

\chapter{Theory}
\label{ch:2}

\ifpdf
    \graphicspath{{Chapter2/Figs/Raster/}{Chapter2/Figs/PDF/}{Chapter2/Figs/}}
\else
    \graphicspath{{Chapter2/Figs/Vector/}{Chapter2/Figs/}}
\fi


%********************************** % First Section  *************************************
\section{Introduction}  %Section - 1.1 
\label{sec:theory_intro}

The Standard Model (SM), proposed in 1970s, has long been the most prominent and 
successful description of fundamental particles and their interactions at the
energies currently experimentally accessible. It's power has been proven with the 
theoretical prediction of particles such as the charm and top quarks, and the W
and
Z bosons, prior to their experimental observation. Furthermore, precision 
electroweak measurements have shown impressive levels of 
agreement with SM predictions \cite{ALEPH:2010aa}.

However, despite it's great success, it is known to be valid only for low 
energies, and is missing significant details of both experimentally 
observed and theoretically predicted physical phenomena. As such, extensions to
the SM have been extensively studied, assuming the SM to be the low-energy 
regime of some greater theory. One such strongly theoretically motivated
extension to the SM is Supersymmetry (SUSY).

This chapter outlines the SM model, including it's particle content and basic 
mechanisms, before describing it's shortcomings - the theoretical 
motivation for theories Beyond the Standard Model (BSM). SUSY will then be 
introduced, along with a discussion of the specific models and frameworks used
for interpretation within this analysis.

\section{The Standard Model}
\label{sec:theory_current}

The Standard Model is a renormalisable Quantum Field Theory describing
fundamental matter particles and their interactions via the strong,
weak and electromagnetic forces. It was collaboratively developed throughout the
1960s \cite
{Glashow1961579,PhysRevLett.19.1264,Salam:1968rm,PhysRevLett.30.1346,PhysRevLett.30.1343}, with it's current Lagrangian
based formalism being finalised in the mid-1970s BIG REF (book).

Requiring local gauge invariance of the SM Lagrangian, $\mathcal{L}_{SM}$,
under
transformations generated by universal space-time symmetry groups,
interactions can be described via conserved quantities. The $SU(3)$ group
describes strong force interactions via colour charge, and $SU(2)$ and $U(1)$ to
describe electroweak interactions via weak isospin and hypercharge respectively.

% The requirement that $\mathcal{L}_{SM}$ be invariant under local gauge
% transformations dictates the structure, and crucially the particle content of
% the SM.

Matter particles are described by spin 1/2 fermions, divided into the six quarks
and six leptons. The quarks, are arrange into three generations: the up and
down, the charm and strange, and the top and bottom, each couplet carrying
individual electric charges of +2/3 and -1/2 respectively. The quarks also carry
colour charge, where combinations of quarks form colour-neutral composite
hadron particles as seen in nature.

% matter particles are described by spin 1/2 fermions, divided into the six quarks and six 
% leptons. the quarks are arranged in three generations, the up and down, the charm
% and strange, and the top and bottom, carrying electric charges of +2/3 and -1/2 
% respectively. the quarks also carry a colour charge, where only colour-neutral 
% composite hadron particles, combinations of quarks, are seen in nature.

The leptons are similarly arranged into three generations, constructed from
$SU(2)$
doublets of weak isospin, $(\nu_L, l)_L$, of left-handed chiral states, and
singlets, $l_R$, of right-handed chiral states, where $l$ represents either the
electron ($e$), muon ($\mu$) or tau ($\tau$), each carrying a -1 electrical
charge, and $\nu_L$ being their corresponding electrically neutral neutrinos.

% the leptons again are arrange in three generations, constructed from SU(2) 
% doublets of weak isospin (neutrino L, l)L of left-handed states, and singlets of
% lR right-handed states, where l represents the electron, muon and tau, each with
% -1 electric charge. with vl being their corresponding electrically neutral neutrinos.

Interactions between the fermions are mediated by spin-1 bosons. The strong 
force, $SU(3)$, is represented by eight electrically neutral, massive gluons 
($g$), each carrying a colour charge. The electroweak force, $SU(2)\times U(1)$
is
mediated by the electrically neutral, massless photon ($\gamma$), the massive,
neutral $Z^0$, and the massive, electrically charged $W^{\pm}$ bosons.

Following the enforcemanet of local gauge invariance after an
$SU(2)\times U (1)$ transformation,
the bosonic mediator of the $U(1)$ symmetry, the $\gamma$, remains massless as
expected, however so do the $W^{\pm}$ and $Z^0$, in disagreement with their
experimentally observed masses. To remedy this, the $SU(2)\times U(1)$ symmetry
undergoes Electroweak Symmetry Breaking (EWSB) via the Higgs Mechanism
\cite{PhysRevLett.13.321,PhysRevLett.13.508,PhysRevLett.13.585}, 
leading to the Higgs, $H^0$, a massive, electrically neutral, spin-0
scalar particle, thereby completing the formalism of the Standard Model.

% finally, the electroweak sector, SU(2)xU(1) is spontaneously broken by the 
% massive, neutral, spin-0 scalar boson H0, allowing for massive particles to 
% attain mass. couples in proportion to mass.

% \emph{does it give masses to every massive particle, or just W/Z and this 
% propogates mass?}

\subsection{Failings of the Standard Model}

While the Standard Model has provided many accurate predictions and descriptions
which have been extensively experimentally verified, the theory is considered
incomplete due to gaps in it's description of the Universe.

The SM provides a QFT description of how the strong, weak and electromagnetic
forces work on a quantum level, however it fails to include any description of
the final fundamental force, gravity. A quantum description of gravity has long
been sought after by particle theorists, in particular to provide an explanation
for it's relative weakness at the EWK scale with respect to the other
fundamental forces.

Following the Big Bang it is believed matter and anti-matter were created in
equal amounts, however given that we exist in a matter dominated universe, there
must exist some physical process by which an asymmetry came to be between the
two types of matter - the so-called `Baryogenesis' process. This could
be described by violation of the Charge-Parity (CP) symmetry,
however the amount of CP violation described by the Standard Model
does not appear to be significant enough to achieve this REF.

In the SM, following the local gauge invariance requirement of the electroweak
sector, $SU(2)\times U(1)$, the $SU(2)$ doublets contain
massless neutrino particles. However, experimental observations of neutrino
oscillation between their various flavours indicating them to have mass
\cite{PhysRevLett.81.1562,PhysRevLett.89.011302}, and
therefore be mixing between different mass eigenstates. This is only described
by extending the Standard Model to include neutrino mixing, described by the
PMNS matrix \cite{Altarelli:2002hx}.

Furthermore, and of particular relevance to the work in this this thesis, are
the Hierarchy Problem and the existence of Dark Matter - both of which will be
described in further detail in the following sections.

\subsubsection{The Hierarchy Problem}

% When calculating beyond tree-level diagram contributions to the Higgs mass,
% $m_h$, divergent terms arise which make the mass increase up to values in
% strong disagreement with the experimentally observed $m_h \approx 125 \gev$ REF.

Particles couple to the Higgs field according to Yukawa couplings, $\lambda_f$,
proportionally to the mass of a the particle. For fermions, the Lagrangian
interaction term is
% 
\begin{equation}
\mathcal{L}_\text{Yukawa} = - \lambda_f \bar{\Psi}H\Psi
\end{equation},
% 
where $\Psi$ is a spin-1/2 Dirac field, and $H$ the Higgs field. It can be shown
that Higgs mass corrections due to beyond tree-level interactions with fermion
fields (shown in figure~\ref{quantum_higgs_fermion_loop}) can be written as
% 
\begin{equation}
% \Delta m_H^2 = -\frac{|\lambda_f|^2}{8\pi^2} [\Lambda_{\text{cut}}^2 + \ldots]
\Delta m_H^2 \propto -|\lambda_f|^2 \Lambda_{\text{cut}}^2
\label{eq:higgs_corr_hierarchy}
\end{equation},
% 
where $\Lambda_{\text{cut}}$ is the cut-off energy scale, up to which the SM is
valid, typically taken to be the Planck scale, leading to a quadratically
divergent correction to $m_H$. Such corrections cause significant tension
between the theoretically calculated value of $m_H$ and the experimentally
measured value, $m_H \simeq 125$ \gev \cite{Chatrchyan:2012ufa, Aad:2012tfa},
requiring significant `fine-tuning' to bring in agreement. This disagreement is
known as the Hierarchy Problem.

\begin{figure}[ht!]
\centering
\includegraphics[width=0.47\textwidth]{Figs/feynman/higgs_loop_jaxo.png}
\caption{Quantum self-corrections to the Higgs mass from a fermion loop.}
\label{fig:quantum_higgs_fermion_loop}
\end{figure}

\subsubsection{Dark Matter}
Significant amounts of so called `Dark Matter' (DM) has been indirectly observed
via cosmological methods through the study of galaxy rotation curves and
gravitational lensing REF. For such a large relative abundance to be present, DM
must be massive and weakly interacting. No candidate particle exists in the SM
to describe such matter.

%********************************** % Second Section  *************************************
\section{Overview of Supersymmetry}  %Section - 1.2
\label{sec:theory_overview}
At it's core, Supersymmetry is a new space-time symmetry
between fermions and bosons. An operator of SUSY acting on a fermion 
(resp. boson) state acts to change the spin by a 1/2, producing a super-partner
boson (resp. fermion) state, maintaining electric charge, colour charge and weak
isospin. A theory invariant under such transformations is said to be
supersymmetric.

As a consequence of this new symmetry, SUSY introduces a rich phenomenology of
new particles (sparticles). The simplest implementation of SUSY is the
Minimally
Supersymmetryic Standard Model (MSSM) \cite{Martin:1997ns} - a so-called
`complete' model, containing
~100 free-parameters representing sparticle masses, phases and mixing angles 
\cite{Dimopoulos:1995ju}.
Within the MSSM, the SU(2) doublets and singlets, $(\nu_L, l)_L$ and $l_R$,
have super-partners $(\tilde{\nu}_L, \tilde{l})_L$ and $\tilde{l}_R$. Similarly
for SU(3), $(q, q')_L$ and $q_R$,
have super-partners $(\tilde{q}_L, \tilde{q}')_L$ and $\tilde{q}'_R$. The
spin-1 SM gauge bosons have spin-1/2 gaugino partners, the gluino, winos and
bino, $\tilde{g}$, $\tilde{W}^{\pm}$, $\tilde{W}^0$, $\tilde{B}^0$.
Following EWSB, five higgsinos remain, $h^0$, $H^0$, $A^0$ and
$H^{\pm}$. Finally, the gauginos and the neutral higgsinos mix to give the
neutralinos, $\tilde{\chi}^0_{i=1-4}$, and the winos and charged higgsinos mix
to give the charginos, $\tilde{\chi}^{\pm}_{i=1-2}$.

Given the lack of experimental evidence for SUSY, sparticles must
have higher masses than their SM partners, implying that Supersymmetry is not
an exact symmetry and therefore must be broken by some mechanism, of which many
exist REF REF.



\subsection{Why SUSY?}
The Supersymmetric extension to the SM provides solutions to many of the
original theory's issues, for example the predicted unification of the running
gauge couplings at higher energy scales. Arguably two of the most important
described in the following two sections.

\subsubsection{Hierarchy Problem}
While in the SM the Higgs gains corrections to its mass from fermion-loop
induced self-interactions, the presence of bosonic super-partners to these
fermions provide additional diagrams (figure~REF), leading to oppositely signed
corrective terms,
% 
\begin{equation}
% \Delta m_H^2 = 2 \times \frac{\lambda_S}{16\pi^2}[\Lambda_{\text{cut}}^2 +\ldots]
\Delta m_H^2 \propto + \lambda_S \Lambda_{\text{cut}}^2
\end{equation}
% 
where $\lambda_S$ is bosonic super-partner's Yukawa coupling. The introduction
of these super-partners, and therefore their additional corrective terms,
provides an elegant solution to the hierarchy problem, where quadratic
divergences to $m_H$ are effectively cancelled.

\subsubsection{Dark Matter}
In most supersymmetric theories, a new conserved quantum number $R_P$
is introduced in order to conserve lepton and baryon number
\cite{Farrar1978575}, defined as
% 
\begin{equation}
R_P = (-1)^{3B+L+2s} \Bigg( =
\begin{array}{l} 
+1 \text{ for SM particles}\\ -1 \text{ for SUSY particles}
\end{array}\Bigg)
\end{equation}
% 
where $B$ is Baryon number, $L$ is Lepton number and $s$ is spin. SM particles
have $R_P = +1$, whereas SUSY particles have $R_P = -1$. $R_P$ is a
multiplicative quantum number, therefore sparticles
can only be pair-produced. The subsequent sparticle decay will therefore consist
of a cascade-type decay, with each sparticle decaying to a final state
containing a lighter, kinematically accessible sparticle. This process will
continue until the lightest supersymmetric particle (LSP) is reached, and is
therefore stable. It is the LSP that, within R-Parity conserving models,
provides a stable, weakly interacting dark matter candidate
\cite{Jungman:1995df}.

\emph{WHAT IS ELECTROWEAK BARYOGENESIS???}

\subsection{Current Search Status (better title needed asap)}
In order for SUSY to provide an adequate solution to the hierarchy problem,
sparticle masses are required to be around the \tev scale. This fact, coupled
with strong experimental bounds on constrained models and the discovery of
the Higgs in a compatible mass-region of the MSSM, have led interest
towards `Natural' SUSY models. Such models negate the limits imposed by
experimental constraints by assuming the first and second generation squarks to
be very heavy, while the third generation be around 1 \tev in order to solve
the hierarchy problem without significant fine-tuning REF \cite{Carena:2008rt}.

\subsection{Simplified Model Spectra}
While `complete' supersymmetric models offer a full description of all
aspects of a given theory (e.g. mass spectra and mixing angles), the large
number of free parameters becomes unmanageable for interpretations
of experimental observations. Even in `compressed' models, such as the
Compressed-MSSM (CMSSM) with 5
free parameters \cite{Kane:1993td}, the interpretation phase-space is vast.

In an effort to make interpretations of experimental data simpler for
experimentalists, while remaining universally of use to theorists,
Simplified Model Spectra (SMS) models were developed \cite{PhysRevD.79.075020}.
SMS models consist of simplified
decay scenarios which can later be considered within complete models.
Typically SMS models consist of either squark or gluino pair-production, with a
subsequent decay to a given final state with a 100\% branching fraction,
effectively moving
experimental searches away from `model-specific' interpretations and
towards more `signature-specific' interpretations \cite{PhysRevD.88.052017}.

SMS models generally contain only two free parameters, normally the mass of a
produced mother particle, $m_{\text{mother}}$, and the mass of the LSP,
$m_{LSP}$. Therefore interpretations can be made over a simple 2-dimensional
parameter space, making for trivial experimental comparisons.

% \subsection{Stop quark production and decay}
% go through the various decay channels of the stop in the mStop vs mLSP plane
% - which are relevant where

SMS models prove to be an excellent tool for the interpretation of results
within the bounds of naturalness. Many SMS models consider third generation
squark production, allowing for direct limits to be placed on the dominant
decays of stop and sbottom squarks.

\subsection{Mass-degeneracy in SUSY}
SUSY models with mass hierarchies spanning a small mass range are said to be
`compressed'. When considered in terms of a simplified model scenario, this
implies the pair-produced mother particle is close in mass to that of the LSP.
The
smaller the mass difference, \deltam ($= m_{mother} - m_{LSP}$), the greater the
experimental challenge, as visible
decay products become softer and typically out of analysis acceptance.

In general, decay channels of the $\tilde{t}$ are dominated by $\sTop\ra
bW\chipm$ and
$\sTop\ra t\chiz$. However, when \deltam < W-boson
mass and a present W would become off-shell, $\tilde{t}$ decay branching
fractions
become
dominated by the two-body decay $\sTop\ra c\chiz$, and the four-body decay
$\sTop\ra bf\bar{f}'\chipm$ \cite{Boehm:1999tr}. In this region, these two decay
channels can provide a dark-matter relic density consistent with cosmological
observations (e.g. from WMAP \cite{Spergel:2003cb}), where annihilation cross
sections (of what?)
are mediated by co-annihilation with the light stop proposed by naturalness
\cite{Balazs:2004bu}.

% \subsubsection{Phenomenology}
% briefly cover this with some poinient plots from early T2cc studies

% show plots indicating how soft gen charms are near the diagonal, so are
% invisible

% will talk later in the signal section about how we gain experimental sensitivity
% to this topology (ISR!)

% \subsubsection{Current Limits}
% \emph{(maybe no need to include, now there are quite a few that are comparable
% to
% ours...!)}

\subsubsection{Targetted models}
This analysis will therefore focus on the two $\tilde{t}$ decays, relevant in
this
highly-compressed region of SUSY phase space, namely \texttt{T2cc}
(\Ttwocc) and \texttt{T2degen} (\Ttwodegen), shown in
figure~\ref{fig:model_feynman}.

\begin{figure}[h!]
  \centering
  \begin{subfigure}[b]{0.46\textwidth}
    \includegraphics[width=\textwidth]{Figs/feynman/T2cc_feynman.pdf}
    \caption{\texttt{T2cc}}
    \label{fig:t2cc_feyn}
  \end{subfigure}
  \begin{subfigure}[b]{0.46\textwidth}
    \includegraphics[width=\textwidth]{Figs/feynman/T2DegenerateStop_feyn.png}
    \caption{\texttt{T2degen}}
    \label{fig:t2degen_feyn}
  \end{subfigure}
  \caption{Feynman diagrams of the two SMS models considered.}
  \label{fig:model_feynman}
\end{figure}